\documentclass[a4paper,10pt]{article}
\usepackage[utf8]{inputenc}
\usepackage[italian]{babel}
\usepackage[a4paper]{geometry}
\usepackage[linkbordercolor ={0 1 0}]{hyperref}



\usepackage{amsmath}
\usepackage{amssymb}
\usepackage{amsfonts}
\usepackage{pstricks}

\usepackage{environ}
\usepackage{mdwlist}
 
\title{Programmi ed esercizi per il Senior}
\date{}
%\abstractname{asas}


\topmargin -2cm
\oddsidemargin -1cm
\textwidth 17.5cm
\textheight	25.5cm
\parindent=0mm

\newif\ifsoluzioni 
 \NewEnviron{sol}{\ifsoluzioni \textit{Soluzione:} \BODY\fi} 
 
 \soluzionitrue
 %\soluzionifalse
 %per mostrare le soluzioni, scegliere \soluzionitrue, altrimenti l'altro
 
 
 %questo nuovo environmente serve per evidenziare lo scheletro della lezione 
 \NewEnviron{short}{
 \center
 \parbox{14cm}{\large \BODY \normalsize}
 
 \vspace{0.3cm}
}

\begin{document}
\maketitle

Questi programmi sono pensati per essere delle linee guida per chi prepara le lezioni del senior, in maniera da sapere gli argomenti standard da spiegare. Non dev'essere un ``copincolla'' da qui, piuttosto un modo per prendere spunto, essere certi di non dimenticare nulla e fissare un livello coerente di prerequisiti.

\bigskip

Le parti contrassegnate con $\bigstar$ sono in più.

\vspace{0.7cm}
Considerazioni e TODO: Una cosa utile sarebbe creare dei programmi unificati per argomento, come è stato fatto per i complessi presenti in G2-Medium. 

In questo modo guardando il programma si può decidere dove mettere la linea di demarcazione tra ``Prerequisiti e pillole'', ``Basic'', ``Medium''. Sarebbe appunto utile per argomenti tecnici come complessi, trigonometria, vettori; ma anche applicabile a sintetica (per esempio Ceva attualmente è prerequisito per il senior).\\

I programmi medium sono abbastanza solidi, quelli del basic invece sono un po' a caso. Ci sta visto che devono essere principalmente esercizi e applicazioni, si può pensare a renderli più delle ``linee guida''. In particolare andrebbero sistemati meglio GB-1 e GB-3.


\clearpage

\section{GB - 1 [Metodi Algebrici]}
\subsection{Programmi}
\begin{short}
 Luoghi di punti con la geometria analitica (es: apollonio, luogo degli ortocentri) e scelte opportune di coordinate; distanze con i prodotti scalari e scrittura di vari punti con i vettori; rette e circonferenze con i complessi (e corde e tangenti); applicazioni della trigonometria.
\end{short}



\vspace{0.3cm}
\large{\textbf{Versione estesa - Senior 2019}}\normalsize
\begin{itemize}
 \item Ricapitolazione veloce di analitica. Il piano cartesiano è formato da coppie di punti $(x,y)$. L'equazione di una retta è $y=mx+k$ o $Ax+By+C=0$. Il coefficiente angolare è $m$ e indica la pendenza di una retta; se passa per $(x_1,y_1),(x_2,y_2)$ il coefficente angolare è $m=\frac{y_2-y_1}{x_2-x_1}$. Due rette sono perpendicolari se il prodotto dei coefficienti angolari fa $-1$. Equazione di un cerchio\\
 Più che mostrare/dimostrare le formule, è meglio svolgere qualche esercizio spiegandole sul momento e applicandole direttamente.

\item \emph{Esercizio} Fissati due vertici di un triangolo e facendo variare il terzo sulla circonferenza circoscritta (fissata), qual è il luogo dei punti percorso dall'ortocentro?

%\begin{sol}$B=(0,0),C=(2,0),O=(1,l),A=(a,b)$. 
%$m_{AC}=\frac{b}{a-2}$, $m_{BH}=\frac{2-a}{b}$. $H=(a,\frac{2-a}{b}%a)$.
%Circocerchio: $x^2-2x+y^2-2yl$, sostituisco e ho $H=(a,b-2l)$. Quindi è lo stesso cerchio spostato in basso di $2l$.
%\end{sol}


\item \textbf{Trigonometria} Ricapitolazione del teorema del seno e del coseno. Tanti esempi e applicazioni: Teorema di Stewart, calcolo di segmenti notevoli in un triangolo, ceviane nestate. 

\emph{Esercizio} IMO SL 2015 G1


\item \textbf{Vettori} Prodotto scalare e uso per calcolare distanze, per esempio lunghezze notevoli in un triangolo. Esercizi:

Engel 12 - E6,E7. Teorema di Varignon.


\end{itemize}
\textbf{Complessi}\\
\begin{itemize}
\item Sono numeri della forma $a+bi$, con $a,b$ reali e $i$ tale che $i^2=-1$. Si rappresentano nel piano complessi in maniera simile al piano cartesiano. Parte reale e immaginaria. Somma e moltiplicazione di numeri complessi, (divisione).\\
Scrittura in forma polare, passaggio da forma polare a cartesiana. Coniugato. Moltiplicare=ruotare. 

\textbf{NOTA} Si guardi il programma di G2-Medium per una traccia più approfondita

\item \textbf{Teorema di Napoleone} Costruiti tre triangoli equilateri sui lati di un triangolo $ABC$, dimostrare che i tre centri formano un triangolo equilatero

\item \textbf{Teorema di Vecten} Costruiti sui lati di $ABC$ tre quadrati con centro $O_A,O_B,O_C$, dimostrare che $O_BO_C\perp AO_A$

\item \textbf{IMO 1998-5}


\item Problemi di gare, in cui una furba scelta delle coordinate porta ad una soluzione abbordabile.
\end{itemize}

\subsection{Esercizi}

\textbf{Analitica}
\begin{enumerate}
\item  Dimostrare la formula distanza di un punto di coordinate $(p,q)$ dalla retta di equazione $Ax+By+C=0$:

$$\text{distanza}=\frac{Ap+Bq+C}{\sqrt{A^2+B^2}}$$

\item  Se $\Gamma$ è un cerchio di equazione $f(x,y)=x^2+y^2+ax+by+c=0$ e $P=(s,t)$ un punto del piano, la potenza di $P$ rispetto a $\Gamma$ si ottiene sostituendo $s,t$ nell'equazione della circonferenza:\\
$$\text{Pow}_P(\Gamma)=f(s,t)=s^2+t^2+as+bt+c$$

\item Formule di sdoppiamento

\item Dati due punti $A$ e $B$, trovare il luogo di punti $C$ tali che $AC/BC=\lambda$ costante fissata.\\
\begin{sol}
 Pongo $A=(0,0),B=(1,0)$ e wlog $\lambda < 1$. $\frac{\sqrt{x^2+y^2}}{\sqrt{(x-1)^2+y^2}}=\lambda$.\\
 $\lambda^2 x^2 + \lambda^2 y^2 =x^2-2x+1+y^2$\\
 $ x^2+y^2+\frac{2}{\lambda^2-1}x - \frac{1}{\lambda^2-1}=0$
 è un cerchio centrato in $(-\frac{1}{\lambda^2-1},0)$.
\end{sol}

\item \textbf{Luogo degli ortocentri} Sia $\Gamma$ una circonferenza fissa e sia $BC$ una corda. Sia $A$ un punto su $\Gamma$ e sia $H$ l'ortocentro di $ABC$.\\
Determinare al variare di $A$ su $\Gamma$ il luogo geometrico descritto da $H$.


\suspend{enumerate}
\textbf{Vettori}
\resume{enumerate}
\item \textbf{Engel 12 - E6} $ABCD$ quadrilatero, $AC\perp BD$ se e solo se $AD^2+BC^2=AB^2+CD^2$.\\
\begin{sol}
 In vettori, si vuole $(A-C)\cdot (B-D) = 0$ sse $(A-B)^2+(C-D)^2=(A-D)^2+(B-C)^2$.
\end{sol}

\item \textbf{Engel 12 - E7} $ABCD$ quadrilatero, $M,N,P,Q$ punti medi di $AB,BC,CD,DA$. $AC\perp BD$ se e solo se $MN=PQ$.\\
\begin{sol}
 $M=(A+B)/2$, $MN=PQ$ si traduce come $(\frac{A+B}{2}-\frac{C+D}{2})^2=(\frac{B+C}{2}-\frac{A+D}{2})^2$, facendo i conti esce la stessa cosa.
\end{sol}

\item \textbf{Teorema di Varignon} $ABCD$ quadrilatero, $M,N,P,Q$ punti medi di $AB,BC,CD,DA$. Allora $MNPQ$ è un parallelogramma.



\suspend{enumerate}
\textbf{Trigonometria}
\resume{enumerate}

\item  Calcolare, in termini dei lati e degli angoli del triangolo $ABC$, le seguenti lunghezze: 

$$AH, HH_a , BH_a , H_bH_c , OM_a , OH, AI, IA' , IO$$
dove $H$ è l’ortocentro, $O$ è il circocentro, $I$ l’incentro, $H_a$ la proiezione di $H$ su $BC$ (e similmente
sono definiti $H_b$ e $H_c$ ), $M_a$ il punto medio di $BC$, $A'$ il punto medio dell’arco $BC$ che non contiene
$A$ nella circonferenza circoscritta ad $ABC$.

\item \textbf{Teorema di Stewart} Sia $ABC$ un triangolo e $P$ un punto sul lato $BC$. Dimostrare la seguente formula: 
$$AB^2\cdot PC + AC^2 \cdot BP = BC \cdot BP \cdot PC + AP^2 \cdot BC$$\\
Caso particolare: lunghezza della mediana è $AM=\frac{1}{2}\sqrt{2b^2+2c^2-a^2}$.

\item \textbf{Ceviane nestate} Sia $ABC$ un triangolo, $AD,BE,CF$ ceviane concorrenti e $P,Q,R$ sui lati di $DEF$ in modo che $DP,EQ,FR$ concorrano anch'esse. Allora anche $AP,BQ,CR$ concorrono.

\item  Triangolo $ABC$, sia $Y$ punto su $BC$ tale che $AY=CY$, sia $Z$ sul segmento $AY$ in modo che $AB=CZ$ e infine sia $X=CZ\cap AB$.\\
Dimostrare che $BXYZ$ è ciclico.\\
\begin{sol}
 Let $\angle CAY = \angle ACY = \alpha$, $\angle BAY = \beta$.
By law of sines in $\triangle ACZ$ and $\triangle ABY$:
$\frac{CZ}{\sin \alpha} = \frac{AC}{\sin \angle AZC}$ and $\frac{AY}{\sin(2\alpha+\beta)} = \frac{AB}{\sin 2\alpha} \Rightarrow \frac{CZ \cdot AY}{\sin \alpha \sin (2\alpha + \beta)} = \frac{AC \cdot AB}{\sin \angle AZC \sin 2\alpha} \Rightarrow \frac{AY}{\sin (2\alpha + \beta)}=\frac{AC \sin \alpha}{\sin \angle AZC \sin 2\alpha}$.
But in $\triangle ACY$ we have $AC = \frac{AY \cdot \sin 2\alpha}{\sin \alpha}$. So previous equation implies that $\frac{AY}{\sin(2\alpha + \beta)} = \frac{AY}{\sin \angle AZC}$.
Therefore $\sin \angle AZC = \sin(2\alpha + \beta)$. If $\angle AZC = 180^{\circ}-2\alpha-\beta$, then $\angle ACZ = \alpha + \beta$, which is impossible. So $\angle XZY = \angle AZC = 2\alpha + \beta = 180^{\circ}-\angle XBY$, which completes the proof.
\end{sol}

\item \textbf{Germania BWM 2003, Round 1 - 3} Sia $ABCD$ un parallelogramma, si prendano $X$ sul lato $AB$ e $Y$ su $BC$ in modo che $AX=CY$. Sia $T=AY\cap CX$. Dimostrare che $DT$ biseca l'angolo $\widehat{ADC}$.\\
\begin{sol}
 Teorema dei seni su $AXT$ e $BYT$ per ottenere $\frac{\sin\widehat AXT}{AT}=\frac{\sin\widehat ATX}{AX}=\frac{\sin\widehat BTY}{BY}=\frac{\sin\widehat BYT}{BT}$.
 Poi $\widehat AXT+\widehat TCD=180$, quindi i seni uguali. Si conclude facendo teorema dei seni su $\triangle ATD$ e $\triangle CTD$.\\
 Sintetica: $AY\cap CD=Z$, similitudine e teorema della bisettrice.
\end{sol}





\suspend{enumerate}
\textbf{Complessi}
\resume{enumerate}

\item (\emph{Proposto come fatto in GM1 senza soluzione}) \textbf{Allineamento} $A,B,C$ sono allineati se e solo se 
$$\frac{a-c}{b-c}=\frac{\overline{a}-\overline{c}}{\overline{b}-\overline{c}}$$

\item (\emph{Proposto come fatto in GM1 senza soluzione}) \textbf{Perpendicolarità} $AC\perp BC$ se e solo se 
$$\frac{a-c}{b-c}= - \frac{\overline{a}-\overline{c}}{\overline{b}-\overline{c}}$$


\item \textbf{Eserciziario Senior 17, G2 - 10} (\emph{Proposto in GB2})

\item \textbf{Teorema di Napoleone} Sia $ABC$ un triangolo e si costruisca un triangolo equilatero su ciascuno dei lati di $ABC$, esterno ad esso. Siano $O_A,O_B,O_C$ i centri dei tre triangoli. Dimostrare che:
\begin{itemize}
\item $O_AO_BO_C$ è un triangolo equilatero.
\item le rette $AO_A,BO_B,CO_C$ concorrono.
\end{itemize}


\begin{sol} Siano $A_1,B_1,C_1$ i vertici dei triangoli equilateri. $BA=\sqrt{3}BO_C$ e analogamente per gli altri lati. Una rotazione di 30 centrata in $B$ manda $BO_C$ in $BA$ e $BO_A$ in $BA_1$. Il triangolo $BO'_CO_A'$ è simile a $BAA_1$, quindi per Talete $O_CO_A=O'_CO'_A=\frac{1}{\sqrt{3}}AA_1=O_BO_C$. Quindi i tre lati sono uguali.

Si fa benissimo in complessi (per G1). Su cut-the-knot ci sono tanti approcci.\cite{napoleoncomplex}
\end{sol}
            
\item \textbf{Teorema di Vecten } Sia $ABC$ un triangolo e si costruisca un quadrato su ciascuno dei lati di $ABC$, esterno ad esso. Chiamati $O_A,O_B,O_C$ i centri dei tre quadrati. Dimostrare che:
\begin{itemize}
 \item le rette $AO_A,BO_B,CO_C$ concorrono.
 \item I segmenti $AO_A$ e $O_BO_C$ sono uguali e perpendicolari tra loro
\end{itemize}

\item su Geometry in Figures \cite{engeofigures}, capitolo 9 ci sono fatti sparsi con quadrati/triangoli costruiti sui lati.


\suspend{enumerate}
\textbf{Vario}
\resume{enumerate}




\item \textbf{Eserciziario Senior 2017, G1 - 12} Sia dato un triangolo $ABC$ e si fissino i punti $A'$,$B'$,$C'$ sui lati opposti ai vertici $A$, $B$,
$C$, rispettivamente, in modo che le rette $AA'$ , $BB'$ , $CC'$ siano concorrenti in un punto $P$ interno al triangolo. Sia $d$ il diametro del cerchio circoscritto al triangolo $ABC$, e sia $S'$ l’area del triangolo $A'B'C'$.\\
Dimostrare che $d \cdot S' = AB'\cdot BC'\cdot CA'$.

\item In un triangolo $ABC$, trovare il punto $P$ che minimizza la quantità $AP^2+BP^2+CP^2$.

\begin{sol}
 $AP^2=(x_A-x_P)^2+(y_A-y_P)^2$, quindi posso risolvere separatamente il trovare la coordinata $x_P$ e $y_P$. 
 Per $x_P$ bisogna minimizzare $3x_P^2-2x_P(x_A+x_B+x_C)$, che è una parabola con zeri $x=0, \frac{2}{3}(x_A+x_B+x_c)$, quindi $x_P=\frac{x_A+x_B+x_c}{3}$ e $P$ è il baricentro
\end{sol}

\end{enumerate}


\subsection{GB - 1, Problemi}
\begin{enumerate}
 \item \textbf{EGMO 2013 - 1} Nel triangolo $ABC$, si prolunghi il lato $BC$ dalla parte di $C$ di un segmento $CD$ tale che $CD=BC$. Si prolunghi poi il lato $CA$ dalla parte di $A$ di un segmento $AE$ tale che $AE= 2CA$.Dimostrare che, se $AD=BE$, allora il triangolo $ABC$ è rettangolo
 
 \item \textbf{IMOSL 1998 - 5} Sia $ABC$ un triangolo, $H$ l'ortocentro, $O$ il circocentro e $R$ il raggio della circonferenza circoscritta. Sia $D$ il simmetrico di $A$ rispetto a $BC$, $E$ il simmetrico di $B$ rispetto $AC$ e $F$ il simmetrico di $C$ rispetto $AB$.\\
 Dimostrare che $D,E,F$ sono allineati se e solo se $OH=2R$.

 \begin{sol}
 Complessi con circoscritta = circonferenza unitaria
\end{sol}

\item \textbf{Yugoslavia TST 1992} All'esterno del triangolo $ABC$ sono costruiti i quadrati $BCDE,CAFG,ABHI$. Siano $P,Q$ punti tali che $GCDQ$ e $EBHP$ siano parallelogrammi.\\
Dimostrare che il triangolo $APQ$ è isoscele e rettangolo.

 
 
 \item \textbf{IMOSL 2015 - G1} Sia $ABC$ un triangolo acutangolo con ortocentro $H$. Sia $G$ il punto per cui il quadrilatero $ABGH$ risulta un parallelogrammo. Sia $I$ il punto della retta $GH$ per cui
la retta $AC$ biseca il segmento $HI$. Sia $J$ l’ulteriore intersezione tra la retta $AC$ e la
circonferenza circoscritta al triangolo $GCI$.
Dimostrare che $IJ = AH$.

\begin{sol}
 Sia $M=GH\cap AC$, Teorema dei seni su $\triangle IJM$ da $\frac{\sin\alpha}{IJ}=\frac{\sin IMJ}{IJ}=\frac{\sin IJM}{MH}=\frac{\sin IGC}{MH}$ per la ciclicità di $GCIJ$. Teorema dei seni su $\triangle MAH$ da $\frac{\sin\alpha}{AH}=\frac{\sin CJH}{MH}$. Per la tesi basta dimostrare che $\widehat{CGH}=\widehat{CAH}=90-\gamma$, ma $CHG$ è rettangolo e $CH=c\cdot cotg(\gamma)=HG\cdot cotg(\gamma)$.
\end{sol}

\item \textbf{ITA TST 2016 - 1} Sia $ABCD$ un quadrilatero. Supponiamo che esista un punto $P$ interno al quadrilatero tale che $\angle APD = \angle BPC = 90^{\circ}$ e $PA \cdot PD = PB \cdot PC$. Sia $O$ il circocentro di $\triangle CPD$.\\
Dimostrare che la retta $OP$ passa per il punto medio di $AB$.

\begin{sol}
Trigonometria: 
\end{sol}
 
 
 
 \end{enumerate}




\clearpage

\section{GB - 2 [Trasformazioni]}
\subsection{Programmi}

\begin{short}
Isometrie: Traslazione, Simmetria, Rotazione. Similitudine. Scrittura di queste trasformazioni in complessi. [$\bigstar$ Affinità]. Applicazioni dell'omotetia. 
Inversione Circolare. Inversione + Simmetria in un triangolo.
\end{short}

\vspace{0.3cm}
\large{\textbf{Versione estesa - Senior 2019}}\normalsize
\begin{itemize}

\item \textbf{Isometrie. }Le isometrie sono trasformazioni che conservano la distanza. Le figure mantengono la stessa forma: le rette vanno in rette, circonferenze in circonferenze, poligoni in poligoni, gli angoli mantengono la misura. 

Le isometrie più importanti sono traslazione, riflessione e rotazione. 

La traslazione si definisce con un vettore $\vec{v}$, che manda ogni punto $P$ in $P+\vec{v}$ (in cartesiane e in complessi).

La rotazione si definisce tramite un centro $C$ e un angolo $\alpha$ tra 0 e 360.  In complessi, se il centro è l'origine, il punto z viene mandato in $z\cdot e^{i\alpha}$; se il centro è un altro punto, allora bisogna fare una traslazione, rotazione e traslare indietro: $z\rightarrow (z-c)\cdot e^{i\alpha}+c$.

La riflessione si definisce tramite una retta $r$, ogni punto viene mandato nel suo simmetrico rispetto a questa retta. La riflessione inverte l'orientazione a differenza della traslazione e della rotazione.

Come per la rotazione, per scrivere in complessi la riflessione si compongono tre trasformazioni: si sceglie un punto $c$ sulla retta e sia $\alpha$ l'angolo che forma con l'asse reale, allora $z$ va in $\overline{(z-c)e^{-i\alpha}}\cdot e^{i\alpha}+c=\overline{(z-c)}\cdot e^{2i\alpha}+c$.

\textit{Esempio easy:}  $ABC$ triangolo, $H$ ortocentro, $AH$ interseca $BC$ in $D$ e la circonferenza circoscritta in $N$. Dimostrare $DH=HN$.

\vspace{0.3cm}
\item $\bigstar$ Fatti sparsi su isometrie 

1) ogni isometria è composizione di al massimo tre riflessioni.

2)Si possono dividere in due gruppi, a seconda se mantengono l'orientamento oppure no. Quelle che mantengono l'orientamento sono traslazione e rotazione, quelle che lo invertono sono riflessione e glissoriflessione(=traslazione lungo una retta e riflessione lungo quella retta), questa sono tutte le isometrie possibili

3) rotazione di $\alpha$+rotazione di $\beta$ = rotazione di $\alpha+\beta$ se $\alpha+\beta \neq 0$, altrimenti è traslazione. Traslazione+rotazione di $\alpha$=rotazione di $\alpha$ con un altro centro. analogamente per rotazione+traslazione]

\item \textbf{Omotetia} Il concetto e le proprietà dell'omotetia dovrebbero essere già noti dalle pillole, qui è utile fare tanti esempi ed esercizi.


\item $\bigstar$ \textbf{Affinità}


\vspace{0.4cm}

\item \textbf{Inversione.} A ogni punto $P$ associa $P'$ tale che $OP\cdot OP'=R^2$. Costruzione con le tangenti (per punto esterno) e al contrario per punto interno. È involutiva, scambia interno ed esterno, i punti sulla circonferenza di inversione rimangono gli stessi. 

Le rette per l'origine rimangono rette per l'origine, circonferenze per l'origine diventano rette non per l'origine [questo si può dimostrare], circonferenze non per l'origine diventano circonferenze non per l'origine. Calcolo di $A'B'=\frac{AB\cdot R^2}{OA \cdot OB}$, dire che $OAB$ e $OB'A'$ sono simili. L'inversione conserva gli angoli tra le curve, ma non gli angoli tra punti.

\textit{Esempio} Teorema di Tolomeo.

In complessi, l'inversione nell'origine di raggio $R$ manda $z$ in $R^2\cdot \overline{z}^{-1}$. 


\item $\bigstar$ \textbf{Invarianza di circonferenze per inversione, circonferenze ortogonali} Si può fare un ponte tra potenze e inversione: una circonferenza $\gamma$ è invariata per inversione in $O$ di raggio $R$ se $pow_{\gamma}(O)=R^2$, cioé se le due circonferenze sono ortogonali. Per esempio $\gamma$ circoscritta ad $ABC$, $P$ è l'intersezione della tangente in $A$ con $BC$, allora l'inversione in $P$ di raggio $PA$ scambia $B$ e $C$ e di conseguenza lascia invariata $\gamma$ ]

\item \textbf{Inversione + Simmetria}
Dato un triangolo $ABC$, si può fare un'inversione di centro $A$ e raggio $\sqrt{AB\cdot AC}$ unita alla simmetria rispetto alla bisettrice di $\widehat{BAC}$. Proprietà della trasformazione: scambia $B$ e $C$, la retta $BC$ con la circoscritta a $ABC$.\\

\item Se c'è una retta $MN$ parallela a $BC$, si può fare un'inversione di raggio $\sqrt{AB\cdot AN}=\sqrt{AM\cdot AC}$ che scambia $B$ con $N$ e $C$ con $M$.

\end{itemize}

\subsection{Esercizi}
\begin{enumerate}
       \item Fare i conti per traslazioni, rotazioni, riflessioni, inversione in complessi. 
       
       \suspend{enumerate}
       \textbf{Simmetrie}
       \resume{enumerate}

       \item \textbf{Problema di Fagnano} Sia $ABC$ un triangolo acutangolo, $P,Q,R$ tre punti variabili sui lati $BC,AC,AB$ rispettivamente. Per quale posizione dei tre punti il perimetro del triangolo $PQR$ è minimo?
       
      \begin{sol}Sia $P_1$ il simmetrico di $P$ rispetto $AB$ e $P_2$ rispetto $AC$. Allora il perimetro $PR+RQ+QP=P_1R+RQ+QP_2$ è la lunghezza della spezzata $P_1RQP_2$, fissato P è minimo se i quattro punti sono allineati. Inoltre $\widehat{P_1AP_2}=2\cdot\widehat{BAC}$ e $AP_1=AP_2=AP$, quindi $P_1P_2=AP \sin{\widehat{BAC}}$ è minimo quando è minimo $AP$. Quindi $P$ è piede dell'altezza da $A$, e in tale caso anche $Q$ e $R$ lo sono
       \end{sol}
      
       \suspend{enumerate}
       \textbf{Rotazioni}
       \resume{enumerate}
      
  \item Teoremi di Napoleone e Vecten (enunciato in G1), la parte che si fa con le rotazioni è dimostrare che il triangolo dei centri è equilatero.

      
      \item \textbf{Eserciziario Senior 17, G2 - 10} Siano $ABMN$ e $BCQP$ i quadrati costruiti sui lati $AB$ e $BC$ di un triangolo, esternamente al triangolo stesso.
      Dimostrare che i centri di tali quadrati ed i punti medi di $AC$ e $MP$ sono i vertici di un quadrato.
      
      \begin{sol}sia $L$ il centro di $ABMN$ e $R$ di $BCQP$, $J$ il punto medio di $AC$. $LJ$ è parallelo a $MC$ per omotetia di centro $B$ e fattore 2, inoltre dopo una rotazione di $90^{\circ}$ va in $BP$ che è parallelo a $JR$. Da questo si deduce che $LJ=JR$ e sono ortogonali. Analogamente si fa per il punto medio di $MP$
      \end{sol}
      
       \suspend{enumerate}
       \textbf{Omotetia}
       \resume{enumerate}
       
       	\item Sia $ABC$ un triangolo, $\omega$ la circonferenza inscritta tangente a $BC$ in $D$. Sia $M$ il punto medio di $BC$ e $E$ il simmetrico di $D$ rispetto a $M$. Sia $T$ il diametralmente opposto a $D$ in $\omega$. \\
       	Dimostrare che $A,T,E$ sono allineati.
       	
       	\item Siano $\Gamma$ e $\omega$ due circonferenza tangenti internamente in $P$, con $\omega$ all'interno di $\Gamma$. Sia $AB$ una corda di $\Gamma$ tangente a $\omega$ in un punto $T$.\\
       	Dimostrare che $PT$ è la bisettrice di $APB$.
       	
       	
       
       \suspend{enumerate}
       \textbf{Omotetia+Simmetria}
       \resume{enumerate}
       
       \item  \textbf{[Lemma della \textit{simmediana}]} Sia $ABC$ un triangolo inscritto in una circonferenza $\gamma$. Le tangenti a $\gamma$ in $B$ e $C$ si intersecano in $P$.
    
       Mostrare che $AP$ è \textit{simmediana} relativa a $BC$, \textit{i.e.} simmetrica della mediana relativa a $BC$ rispetto alla bisettrice dell'angolo $\angle BAC$.   
       
       \begin{sol}
       Sia $\omega$ la circonferenza di centro $P$ e raggio $PB$. $\Omega\cap AB=D$, $\Omega\cap AC=E$. Per angle chasing $DPE$ allineati è $DE$ è antiparallelo a $BC$, quindi simmetria+omotetia manda $ABC$ in $AED$ e $AM$ in $AP$ in quanto mediane, da cui $AP$ simmediana.
       \end{sol}

       
       \suspend{enumerate}
       \textbf{Inversione}
       \resume{enumerate}
       \item Data l'inversione di centro $O$ e raggio $R$, due punti $A$ e $B$ vanno in $A'$ e $B'$. Determinare la lunghezza di $A'B'$ conoscendo le lunghezze di $OA,OB,AB$.

       
       \item \textbf{Teorema di Tolomeo} Sia $ABCD$ un quadrilatero, $AC\cdot BD\leq AD\cdot BC + AB\cdot CD$ e l'uguale vale sse $ABCD$ è ciclico.
       
       
       	 \item \textbf{[Teorema di Feuerbach]} Mostrare che la circonferenza di Feuerbach è tangente alla circonferenza inscritta e alle circonferenze exinscritte.
    
    \begin{sol} Sia $M$ il punto medio di $BC$ e $D$ e $G$ rispettivamente i punti in cui la circonferenza inscritta e quella ex-inscritta opposta ad $A$ incontrano $BC$. Invertire in $M$ con raggio $MD$.

    Per prima cosa si nota che il piede della perpendicolare e il piede della bisettrice su BC si scambiano perché $MH\cdot MI=MD^2$. Inoltre si mostra passando per la circoscritta che la retta immagine della circonferenza dei nove punti fa un angolo di beta - gamma con BC. Dunque la cfr dei nove punti va nella simmetrica della retta BC rispetto alla bisettrice che tange entrambe le circonferenze inscritta ed exinscritta. Inoltre queste due si scambiano
    \end{sol}
       
       \suspend{enumerate}
       \textbf{Inversione+Simmetria}
       \resume{enumerate}       
      
    
	 \item  \textbf{[Lemma della \textit{simmediana}]} Sia $ABC$ un triangolo inscritto in una circonferenza $\gamma$. Le tangenti a $\gamma$ in $B$ e $C$ si intersecano in $P$.
    
     Mostrare che $AP$ è \textit{simmediana} relativa a $BC$, \textit{i.e.} simmetrica della mediana relativa a $BC$ rispetto alla bisettrice dell'angolo $\angle BAC$.    
    

	

 \end{enumerate}

 
 \subsection{Problemi}
 \begin{enumerate}
    \item \textbf{IMOSL2013 - G2}  Sia $ABC$ un triangolo, e sia $\omega$ la sua circonferenza circoscritta.  Siano $M$ il punto medio di $AB$, $N$ il punto medio di $AC$, $T$ il punto medio dell’arco $BC$ di $\omega$ che noncontiene $A$. La circonferenza circoscritta al triangolo $AMT$ interseca l’asse di $AC$ in un punto $X$ interno al triangolo $ABC$. La circonferenza circoscritta al triangolo $ANT$ interseca l’asse di $AB$ in un punto $Y$ interno al triangolo $ABC$. Le rette $MN$ e $XY$ si intersecano in $K$.\\
    Dimostrare che $KA=KT$.
    
    \begin{sol}La simmetria rispetto all'asse di $AT$ manda $M$ in $X$ e $N$ in $Y$, quindi $K$ rimane fisso e sta sull'asse.
    \end{sol}
    
    \item \textbf{EGMO 2016 - 4} Due circonferenze aventi lo stesso raggio, $\omega_1$ e $\omega_2$ , si intersecano in due punti distinti $X_1$ and $X_2$ . Si consideri una circonferenza $\omega$ tangente esternamente a $\omega_1$ nel punto $T_1$ e internamente a $\omega_2$ nel punto $T_2$.\\ Si dimostri che il punto d’intersezione fra le rette $X_1T_1$ e $X_2T_2$ giace su $\omega$.
    \begin{sol}
    Inversione in $X_1$
    \end{sol}
    
    \item \textbf{Allenamenti EGMO 2019 - G6}
    Dato il triangolo $\Delta ABC$ consideriamo $\omega_B$ la circonferenza passante per A, B e tangente in A al lato AC e, simmetricamente, $\omega_C$ la circonferenza passante per A, C e tangente in A al lato AB. Sia D il punto di intersezione di $\omega_B$ e $\omega_C$ , e sia E il punto sulla retta AD tale che $AD = DE$.\\
    Dimostrare che E sta sulla circonferenza circoscritta al triangolo $\triangle ABC$.

    \begin{sol}
    invertire in A.
    \end{sol}

    \item \textbf{Senior 2013 TF}
    Sia $ABC$ un triangolo. Sia D l’ulteriore intersezione tra la circonferenza passante per C e tangente alla retta AB in A e la circonferenza passante per B e tangente alla retta AC in A.
    Sia E il punto sulla retta AB (diverso da A) tale che $BA = BE$. Sia F l’ulteriore intersezione tra la retta AC e la circonferenza circoscritta al triangolo $ADE$.\\
    Dimostrare che $AC = AF$.

    \begin{sol}
    Invertire in A.
    \end{sol}

    
    
	\item \textbf{IMOSL2011 - G4} Sia $ABC$ un triangolo acutangolo e $\Gamma$ la sua circonferenza circoscritta. Sia $B_0$ il punto medio di $AC$ e $C_0$ il punto medio di $AB$. Sia $D$ il piede dell'altezza da $A$ su $BC$ e sia $G$ il baricentro di $ABC$. Sia $\omega$ la circonferenza passante per $B_0,C_0$ e tangente a $\Gamma$ in un punto $X\neq A$. 
	
	Dimostrare che $D,X,G$ sono allineati.
	
	\begin{sol}
	\emph{nota: la soluzione proposta è basic difficile/medium facile}
	
	Inversione + simmetria di centro A e raggio $\sqrt(AB*AB_0)$, scambia B e $B_0$, C e $C_0$, manda $D$ nel centro di $\Gamma$ O e $\omega$ in una circonferenza per $B$ e $C$ tangente all'immagine di $\Gamma$, $B_0C_0$, in un punto $Y$. Poiché $BC$ e $B_0C_0$ sono paralleli, $Y$ sta sull'asse di $BC$, quindi $OY$ è perpendicolare a $B_0C_0$. 
	
	Sia $T$ l'intersezione delle tangenti a $\Gamma$ per A,X e di $B_0C_0$, è centro radicale di $\Gamma, \omega$ e la circoscritta a $AB_0C_0$. ATXYO è ciclico, l'immagine sotto inversione è la retta XDY. Ora basta mostrare DY intersecato la mediana $AA_0$ è G, ma $AD$ è il doppio di $XA_0$ e sono paralleli, quindi l'intersezione è proprio G.
	\end{sol}
	


	\end{enumerate}

\clearpage

\section{GB - 3 [Sintetica]}
\subsection{Programmi}
\begin{short}
 Circonferenza di Apollonio. Circonferenza di Feuerbach. Simmediana. Segmenti di tangenza di Incerchi/Excerchi, punti di Gergonne e Nagel. Retta di simson. Applicazioni di potenze e assi radicali. 
\end{short}

\begin{itemize}

\item \textbf{Coniugato isogonale.} Prendiamo un punto $P$ in $\triangle ABC$. Sia $A’$ su $BC$ tale che $AA’$ sia la simmetrica di $AP$ wrt bisettrice di $\angle BAC$. Definiamo $B’,C’$ analogamente. Allora $AA’,BB’,CC’$ concorrono in $P’$, il coniugato isogonale di $P$.\\
\textit{Fatto:} $H$ e $O$ sono coniugati isogonali.\\
%Esercizio (per casa?): Si prendano $P,Q$ in $\triangle ABC$ tali che $P,Q$ sono coniugati isogonali. Allora il triangolo pedale di $P$ e il triangolo pedale di $Q$ giacciono su una circonferenza che ha come centro il punto medio di $PQ$.

\item \textbf{Coniugato isotomico.} Prendiamo un punto $P$ in $\triangle ABC$ e tracciamo le ceviane che concorrono in $P$. Siano $A_1,B_1,C_1$ le intersezioni di queste ceviane con i lati $BC,AC,AB$ rispettivamente. Si prenda $A_2$ il simmetrico di $A_1$ rispetto al punto medio di $BC$. $B_2,C_2$ sono definiti analogamente. Allora $AA_2,BB_2,CC_2$ concorrono in $P'$, il coniugato isotomico di $P$.\\

\item \textbf{Punto di Gergonne.} Si prendano $D,E,F$ i punti di tangenza dell'inscritta ad $\triangle ABC$ con i lati $AB,BC,CA$. Le rette $AD,BE,CF$ concorrono nel punto di Gergonne. 

\item \textbf{Punto di Nagel.} Considieriamo un triangolo $\triangle ABC$, con $D$ punto di tangenza dell'$A-$excerchio con $BC$, $E$ punto di tengenza del $B-$excerchio con $AC$ e $F$ punto di tangenza del $C-$excerchio con $AB$. Le rette $AD,BE,CF$ concorrono nel punto di Nagel. In un triangolo, il punto di Nagel è il coniugato isotomico del punto di Gergonne. \\
\textit{Fatto:} In un triangolo $\triangle ABC$ l'incentro è il punto di Nagel del triangolo mediale di $\triangle ABC$.\\

Segmenti di tangenza dell'incerhio e dell'excerchio.

\item \textbf{Retta di Simson.} Prendiamo un punto $P$ sulla circoscritta al triangolo $\triangle ABC$. Le proiezioni di $P$ sui lati di $\triangle ABC$ stanno su una stessa retta. \\
\textit{Fatto:} Se $P,Q$ giacciono sulla circoscritta ad $\triangle ABC$, l'angolo tra le le rette di Simson di $P$ e $Q$ è $\frac 12\angle POQ$. Se $P,Q$ sono diametralmente opposti, l'intersezione delle rette di Simson sta sulla circonferenza di Feuerbach di $\triangle ABC$.


\item \textbf{Potenze, assi radicali e centri radicali.} Il luogo dei punti che hanno la stessa potenza rispetto a due circonferenze è una retta, l'asse radicale delle due circonferenze. Prese $3$ circonferenze $\Gamma_1,\Gamma_2,\Gamma_3$, gli assi radicali $r_{12},r_{23},r_{31}$ concorrono nel centro radicale delle circonferenze.

\end{itemize}


\subsection{Esercizi}
\begin{enumerate}

	\item \emph{[Copiato da GM]} Sia $ABC$ un triangolo con ortocentro $H$ e siano $D$, $E$ e $F$ i piedi delle altezze che cadono sui lati $BC$, $CA$ e $AB$ rispettivamente. Sia $T=EF\cap BC$.
	
	Mostrare che $TH$ è perpendicolare alla mediana condotta da $A$.
	
	\begin{sol}Oltre alla soluzione per inversione, pensavo anche qualcosa con gli assi radicali: il centro radicale delle circonferenze per $AEFH, BCH, ABEF$ è T, quindi TH passa per l'intersezione di AEFH e BCH che chiamo P. Allora APH è retto in quanto diametro.\\
	Poi sia $A'$ il simmetrico di $A$ rispetto $M$ punto medio di $BC$. Allora $BHCA'$ è ciclico e per angoli $\widehat{HPA'}=\widehat{HBA'}=90$.

    inversione di centro A e raggio $AH\cdot HD$. La tesi diventa equivalente a mostrare che la circonferenza per D (immagine di H), per l'intersezione della circoscritta con AEF (immagine di T) e A ha la retta AM come diametro. Questo segue perché in effetti $M,T',A,D$ sono ciclici
	\end{sol}
	
	
	\item Sia $ABC$ un triangolo, $E,F$ i piedi delle altezze su $AC,AB$. Sia $H$ l'ortocentro, $M$ il punto medio di $BC$ e $Q$ l'intersezione più vicina ad $A$ di $HM$ con la circoscritta $\Gamma$. Sia $T=EF\cap BC$. Dimostrare che $T,Q,A$ sono allineati.
	
	\begin{sol} Usare due fatti 1) Il punto Q è l'intersezione di $\Gamma$ con la circonferenza di diametro $AH$ ed è allineato con H,M e il simmetrico di A rispetto O 2) Assi radicali di AQH, ABC, BCEF.
	\end{sol}	
	
	\item Sia $ABC$ un triangolo con $I$ incentro e $I_A$ centro della circonferenza ex-inscritta relativa ad $A$. Sia $\Gamma$ la circonferenza circoscritta ad $ABC$ e sia $M$ il punto medio dell'arco $BC$ non contenente $A$.
	
	Dimostrare che $B,I,C,I_A$ si trovano su una stessa circonferenza di centro $M$
	

	
\begin{sol}Calcolare i segmenti di tangenza di inscritta ed ex-inscritta, poi omotetia in $A$
 
\end{sol}

	
	\item Proprietà varie della circonferenza di Feuerbach. 
\end{enumerate}

\subsection{Problemi}
\begin{enumerate}
    \item \textbf{Polish MO 2018 - 5} Sia $ABC$ un triangolo acutangolo con $AB\neq AC$
    e siano $E,F$ i piedi delle altezze su $AC$ e $AB$. La tangente in $A$ alla circoscritta interseca $BC$ in $P$. La retta parallela a $BC$ passante per $A$ interseca $EF$ in $Q$. 
    
    Dimostrare che $PQ$ è perpendicolare alla mediana passante per $A$ del triangolo $ABC$
    
    \begin{sol}Assi radicali swag: 1) La circonferenza degenere di centro $A$, la circoscritta a $AEF$ e a $BCEF$ hanno $Q$ come centro radicale (in quanto sta su $EF$ per le ultime due e $AQ$ tange la circoscritta $AEF$ per le prime due). 2) $PA^2=PB\cdot PC$, quindi P sta sull'asse radicale tra $A$ e la circoscritta a $BCEF$. Dunque $PQ$ è asse radicale delle due circonferenze ed è perpendicolare alla congiungente dei centri, che è $AM$
    \end{sol}
    
    \item \textbf{RUSSIAN OLYMPIAD 2010} Il triangolo $\triangle ABC$ ha perimetro $4$. I punti $X,Y$ sono sulle semirette $AB,AC$ e sono tali che $AX=AY=1$. I segmenti $BC$ e $XY$ si intersecano in $M$. Dimostrare che o il perimetro di $\triangle ABM$ o il perimetro di $\triangle ACM$ è $2$.
    
    \begin{sol}
    Prendo $U,V$ i simmetrici di $A$ rispetto a $B,C$: sono i punti di tangenza dell'$A-$excerchio. Diciamo che l'excerchio tange $BC$ in $T$. Prendendo circonferenza degenere di centro $A$, la retta $XY$ è asse radicale di quella circonferenza e l'$A-$excerchio. (Quale triangolo ha perimetro $2$ è in base a $T\in MC$ o $T\in BM$)
    \end{sol}
    
    \item \textbf{IMO 2006 - 1} Sia $I$ l'incentro di $\triangle ABC$. Si prenda un punto $P$ interno ad $\triangle ABC$ che soddisfa 
    $$\angle PBA + \angle PCA = \angle PBC + \angle PCB.$$
    Dimostrare che $AP\geq AI$, dove l'uguaglianza vale se e solo se $P$ coincide con $I$.
    
    \begin{sol} 
    Sia $P'$ la seconda intersezione tra $CP$ e $\odot ABC$. $PP'=P'B$ e $\angle BPC = \pi/2 + \alpha/2$, quindi $BPIC$ è ciclico. Il centro di $\odot BPIC$ sta su $AI$, quindi $\angle DIP \leq 90$, da cui $\angle AIP\geq 90$, da cui la tesi.
    \end{sol}
    
    \item \textbf{IMO 2008 - 1} Sia $\triangle ABC$ un triangolo acutangolo con ortocentro $H$. La circonferenza $\Gamma_A$ con centro il punto medio di $BC$ passante per $H$ interseca $BC$ in $A_1,A_2$. Definiamo analogamente $B_1,B_2,C_1,C_2$. Dimostrare che $A_1A_2B_1B_2C_1C_2$ è ciclico.  
    
    \begin{sol}
    Sia $X$ la seconda intersezione di $\Gamma_A$ e $\Gamma_B$. Dimostriamo che $A,X,H$ sono allineati, il che implica $B_1B_2C_1C_2$ ciclico. Il centro della circonferenza è $O$, quindi passa anche per $A_1A_2$.
    \end{sol}
    
\end{enumerate}

\clearpage

\section{GM - 1 [Numeri complessi e coordinate baricentriche]}
\subsection{Programmi}

Numeri complessi nella geometria euclidea. Si assume che si possegga una discreta maneggevolezza con il piano complesso.
Rapido ripasso sulla forma polare dei numeri complessi e significato geometrico delle operazioni.

Condizione di allineamento e scrittura dell'equazione di una retta per due punti. Condizione di parallelismo e scrittura della parallela ad una retta passante per un punto ad essa esterno. Condizione di perpendicolarità e scrittura della perpendicolare ad una retta passante per un punto ad essa esterno. Birapporto fra 4 numeri complessi e condizione di ciclicità.

Equazione di una generica circonferenza. Scelta classica delle coordinate: circonferenza circoscrita $\equiv$ circonferenza unitaria. Punti notevoli nella scelta classica delle coordinate. Esempio di quanto si semplificano i conti: intersezione di due corde generiche. Coordinate $u,v,w$ per l'incentro. 

\vspace{0.5cm}
Definizione di coordinate baricentriche. 

Come verificare l'allineamento di tre punti ed equazione di una retta generica. Intersezione di due rette. Area di un triangolo di cui si conoscono le coordinate dei vertici. Punto all'infinito di una retta. Quando due rette sono parallele?

Punti notevoli e notazione di Conway: baricentro, incentro, ortocentro, circocentro, excentri, nagel, gergonne, lemoine... Coniugati isogonali e coniugati isotomici.

Equazione della circonferenza circoscritta (come coniugato isogonale della retta all'infinito).
Equazione di una circonferenza in posizione generale. Equazione dell'asse radicale fra una circonferenza in posizione generale e la circonferenza circoscritta al triangolo referenziale: relazione di tale equazione con le potenze dei vertici del triangolo referenziale rispetto alla circonferenza in posizione generale. Formula di sdoppiamento per la tangente e la polare.

\vspace{0.3cm}
\large{\textbf{Versione estesa - Senior 2019}}\normalsize

\vspace{0.3cm}
\textbf{Numeri Complessi}:
\begin{itemize}
\item \textbf{Introduzione:} Un numero complesso si scrive come $z=a+bi$, dove $i^2=-1$ e $a,b$ sono numeri reali. Il numero $a$ si dice parte reale e il numero $b$ si dice parte immaginaria. Si può anche scrivere come $z=\rho e^{i\theta}$, dove $\rho>0$ è detto modulo e $0\leq \theta\leq 2\pi$ è detto argomento. Sul piano di Gauss, $\rho$ è la distanza del punto $(a,b)$ dall'origine e $\theta$ è l'angolo formato, in senso antiorario, col semiasse positivo delle $x$.

Identifichiamo un numero complesso con il punto $(a,b)$ del piano di Gauss e alle volte con il vettore che parte dall'origine e arriva ad $(a,b)$.

Per passare dalle coordinate polari a quelle cartesiane: $a=\rho \cos\theta$ e $b=\rho\sin\theta$. Viceversa $\rho=\sqrt{a^2+b^2}$ e $\theta=\arctan{\frac{b}{a}}$. 

\item \textbf{Operazioni:} Per quanto riguarda le operazioni, $z\to z+w$ corrisponde ad una traslazione del vettore $w$; $z\to zw$ - con $w=\rho e^{i\theta}$ corrisponde ad una rotazione in senso antiorario di $\theta$ più una omotetia di centro l'origine e fattore $\rho$; $z\to \bar{z}$ corrisponde ad una simmetria rispetto all'asse reale. 

\item \textbf{Angoli e similitudini:}

\emph{Osservazione}. Sia $g(z)\doteq \frac{z}{\bar z}$. Se $z=\rho e^{i\theta}$, allora $g(z)=e^{2i\theta}$. 

Dati tre numeri complessi (punti) nel piano di Gauss $a$, $b$ e $c$, detto $\theta$ l'angolo $\angle abc$ (ovvero l'angolo di cui ruotare $ab$ in senso antiorario attorno a $b$ perché la retta $ab$ coincida con $bc$ e in più $a$ e $c$ siano dalla stessa parte rispetto a $b$), si ha che esiste un numero reale $\rho>0$ tale che 
$$
c-b=(a-b)\rho e^{i\theta},
$$
dove $\rho$ non è altro che il rapporto fra le lunghezze dei segmenti $\overline{cb}$ e $\overline{ab}$. 

\textbf{Controlla} \emph{Conseguenza 1: Triangoli simili}. Se i triangoli $abc$ e $def$ sono ordinatamente simili, allora 
$$
\frac{c-b}{a-b}=\frac{\overline{cb}}{\overline{ab}}e^{i\angle abc}=\frac{\overline{fe}}{\overline{de}}e^{i\angle def}=\frac{f-e}{d-e},
$$
e vale anche il viceversa.

\emph{Conseguenza 2: Equazione dell'angolo}. Dall'osservazione, se $\theta$ è l'angolo $\angle abc$ si ha 
$$
e^{2i\theta}=g\left(\frac{c-b}{a-b}\right)=\frac{c-b}{a-b}\cdot \frac{\bar a-\bar b}{\bar c-\bar b}.
$$

Dunque per mostrare che $\angle abc=\angle def$ basta mostrare 
$$
\frac{c-b}{a-b}\cdot \frac{\bar a-\bar b}{\bar c-\bar b}=\frac{f-e}{d-e}\cdot \frac{\bar d-\bar e}{\bar f-\bar e}
$$
che è come dire che
$$
\frac{c-b}{a-b}\frac{d-e}{f-e} \qquad \mbox{è reale}.
$$
\item \textbf{Allineamenti, parallelismi e perpendicolarità:}

\emph{Allineamento.} Se $a$, $b$ e $c$ sono allineati, allora $\angle abc=\pi$ e dunque dall'equazione dell'angolo 
$$
\frac{c-b}{a-b}=\frac{\bar c-\bar b}{\bar a-\bar b},
$$
e vale anche il viceversa. 
\emph{Perpendicolarità 1.} Se $ab\perp bc$ allora dall'equazione dell'angolo 
$$
\frac{c-b}{a-b}=-\frac{\bar c-\bar b}{\bar a-\bar b}.
$$
\emph{Parallelismo.} Come esercizio, mostrare che $ab\parallel cd$ se e solo se 
$$
\frac{d-c}{b-a}=\frac{\bar d-\bar c}{\bar b-\bar a}.
$$
\emph{Perpendicolarità 2.} Come esercizio mostrare che $ab\perp cd$ se e solo se 
$$
\frac{d-c}{b-a}=-\frac{\bar d-\bar c}{\bar b-\bar a}.
$$
\item \textbf{Birapporti e ciclicità:}

\emph{Birapporto.} Dati quattro numeri complessi $z_1,z_2,z_3,z_4$ si dice birapporto $[z_1,z_2,z_3,z_4]$ la quantità
$$
\frac{z_1-z_2}{z_3-z_2}\cdot \frac{z_3-z_4}{z_1-z_4}.
$$
Mediante l'equazione dell'angolo è immediato notare che $[z_1,z_2,z_3,z_4] \in \mathbb R$ se e solo se $z_1z_2z_3z_4$ è ciclico.

\item \textbf{Circonferenza unitaria e scelta delle coordinate:}
 
\emph{Circonferenza unitaria e coordinate classiche.} Nel piano cartesiano la circonferenza unitaria ha equazione $z\bar z=1$. Molto spesso nella risoluzione dei problemi è utile settare la circonferenza circoscritta come circonferenza unitaria, dunque tutti i punti su essa soddisfano $z\bar z=1$ e $o$, il circocentro, diviene l'origine degli assi. Siccome vale $h+2o=3g$ in generale, visti i rapporti sulla retta di Eulero, e, sempre in generale, $g=\frac{a+b+c}{3}$, si ottiene che in questa scelta di coordinate $h=a+b+c$.

\emph{Coordinate dell'incentro} L'incentro è più difficile da gestire. In un problema con l'incentro conviene usare la notazione $u,v,w$. Infatti (dare come esercizio), dato un triangolo $abc$, esistono sempre tre numeri complessi $u,v,w$ tali che $a=u^2$, $b=v^2$, $c=w^2$ e l'incentro $i=-(uv+vw+uw)$. \emph{Hint:} Mostrare che esistono $u$, $v$ e $w$ tali che $a=u^2$, $b=v^2$, $c=w^2$ e i punti $-uv$, $-vw$, $-uw$ sono i punti medi degli archi $ab$, $bc$ e $ca$ che non contengono i terzi punti. 

\emph{Dagli esercizi:}

\item \textbf{[Seconda intersezione di due circonferenze in complessi]} Siano dati 4 punti $a, b, c, d$ nel piano complesso che non formano un parallelogrammo.

Mostrare che esiste una e una sola rotomotetia che
manda $a$ in $b$ e $c$ in $d$. Detto $x$ il centro di tale rotomotetia e $\alpha$ il numero complesso che rappresenta la rotomotetia, si ha
$$
x=\frac{ad-bc}{a-b-c+d}
$$
$$
\alpha=\frac{b-d}{a-c}.
$$

Mostrare che l'intersezione delle circonferenze circoscritte a $ABX$ e $CDX$ dove $AC$ e $BD$ sono segmenti non paralleli le cui rette si intersecano in $X$, è il centro della rotomotetia che manda $A$ in $B$ e $C$ in $D$. 

%\textbf{Soluzione:} Sia $x$ il centro della rotomotetia, $\alpha$ il numero complesso che rappresenta la rotomotetia - ovvero l'argomento di $\alpha$ è l'angolo di rotazione e il modulo di $\alpha$ è la ragione della rotomotetia. Se manda $a$ in $b$, allora 
%$$
%b-x = (a-x)\alpha ,
%$$
%e poiché manda $c$ in $d$ si ha anche 
%$$
%d-x = (c-x)\alpha.
%$$
%Dunque per confronto 
%$$
%\frac{b-x}{a-x}=\frac{d-x}{c-x}
%$$
%da cui 
%$$
%(b-x)(c-x)=(d-x)(a-x)\Rightarrow x=\frac{ad-bc}{a-b-c+d}.
%$$

%Svolgendo i calcoli si ha infine

%$$
%\alpha=\frac{b-d}{a-c}.
%$$
 
\emph{Dai problemi:}

\item \textbf{[BMO 2009 - 2]} Sia $MN$ una segmento parallelo al lato $BC$ del triangolo $ABC$, con $M$ sul lato $AB$ e $N$ sul lato $AC$. Le rette $BN$ e $CM$ si incontrano in $P$. Le circonferenze circoscritte a $BMP$ e $CNP$ si incontrano in due punti distinti $P$ e $Q$. 
 
 Mostrare che $\angle BAQ = \angle PAC$.
 
 %\textbf{Soluzione:} Diciamo che $a$ è l'origine del nostro piano di Gauss, mentre $b$ e $c$ sono due generici punti. Visto che $mn\parallel bc$ e $m\in ab$, $m\in ac$ si ha che esiste $\lambda \in \mathbb R$ tale che $m=\lambda b$ e $n=\lambda c$. Essendo $q$ il centro della rotomotetia che manda $m$ in $b$ e $c$ in $n$, allora 
 %$$
 %q=\frac{mn-bc}{m+n-b-c}=\frac{\lambda^2bc-bc}{\lambda b+\lambda c-b-c}=\frac{(\lambda+1)bc}{b+c}.
 %$$
 
 %Per trovare trovare $p$ basterebbe imporre $p\in mc$ e $p\in bn$. \emph{Proporlo come esercizio}. D'altra parte non ce n'è bisogno: infatti noi siamo interessati poi solo all'angolo $\angle CAP$ e dunque non tanto ci servono le coordinate di $P$ quanto capire chi è la retta $AP$, che è la mediana di $ABC$. Dunque possiamo dire che esiste un certo $\eta$ reale tale che 
% $$
 %p=\eta(b+c).
 %$$ 
 
 %Per l'equazione dell'angolo, se $\theta=\angle BAQ$ si ha 
 %$$
 %e^{2i\theta}=\frac{q-a}{b-a}\frac{\bar b-\bar a}{\bar q-\bar a}=\frac{c(\bar b+\bar c)}{\bar c(b+c)},
 %$$
 %mentre se $\theta'=\angle PAC$ si ha 
 %$$
 %e^{2i\theta'}=\frac{c-a}{p-a}\frac{\bar p-\bar a}{\bar c-\bar a}=\frac{c(\bar b+\bar c)}{\bar c(b+c)}.
 %$$
 
 %Da ciò, con un attimo di discussione, si ottiene che $\theta=\theta'$ che implica la tesi. 
 
 \item \textbf{[RMM 2012 - 2]} Sia $ABC$ un triangolo non isoscele e siano $D$, $E$ e $F$ rispettivamente i punti medi dei lati $BC$, $CA$ e $AB$. La circonferenza $BCF$ e la retta $BE$ si intersecano nuovamente in $P$ e la circonferenza $ABE$ e la retta $AD$ in $Q$. Le rette $DP$ e $FQ$ si incontrano in $R$. 
 
 Mostrare che il baricentro $G$ del triangolo $ABC$ giace sulla circonferenza circoscritta al triangolo $PQR$.
 
 %\textbf{Soluzione:} Per mostrare la ciclità è sufficiente mostrare che, detto $\theta=\angle GPD$ e $\theta'=\angle GQF$, si ha 
 %$$
 %\theta=\theta'.
 %$$  
 %Dall'equazione dell'angolo risulta che per fare ciò è sufficiente mostrare
 %$$
 %\frac{d-p}{g-p}\frac{g-q}{f-q}\in\mathbb R.
 %$$
 %Il problema è dunque spostato a trovare i punti $p$ e $q$. Qui usiamo un'osservazione sintetica. Si ha che 
 %$$
 %\angle GDE=\angle GAB = \angle QEG,
 %$$
 %dove la prima è vera per il parallelismo $AB\parallel ED$ e la seconda è vera poiché $ABEQ$ è ciclico. 
 %Analogamente si ha $\angle EQD=\angle GED$ e dunque i triangoli $GDE$ e $GEQ$ sono ordinatamente simili.
 %Dunque, scegliendo $g=0$, risulta, visto che $GD\cdot GQ=GE^2$,
 %$$
 %q=d\frac{|e|^2}{|d|^2}=\frac{e\bar e}{\bar d}
 %$$
 %e analogamente 
 %$$
 %p=\frac{f\bar f}{\bar e}.
 %$$
 %A questo punto 
 %$$
 %\frac{d-p}{g-p}\frac{g-q}{f-q}=\frac{(d\bar e-f\bar f)e\bar e}{(f\bar d-e\bar e)f\bar f}
 %$$
 %e poiché, essendo $g=0$, si ha $d+e+f=0$, la precedente espressione è uguale a 
 %$$
 %\frac{|e|^2}{|f|^2}
 %$$ 
 %che è un numero reale, come si voleva.
\end{itemize}
\textbf{Coordinate baricentriche}:
\begin{itemize}
	\item \textbf{Definizioni:}
	\emph{Terna omogenea}. Con $[x:y:z]$ indico una terna omogenea di numeri non tutti nulli, ovvero $[x:y:z]=[u:v:w]$ se e solo se esiste $k\in\mathbb R\setminus\{0\}$ tale che $u=kx$, $v=ky$ e $w=kz$. 
	
	\emph{Coordinate baricentriche}. Dato $ABC$ un triangolo e $P$ un punto sullo stesso piano di $ABC$, allora le coordinate baricentriche di $P$ sono 
	$$
	[ |BCP|: |CAP|: |ABP| ],
	$$
	dove $|\cdot|$ indica l'area con segno, ovvero è un numero che ha come modulo l'area di $\cdot$ e come segno $+$ o $-$ a seconda che il verso in cui sono scritti i vertici sia lo stesso o l'opposto rispetto a quello in cui sono assegnati $ABC$. 
	\item \textbf{Alcuni punti:} 
	\emph{Notazione di Conway}. Da ora in poi scriveremo
	$$
	S_A\doteq\frac{b^2+c^2-a^2}{2}, \quad S_B\doteq\frac{a^2+c^2-b^2}{2}, \quad 
	S_C\doteq\frac{a^2+b^2-c^2}{2}.
	$$ 
	
	Inoltre indichiamo con $a$ la lunghezza di $BC$, con $b$ la lunghezza di $AC$ e con $c$ la lunghezza di $AB$. Siano $\alpha,\beta,\gamma$ rispettivamente gli angoli in $A$, $B$ e $C$.
	
	I vertici hanno coordinate 
	$$
	A=[1:0:0], \quad B=[0:1:0], \quad C=[0:0:1],
	$$
	i punti medi hanno coordinate 
	$$
	M_{BC}=[0:1:1], \quad \mbox{e cicliche},
	$$
	i piedi delle bisettrici hanno coordinate 
	$$
	D_{BC}=[0:b:c], \quad \mbox{e cicliche},
	$$
	il baricentro ha coordinate 
	$$
	G=[1:1:1],
	$$
	e l'incentro ha coordinate 
	$$
	I=[a:b:c].
	$$
	Per l'ortocentro calcoliamo le aree
	$$
	H=\left[\frac{a}{\cos\alpha}:\frac{b}{\cos b}:\frac{c}{\cos\gamma}\right]=[\tan\alpha:\tan\beta:\tan\gamma]=[S_BS_C:S_CS_A:S_AS_B],
	$$
	dove nell'ultimo passaggio usiamo la notazione di Conway. Per i piedi delle altezze notiamo che in generale le tracce di un punto sono semplici da trovare e dunque
	$$
	H_{BC}=[0:S_C:S_B].
	$$
	Troviamo il circocentro 
	$$
	O=[\sin2\alpha:\sin2\beta:\sin2\gamma]=[a^2S_A:b^2S_B:c^2S_C].
	$$
	dove nell'ultimo passaggio usiamo la notazione di Conway.
	
	\emph{Esercizio:} Il coniugato isogonale di $[x:y:z]$ ha coordinate $\left[\frac{a^2}{x}:\frac{b^2}{y}:\frac{c^2}{z}\right]$ e quello isotomico ha coordinate $\left[\frac{1}{x}:\frac{1}{y}:\frac{1}{z}\right]$. 
	
	\emph{Esercizio:} Mostrare che le coordinate dell'excentro relativo ad $A$ sono
	$$
	I_A=[-a:b:c],
	$$
	le coordinate del punto di Lemoine sono 
	$$
	L=[a^2:b^2:c^2],
	$$
	quelle di Gergonne sono 
	$$
	Ge=\left[\frac{1}{p-a}:\frac{1}{p-b}:\frac{1}{p-c}\right],
	$$
	e di Nagel 
	$$
	Na=\left[p-a:p-b:p-c\right].
	$$
	
	\item \emph{Osservazioni.} Ci sono dei punti che \emph{non stanno sul nostro piano}. Infatti si mostra che $S$, l'area di $ABC$, è uguale a $|BCP|+|CAP|+|ABP|$. Dunque i punti $[x:y:z]$ tali che $x+y+z=0$ sono dei punti \emph{fantasma} sul nostro piano. Diciamo che sono \emph{sulla retta all'infinito}.
	
	Notare che per qualsiasi altra scelta di $[x:y:z]$ con $x+y+z\neq 0$ esiste uno e un solo punto sul piano che ha quelle come coordinate.
	
	Le coordinate $P=[\alpha:\beta:\gamma]$ tali che $\alpha+\beta+\gamma=1$ sono dette coordinate baricentriche esatte di $P$ e sono tali che
	$$
	\vec{P}=\alpha\vec{A}+\beta\vec{B}+\gamma\vec{C}.
	$$
	
	Questo segue molto velocemente dalla definizione delle coordinate che tira in ballo l'area. Dunque per trovare il punto medio fra due punti bisogna usare le coordinate baricentriche esatte - o perlomeno con la stessa somma delle coordinate!
	
	\emph{Esercizio:} trova le coordinate del punto di Feuerbach, ovvero il centro della circonferenza di Feuerbach, che è il punto medio fra $O$ e $H$.
	
	\item \textbf{Rette}: 
	\emph{Equazione.} Una generica retta ha equazione $lx+my+nz=0$ per qualche $l,m,n$ reali. \emph{Pensarci per esercizio}.
	
	\emph{Punto all'infinito.} Il punto all'infinito di questa retta è $[m-n:n-l:l-m]$. 
	
	\emph{Intersezione di due rette.} Due rette $lx+my+nz=0$ e $l'x+m'y+n'z=0$, una non multipla dell'altra, si intersecano sempre nell'unico punto di coordinate omogenee
	$$
	\left[n'm-nm':nl'-n'l:lm'-l'm\right].
	$$
	
	\emph{Rette parallele}. Dunque anche due rette parallele si intersecano sempre, visto il punto precedente. In effetti si intersecano nel punto all'infinito di entrambe.
	
	\emph{Retta per due punti}. Dati due punti $[a:b:c]$ e $[a':b':c']$, la retta passante per questi due punti ha equazione 
	$$
	\begin{bmatrix}
		a & b & c \\
		a' & b' & c' \\
		x & y & z
	\end{bmatrix}
	=0
	$$
	
	Da questa segue anche la condizione di allineamento di tre punti, e la scrittura di una retta parallela ad una data, passante per un punto.
	
	\emph{Rette perpendicolari}. \emph{Da fare come esercizio}
	Il punto all'infinito di una retta perpendicolare a $px+qy+rz=0$ è
	\begin{equation}
	[S_Bg-S_Ch:S_Ch-S_Af:S_Af-S_Bg]
	\end{equation}
	dove $[f:g:h]=[q-r:r-p:p-q]$ è il punto all'infinito della retta.
	
	\emph{Dagli esercizi:}
	\item \textbf{[Coordinate dei vertici del \textit{triangolo tangenziale} in baricentriche]}  Dato un triangolo $ABC$ referenziale in un sistema di coordinate baricentriche, mostrare che la tangente condotta da $A$ alla circonferenza circoscritta ad $ABC$ ha equazione
	\begin{equation}
	c^2y+b^2z=0.
	\end{equation}
	
	Ciclando opportunamente, calcolare le coordinate dei vertici del triangolo tangenziale (\textit{i.e.} il triangolo formato dalle intersezione delle tangenti condotte da $A$, $B$ e $C$ alla circonferenza circoscritta ad $ABC$).
	%\textbf{Soluzione:}
	%Calcoliamo le coordinate di $P$, intersezione della tangente condotta da $A$ alla circoscritta e $BC$. Risulta che 
	%$$
	%P=[0:-b^2:c^2]
	%$$
	%da cui si ottiene subito che la tangente, dovendo passare per $A=[1:0:0]$ è
	%$$
	%c^2y+b^2z=0.
	%$$
	
	%Ciclando si ottiene che i vertici del triangolo tangenziale sono $[a^2:b^2:-c^2]$ e ciclici.
	\emph{Dai problemi:}
	\item \textbf{[MOP 2006]} Sia $ABC$ un triangolo inscritto in una circonferenza $\omega$. $P$ giace su $BC$ in modo tale che $PA$ è tangente a $\omega$. La bisettrice di $\angle APB$ interseca i segmenti $AB$ e $AC$ rispettivamente in $D$ ed $E$ e i segmenti $BE$ e $CD$ si intersecano in $Q$. Supponiamo che la retta $PQ$ passi per il centro di $\omega$. 
	
	Calcolare $\angle BAC$.
	
	%\textbf{Soluzione:}
	%Dagli esercizi sappiamo che $P=[0:b^2:-c^2]$. Dal teorema della bisettrice si deduce che $D=[c:b:0]$ e $E=[b:0:c]$, da cui $Q=[bc:b^2:c^2]$. A questo punto usando $O=[a^2S_A:b^2S_B:c^2S_C]$ e imponendo il determinante uguale a 0 si deduce $\angle BAC=60$.
\end{itemize}


\subsection{Esercizi}
\begin{enumerate}
	\item \textbf{[Scrittura del coniugato isogonale in complessi]} Dimostrare che in un triangolo $abc$ inscritto in una
	circonferenza unitaria centrata nell'origine, il coniugato isogonale di $p$ è
	\begin{equation}
	q=\frac{-p+a+b+c-\overline{p}(ab+bc+ca)+\overline{p}^2abc}{(1-p\overline{p})}.
	\end{equation}
	\item \textbf{[Seconda intersezione di due circonferenze in complessi]} Siano dati 4 punti $a, b, c, d$ nel piano complesso che non formano un parallelogrammo.
	
	Mostrare che esiste una e una sola rotomotetia che
	manda $a$ in $b$ e $c$ in $d$. Detto $x$ il centro di tale rotomotetia e $\alpha$ la ragione, si ha
	$$
	c=\frac{ad-bc}{a-b-c+d}
	$$
	$$
	\alpha=\frac{b-d}{a-c}.
	$$
	
	Mostrare che l'intersezione delle circonferenze circoscritte a $ABX$ e $CDX$ dove $AC$ e $BD$ sono segmenti non paralleli le cui rette si intersecano in $X$, è il centro della rotomotetia che manda $A$ in $B$ e $C$ in $D$. 
	
	\textbf{Soluzione:} Sia $x$ il centro della rotomotetia, $\alpha$ il numero complesso che rappresenta la rotomotetia - ovvero l'argomento di $\alpha$ è l'angolo di rotazione e il modulo di $\alpha$ è la ragione della rotomotetia. Se manda $a$ in $b$, allora 
	$$
	b-x = (a-x)\alpha ,
	$$
	e poiché manda $c$ in $d$ si ha anche 
	$$
	d-x = (c-x)\alpha.
	$$
	Dunque per confronto 
	$$
	\frac{b-x}{a-x}=\frac{d-x}{c-x}
	$$
	da cui 
	$$
	(b-x)(c-x)=(d-x)(a-x)\Rightarrow x=\frac{ad-bc}{a-b-c+d}.
	$$
	
	Svolgendo i calcoli si ha infine
	
	$$
	\alpha=\frac{b-d}{a-c}.
	$$
	\item \textbf{[Una caratterizzazione della \textit{polare} come luogo dei \emph{quarti armonici}]} Sia $\gamma$ la circonferenza unitaria centrata nell'origine e sia $P$ un punto qualsiasi. Siano $r$ ed $s$ la polare di $P$ rispetto a $\Gamma$ e una retta passante per $P$ rispettivamente. 
		\begin{itemize}
		\item Mostrare che $r$ ha equazione
			\begin{equation}
			x\bar{p}-2+\bar{x}p=0
			\end{equation}
			dove $p$ è il numero complesso associato a $P$.
		\item  Supponiamo che $s$ intersechi $\gamma$ in $A_1, A_2$, ed $r$ in $Q$. Mostrare che $(P,Q;A1,A2)=-1$.
		\end{itemize}
	\item \textbf{[Scrittura del circocentro di un triangolo generico in complessi]} Mostrare che il circocentro del triangolo $z_1z_2z_3$ è
		\begin{equation}
		\frac{z_1\bar{z_1}(z_2-z_3)+z_2\bar{z_2}(z_3-z_1)+z_3\bar{z_3}(z_1-z_2)}{\begin{vmatrix}
			z_1 & \bar{z_1} & 1 \\
			z_2 & \bar{z_2} & 1 \\
			z_3 & \bar{z_3} & 1 
			\end{vmatrix}}.
		\end{equation}
	\item \textbf{[Teorema di Brocard]} Sia $ABCD$ un quadrilatero inscritto in una circonferenza di centro $O$. Le rette $AB$ e $CD$ si intersecano in $E$, le rette $AD$ e $BC$ si intersecano in $F$ e le rette $AC$ e $BD$ si intersecano in $G$. \\
	Mostrare che $O$ è ortocentro di $EFG$.
	
	\item Sia $ABC$ un triangolo di ortocentro $H$. Da $A$ si conducano le due tangenti alla circonferenza di diametro $BC$ che la intersecano in $P$ e $Q$. \\
	Mostrare che $H\in PQ$.
	
	\begin{sol} la circonferenza unitaria è quella di diametro BC.
	I punti x che stanno su tale circonferenza e per cui $AX \perp OX$ soddisfano una quadratica. Sia $H'$ l'intersezione di $AH$ con $PQ$. Basta mostrare che $CH' \perp AB.$
	\end{sol}
	
	\item \textbf{[Una caratterizzazione del \textit{punto di Lemoine}]} Sia $ABC$ un triangolo e siano $D$, $E$ e $F$ i punti medi di $BC$, $CA$ e $AB$ rispettivamente. Siano $X$, $Y$ e $Z$ i punti medi delle altezze condotte da $A$, $B$ e $C$ rispettivamente. 
	
	Mostrare che $DX$, $EY$ e $FZ$ si intersecano in un punto di coordinate baricentriche $[a^2:b^2:c^2]$. Chi è tale punto nel triangolo referenziale?
	
	$(\star)$ Mostrare che tale punto (il \textit{punto di Lemoine}) è l'unico punto ad essere baricentro del proprio triangolo pedale.
	\item \textbf{[Coordinate dei vertici del \textit{triangolo tangenziale} in baricentriche]}  Dato un triangolo $ABC$ referenziale in un sistema di coordinate baricentriche, mostrare che la tangente condotta da $A$ alla circonferenza circoscritta ad $ABC$ ha equazione
	\begin{equation}
	c^2y+b^2z=0.
	\end{equation}
	
	Ciclando opportunamente, calcolare le coordinate dei vertici del triangolo tangenziale (\textit{i.e.} il triangolo formato dalle intersezione delle tangenti condotte da $A$, $B$ e $C$ alla circonferenza circoscritta ad $ABC$).
	
	\textbf{Soluzione:}
	Calcoliamo le coordinate di $P$, intersezione della tangente condotta da $A$ alla circoscritta e $BC$. Risulta che 
	$$
	P=[0:-b^2:c^2]
	$$
	da cui si ottiene subito che la tangente, dovendo passare per $A=[1:0:0]$ è
	$$
	c^2y+b^2z=0.
	$$
	
	Ciclando si ottiene che i vertici del triangolo tangenziale sono $[a^2:b^2:-c^2]$ e ciclici.
	\item Sia dato un triangolo $ABC$ e un punto $P$ di coordinate baricentriche $[u:v:w]$ scegliendo come triangolo referenziale $ABC$.
		\begin{itemize}
			\item \textbf{[Proiezione di un punto sui lati in baricentriche]} Mostra che, dette $P_A$, $P_B$ e $P_C$ le proiezioni di $P$ sui lati $BC$, $CA$ e $AB$, si ottiene
			$$
			P_A = [0: S_Cu+a^2v:S_Bu+a^2w]
			$$
			$$
			P_B = [S_Cv+b^2u: 0: S_Av+b^2w]
			$$
			$$
			P_C = [S_Bw+c^2u: S_Aw+c^2v: 0]
			$$			
			dove $S_A=\displaystyle\frac{b^2+c^2-a^2}{2}$ e cicliche.
			\item  \textbf{[Punto all'infinito della retta perpendicolare in baricentriche]} Usando il punto precedente mostrare che 
			il punto all'infinito di una retta perpendicolare a $px+qy+rz=0$ è
			\begin{equation}
			[S_Bg-S_Ch:S_Ch-S_Af:S_Af-S_Bg]
			\end{equation}
			dove $[f:g:h]=[q-r:r-p:p-q]$ è il punto all'infinito della retta.
		\end{itemize}
	\item \textbf{[Intersezione delle ceviane per un punto P con la circoscritta in baricentriche]} Sia $P=[u:v:w]$, dove le coordinate baricentriche 
	sono riferite ad $ABC$. Dette $P^A$, $P^B$ e $P^C$ rispettivamente le intersezioni di $AP$, $BP$ e $CP$ con la circonferenza circoscritta, mostrare che 
	$$
	P^A=\left[\displaystyle\frac{-a^2vw}{c^2v+b^2w}:v:w\right]
	$$
	$$
	P^B=\left[u:\displaystyle\frac{-b^2uw}{a^2w+c^2u}:w\right]
	$$
	$$
	P^C=\left[u:v:\displaystyle\frac{-c^2uv}{a^2v+b^2u}\right].
	$$
	\item Ricordiamo il seguente fatto noto di geometria elementare: un punto $P$ sta sulla circonferenza circoscritta ad un triangolo $ABC$ se e solo se le sue proiezioni sui lati $AB$, $BC$ e $CA$ sono allineate (su quella che si chiama \textit{retta di Simson}). 
	
	Usando questo fatto e l'esercizio 9 mostrare che l'equazione della circonferenza circoscritta al triangolo referenziale è
	\begin{equation}
	a^2yz+b^2xz+c^2xy=0.
	\end{equation}
	\item Mostrare che l'asse radicale fra la circonferenza circoscritta al triangolo referenziale e 
		\begin{itemize}
			\item la circonferenza di Feuerbach è $S_Ax+S_By+S_Cz=0$.
			\item la circonferenza inscritta è $(p-a)^2x+(p-b)^2y+(p-c)^2z=0$, essendo $p=\displaystyle\frac{a+b+c}{2}$.
		\end{itemize} 
	\item \textbf{[Distanza fra due punti in baricentriche]} Siano $P=[u:v:w]$ e $Q=[u':v':w']$ le coordinate \textbf{baricentriche esatte} di due punti rispetto a un triangolo referenziale $ABC$.
	\begin{itemize} 
	\item Mostrare che
	\begin{equation}
	PQ^2=S_A(u-u')^2+S_B(v-v')^2+S_C(w-w')^2.
	\end{equation}
	\item Dato un generico punto $P=[u:v:w]$, mostrare che 
	\begin{equation}
	AP^2=\frac{c^2v^2+2S_Avw+b^2w^2}{(u+v+w)^2}
	\end{equation}
	e dedurre, ciclicamente, le espressioni per $BP^2$ e $CP^2$.
	\end{itemize}
	\item Mostrare che il coniugato isogonale del punto di Nagel (risp. Gergonne) è il centro di omotetia esterno (risp. interno) della circonferenza inscritta e circoscritta. 
\end{enumerate}


\subsection{Problemi}
\begin{enumerate}
	\item \textbf{[BMO 2009 - 2]} Sia $MN$ una segmento parallelo al lato $BC$ del triangolo $ABC$, con $M$ sul lato $AB$ e $N$ sul lato $AC$. Le rette $BN$ e $CM$ si incontrano in $P$. Le circonferenze circoscritte a $BMP$ e $CNP$ si incontrano in due punti distinti $P$ e $Q$. 
	
	Mostrare che $\angle BAQ = \angle PAC$.
	
	\textbf{Soluzione:} Diciamo che $a$ è l'origine del nostro piano di Gauss, mentre $b$ e $c$ sono due generici punti. Visto che $mn\parallel bc$ e $m\in ab$, $m\in ac$ si ha che esiste $\lambda \in \mathbb R$ tale che $m=\lambda b$ e $n=\lambda c$. Essendo $q$ il centro della rotomotetia che manda $m$ in $b$ e $c$ in $n$, allora 
	$$
	q=\frac{mn-bc}{m+n-b-c}=\frac{\lambda^2bc-bc}{\lambda b+\lambda c-b-c}=\frac{(\lambda+1)bc}{b+c}.
	$$
	
	Per trovare trovare $p$ basterebbe imporre $p\in mc$ e $p\in bn$. \emph{Proporlo come esercizio}. D'altra parte non ce n'è bisogno: infatti noi siamo interessati poi solo all'angolo $\angle CAP$ e dunque non tanto ci servono le coordinate di $P$ quanto capire chi è la retta $AP$, che è la mediana di $ABC$. Dunque possiamo dire che esiste un certo $\eta$ reale tale che 
	$$
	p=\eta(b+c).
	$$ 
	
	Per l'equazione dell'angolo, se $\theta=\angle BAQ$ si ha 
	$$
	e^{2i\theta}=\frac{q-a}{b-a}\frac{\bar b-\bar a}{\bar q-\bar a}=\frac{c(\bar b+\bar c)}{\bar c(b+c)},
	$$
	mentre se $\theta'=\angle PAC$ si ha 
	$$
	e^{2i\theta'}=\frac{c-a}{p-a}\frac{\bar p-\bar a}{\bar c-\bar a}=\frac{c(\bar b+\bar c)}{\bar c(b+c)}.
	$$
	
	Da ciò, con un attimo di discussione, si ottiene che $\theta=\theta'$ che implica la tesi. 
	
	\item \textbf{[RMM 2012 - 2]} Sia $ABC$ un triangolo non isoscele e siano $D$, $E$ e $F$ rispettivamente i punti medi dei lati $BC$, $CA$ e $AB$. La circonferenza $BCF$ e la retta $BE$ si intersecano nuovamente in $P$ e la circonferenza $ABE$ e la retta $AD$ in $Q$. Le rette $DP$ e $FQ$ si incontrano in $R$. 
	
	Mostrare che il baricentro $G$ del triangolo $ABC$ giace sulla circonferenza circoscritta al triangolo $PQR$.
	
	\textbf{Soluzione:} Per mostrare la ciclità è sufficiente mostrare che, detto $\theta=\angle GPD$ e $\theta'=\angle GQF$, si ha 
	$$
	\theta=\theta'.
	$$  
	Dall'equazione dell'angolo risulta che per fare ciò è sufficiente mostrare
	$$
	\frac{d-p}{g-p}\frac{g-q}{f-q}\in\mathbb R.
	$$
	Il problema è dunque spostato a trovare i punti $p$ e $q$. Qui usiamo un'osservazione sintetica. Si ha che 
	$$
	\angle GDE=\angle GAB = \angle QEG,
	$$
	dove la prima è vera per il parallelismo $AB\parallel ED$ e la seconda è vera poiché $ABEQ$ è ciclico. 
	Analogamente si ha $\angle EQD=\angle GED$ e dunque i triangoli $GDE$ e $GEQ$ sono ordinatamente simili.
	Dunque, scegliendo $g=0$, risulta, visto che $GD\cdot GQ=GE^2$,
	$$
	q=d\frac{|e|^2}{|d|^2}=\frac{e\bar e}{\bar d}
	$$
	e analogamente 
	$$
	p=\frac{f\bar f}{\bar e}.
	$$
	A questo punto 
	$$
	\frac{d-p}{g-p}\frac{g-q}{f-q}=\frac{(d\bar e-f\bar f)e\bar e}{(f\bar d-e\bar e)f\bar f}
	$$
	e poiché, essendo $g=0$, si ha $d+e+f=0$, la precedente espressione è uguale a 
	$$
	\frac{|e|^2}{|f|^2}
	$$ 
	che è un numero reale, come si voleva.
%	Dall'equazione dell'angolo si ha 
%	$$
%	e^{2i\theta}=\frac{d-p}{g-p}\frac{\bar g-\bar p}{\bar d-\bar p}%=\frac{f\bar f-d\bar e}{f\bar f-\bar d e}
%	$$
%	e 
%	$$
%	e^{2i\theta'}=\frac{f-q}{g-q}\frac{\bar g-\bar q}{\bar f-\bar q}%=\frac{e\bar e-f\bar d}{e\bar e-\bar f d}.
%	$$
%	Per mostrare che $e^{2i\theta}=e^{2i\theta'}$, e dunque $\theta=\theta'$ e dunque la tesi, è sufficiente mostrare che 
	
	che segue poiché $d+e+f=0$, essendo $g=0$, e sostituendo. 
	
	
	\item \textbf{[USAMO 2016 - Day 2 - 2]} Un pentagono equilatero $AMNPQ$ è inscritto in un triangolo $ABC$ in modo che $M\in AB$, $Q\in AC$ e $N,p \in BC$. Sia $S$ l'intersezione di $MN$ e $PQ$ e denotiamo con $l$ la bisettrice di $\angle MSQ$. 
	
	Mostrare che, detto $I$ l'incentro di $ABC$, $OI$ è parallelo a $l$.
	\item \textbf{[IMO 2008 - 6]} Sia $ABCD$ un quadrilatero convesso con $BA \neq BC$. Siano $\omega_1$ e $\omega_2$ le circonferenze inscritte ai triangoli $ABC$ e $ADC$ rispettivamente. Supponiamo che esista una circonferenza $\omega$ tangente alla retta $BA$ oltre A, alla retta $BC$ oltre $C$, alla retta $AD$ e alla retta $CD$.
	
	Mostrare che le tangenti esterne comuni a $\omega_1$ e $\omega_2$ si intersecano su $\omega$.
	
	\item \textbf{[BMO 2015 - 2]} Sia $ABC$ un triangolo scaleno con incentro $I$ e circonferenza circoscritta $\omega$. $AI$, $BI$ e $CI$ intersecano $\omega$ di nuovo nei punti $D$, $E$ e $F$ rispettivamente. Le rette parallele a $BC$, $CA$ e $AB$ condotte da $I$ intersecano $EF$, $DF$ e $DE$ rispettivamente nei punti $K$, $L$ e $M$.
	
	Mostrare che $K$, $L$ e $M$ sono allineati.
	
	\item \textbf{[IMO 2012 - 1]} Dato un triangolo $ABC$, sia $J$ il centro della circonferenza ex-inscritta opposta al vertice $A$, la quale tange $BC$ in $M$ e le rette $AB$ e $AC$ in $K$ e $L$ rispettivamente. Le rette $LM$ e $BJ$ si intersecano in $F$ e le rette $KM$ e $CJ$ si intersecano in $G$. Sia $S$ il punto d'intersezione fra $AF$ e $BC$ e sia $T$ il punto d'intersezione fra $AG$ e $BC$. 
	
	Mostrare che $M$ è il punto medio di $ST$.
	
	\textbf{Soluzione:}
	
	\item \textbf{[IMO SL 2011 - 4]} Sia $ABC$ un triangolo acutangolo scaleno, e sia $\gamma$ la sua circonferenza circoscritta.
	Siano $A_0$ il punto medio di BC, $B_0$ il punto medio di $AC$ e $C_0$ il punto medio di $AB$. Sia
	$D$ il piede dell’altezza uscente da $A$, $D_0$ la proiezione di $A_0$ sulla retta $B_0C_0$ e $G$ il
	baricentro di $ABC$. Sia $\gamma_1$ la circonferenza passante per $B_0$ e $C_0$, e tangente a $\gamma$ in un
	punto $P$ diverso da $A$.
	\begin{itemize}
	\item Dimostrare che la retta $B_0C_0$ e le tangenti a $\gamma$ nei punti $A$ e $P$ sono concorrenti.
	\item Dimostrare che i punti $D_0$, $G$, $D$, e $P$ sono allineati.
	\end{itemize}
	\item \textbf{[USA TST 2012 - December Test - 1]} In un triangolo acutangolo $ABC$ si ha $\angle A<\angle B$ e $\angle A<\angle C$. Sia $P$ un punto variabile su $BC$. I punti $D$ e $E$ giacciono su $AB$ e $AC$ rispettivamente in modo che $BP=PD$ e $CP=PE$.
	
	Mostrare che al variare di $P$ sul segmento $BC$, la circonferenza circoscritta al triangolo $ADE$ passa per un punto fisso oltre $A$.
	\item  \textbf{[IMO 2019 - 6]} 
	Sia $I$ l'incentro di un triangolo acutangolo $ABC$ con $AB\neq AC$. La circonferenza inscritta $\omega$ di $ABC$ è tangente a $BC$, $CA$ e $AB$ in $D$, $E$ e $F$ rispettivamente. La retta per $D$ e perpendicolare ad $EF$ interseca $\omega$ di nuovo in $R$ e la retta $AR$ interseca $\omega$ di nuovo in $P$. Sia $Q$ la seconda intersezione, diversa da $P$, delle circonferenze circoscritte ai triangoli $PBF$ e $PCE$.
	
	Mostrare che le rette $DI$ e $PQ$ si incontrano su una retta per $A$ perpendicolare ad $AI$.
	
	\textbf{Soluzione:} Usare come circonferenza unitaria la circonferenza inscritta. Consulta \url{https://artofproblemsolving.com/community/c6h1876745p12752769}.
	
	\item \textbf{[MOP 2006]} Sia $ABC$ un triangolo inscritto in una circonferenza $\omega$. $P$ giace su $BC$ in modo tale che $PA$ è tangente a $\omega$. La bisettrice di $\angle APB$ interseca i segmenti $AB$ e $AC$ rispettivamente in $D$ ed $E$ e i segmenti $BE$ e $CD$ si intersecano in $Q$. Supponiamo che la retta $PQ$ passi per il centro di $\omega$. 
	
	Calcolare $\angle BAC$.
	
	\item \textbf{[USAMO 2001]} Sia $ABC$ un triangolo di circonferenza inscritta $\omega$. Siano $D_1$ ed $E_1$ i punti in cui $\omega$ tange $BC$ e $AC$ rispettivamente. Siano $D_2$ ed $E_2$ i punti sui lati $BC$ e $AC$ rispettivamente tali che $CD_2=BD_1$ e $CE_2=AE_1$ e sia $P$ il punto d'intersezione dei segmenti $AD_2$ e $BE_2$. La circonferenza $\omega$ interseca il segmento $AD_2$ in due punti, il più vicino dei quali al vertice $A$ sia detto $Q$. 
	
	Mostrare che $AQ=D_2P$.
	
	\textbf{Soluzione:} Scrivere tutti i punti in coordinate baricentriche normalizzate. Per trovare $Q$ notare che $I$ è il punto medio di $QD_1$. Infine per mostrare $AQ=D_2P$ usare i displacement dati i punti con le coordinate normalizzate. 
	

\end{enumerate}

\clearpage

\section{GM - 2, [Geometria proiettiva]}

\begin{short}
 Punti all'infinito. Lunghezze con segno (velocemente). Birapporto tra 4 punti su una retta. Proiezione del birapporto, quindi birapporto tra 4 rette o 4 punti su circonferenza. Quaterna Armonica, quadrilatero armonico e le loro proprietà e configurazioni. \\
 Teorema di Desargues. Teorema di Pascal. Teorema di Pappo. \\
 Polo e Polare. Teorema di La Hire. Lemma della polare. Teorema di Brokard.  Dualità polo-polare. 
\end{short}



\vspace{0.3cm}
\large{\textbf{Versione estesa}}\normalsize
\begin{itemize}

\vspace{0.3cm}
\item \textbf{Introduzione} In geometria euclidea due rette si intersecano oppure sono parallele. Questo crea dei problemi quando in una dimostrazione si prende l'intersezione di rette perché il punto potrebbe non esistere. Si può ovviare a questo problema aggiungendo per ogni retta (per ogni insieme di rette parallele) un punto all'infinito.  Affinché le cose funzionino bene, diciamo che tutti i punti all'infinito sono allineati sulla retta all'infinito.\\
In questo modo si crea il piano proiettivo, come unione del piano euclideo con la retta all'infinito.


\item \textbf{Lunghezze con segno} Su una retta $r$ sono presenti alcuni punti $A,B,C\ldots$. Si scelga un verso sulla retta e si considerino i segmenti su di essa come vettori, con segno positivo se orientati nel verso scelto e negativo altrimenti. Il vantaggio di questo è che vale $\overline{AC}=\overline{AB}+\overline{BC}$ per qualsiasi posizione reciproca di $A,B,C$.

\item \textbf{Birapporto} Dati 4 punti $A,B,C,D$ su una retta, si definisce il birapporto è la seguente quantità:
$$(A,B;C,D)=\frac{\frac{AC}{AD}}{\frac{BC}{BD}}=\frac{AC\cdot BD}{BC\cdot AD}$$
dove le lunghezze sono prese con segno.

\item $\bigstar$ \textbf{Permutazione del birapporto} Se $(A,B;C,D)=k$, qual è il valore del birapporto se si permuta l'ordine in cui si prendono i punti? Le $4!=24$ possibilità si dividono in $6$ gruppi in ciascuno dei quali il birapporto è lo stesso. Se si scambiano le due coppie oppure si inverte l'ordine in entrambe il birapporto non cambia: $(A,B;C,D)=(C,D;A,B)=(B,A;D,C)$.

Se si scambiano i primi due o gli ultimi due, il birapporto diventa reciproco: $(A,B;D,C)=(B,A;C,D)=1/k$.

Se si scambia il secondo e il terzo $B \leftrightarrow C$, si ottiene $(A,C;B,D)=1-k$.

Se si scambia il primo e il terzo $A \leftrightarrow C$, si ottiene $(C,B;A,D)=\frac{k}{k-1}$.


Combinando queste trasformazioni, si possono ottenere i valori di $(A,C;D,B)=\frac{1}{1-k}$ e $(A,D;B,C)=\frac{k-1}{k}$.

\item \textbf{Suriettività e iniettività del birapporto} Un'altra cosa interessante è fissare i punti $A,B,C$ e vedere come varia il birapporto $(A,B;C,D)$ al variare di $D$ sulla retta. Questa è una funzione biettiva dalla retta proeittiva in $\mathbb{R}\cup\infty$, nei casi degeneri in cui $D$ coincide con uno dei punti assume i valori degeneri di $0,1,\infty$; se $D=\infty$, il birapporto vale $AC/BC$. In particolare, se $(A,B;C,D_1)=(A,B;C,D_2)$, allora $D_1=D_2$.

\item \textbf{Invarianza del birapporto per proiezione} Siano $A,B,C,D$ su una retta $r$, sia $r'$ un'altra retta e sia $P$ un punto del piano. Proietto i punti su $r'$: $A'=PA\cap r$ e analogamente per gli altri. Allora $(A,B;C,D)=(A',B';C',D')$.\\
Dimostrazione: applico il teorema dei seni ai triangoli $PAC,PBC,PAD,PBD$ in modo da sostituire $AC$ con $\sin APC$ e cicliche, semplificando i segmenti $AP$, $BP$ e gli altri angoli. Si ottiene $(A,B;C,D)=\frac{\sin APC \cdot \sin BPD}{\sin BPC \cdot APD}$, che è uguale all'altro birapporto perché $PAA'$ sono allineati (e cicliche).\\
Il birapporto si conserva anche nel caso i punti vengono proiettati su un cerchio.\\
ATTENZIONE: il punto da cui si proiettano deve stare sul cerchio (o in generale sulla conica). 

\textit{Esercizio} teorema della farfalla.

\item \textbf{Lemmetti} 1) $r,s$ si intersecano in $P$, $A,B,C$ su $r$ e $A',B',C'$ su $s$. Allora $AA',BB',CC'$ concorrono se e solo se $(P,A;B,C)=(P',A';B',C')$.\\
2) Le rette $(l,r,s,t)$ concorrono in $P$ e $(l,r',s',t')$ concorrono in $Q$. Allora $r\cap r'$, $s \cap s'$, $t\cap t'$ sono allineati.

\item \textbf{Teorema di Desargues} Siano $ABC$ e $A'B'C'$ due triangoli. Si chiamino $X=BC\cap B'C'$, $Y=AC\cap A'C'$, $Z=AB\cap A'B'$. Le rette $AA',BB',CC'$ concorrono se e solo se $X,Y,Z$ sono allineati.

\textit{Esempio} $ABC$ triangolo, $M_A,M_B,M_C$ punti medi, $D_{\infty}$ punto all'infinito di $AH$ e $B_{\infty}$ di $BH$. Per Desargues su $A,M_A,D_\infty$ e $B,M_B,E_{\infty}$ G,H,O sono allineati. \\
\textit{Esempio} Retta tripolare


\item \textbf{Quaterna Armonica} Quattro punti su una retta si dicono una quaterna armonica se $(A,B;C,D)=-1$. Per quanto detto sulle permutazioni, una quaterna è armonica se e solo se non è degenere e $(A,B;C,D)=(B,A;C,D)$.

Una quaterna armonica dev'essere "incatenata": fissati $A,B$, uno tra $C$ e $D$ deve stare all'interno del segmento $AB$ e uno all'esterno. Analogamente si avrà che uno tra $A$ e $B$ sta all'interno del segmento $CD$ e uno all'esterno.

\emph{Esercizio} Sia $ABC$ un triangolo, $D,E,F$ sui lati. Sia $G=EF\cap BC$. Allora $AD,BE,CF$ concorrono se e solo se $(B,C;D,G)=-1$.\\
\begin{sol}
1) Ceva + Menelao 2) Proiettare da $A$ e da $BE\cap CF$ per vedere $(B,C;D,G)=(C,B;D,G)$ 
\end{sol}

\item \textbf{Apollonio} Reminder veloce che fissati $A,B$ nel piano e $k\in \mathbb{R}$, il luogo dei punti $P$ tali che $\frac{|AP|}{|BP|}$ è una circonferenza $\omega$ con centro su $AB$. Chiamati $C,D=\omega \cap AB$, $(A,B;C,D)=-1$ e per ogni $P$ su $\omega$, $PC$ e $PD$ sono bisettrici interna ed esterna di $\triangle ABP$. 

\emph{Esercizio} Dati $A,B,C,D$ su una retta e $P$ punto esterno, due delle seguenti condizioni implicano la terza:\\
1) $(A,B;C,D)=-1$ \hspace{0.2cm} 2) $\widehat{APB}=90$  \hspace{0.3cm} 3)$PB$ biseca $\widehat{CPD}$.

\item \textbf{Proprietà della quaterna armonica} Dati quattro punti $A,B,C,D$ su una retta, sono equivalenti:\\
1) $(A,B;C,D)=-1$ \hspace{0.2cm}2) $MA\cdot MB=MC^2$ \hspace{0.2cm} 3) $CA\cdot CB=CD\cdot CN$  \hspace{0.2cm}\\ 4)  $\displaystyle\frac{2}{AB}=\displaystyle\frac{1}{AC}+\displaystyle\frac{1}{AD}$  \hspace{0.2cm} 5)  $AB^2+CD^2=4MN^2$  \hspace{0.2cm} 6)  $\frac{NC}{ND}=\left(\frac{AC}{AD}\right)^2=\left(\frac{AC}{AD}\right)^2$

\item \textbf{Esempi di quaterne armoniche}\\
1) Due circonferenze e i loro centri di similitudine (interno ed esterno)\\
2) Vertici di un triangolo e piedi delle bisettrici sul lato\\
3) Due punti inversi rispetto a un cerchio $P$ e $P'$, e le intersezioni del cerchio con la retta $PP'$.

\item \textbf{Simmediana}\\
Sia $ABC$ un triangolo inscritto in $\Gamma$, $P$ l'intersezione delle tangenti a $\Gamma$ in $B$ e $C$, $D=AP\cap \Gamma$ e $Q=AP\cap BC$. Allora $\triangle PDC \sim \triangle PCA$, $(A,D;Q,P)=-1$ e $AB\cdot CD=AC\cdot BD$.

\item \textbf{Quadrilatero armonico}
Dati quattro punti $A,B,C,D$ su una circonferenza in quest'ordine, le seguenti proprietà sono equivalenti e in tal caso $ABCD$ viene detto quadrilatero armonico:\\
1) $(A,C;B,D)=-1$ \hspace{0.2cm} 2) $AB\cdot CD= BC\cdot AD$ \hspace{0.2cm} 3) $BD$ è simmediana in $\triangle ABC$ 4) $BD$ e le tangenti a $\Gamma$ in $A$ e $C$ concorrono \hspace{0.2cm}
5) Detto $M$ punto medio di $AC$, $\angle BMA=\angle AMD$  \hspace{0.2cm} 6) Le bisettrici di $\angle ABC$ e $\angle ADC$ concorrono su $AC$  

\item \textbf{Altre proprietà del quadrilatero armonico}\\
1) $\frac{AB^2}{AD^2}=\frac{MB}{MD}$\\
2) I triangoli $BMC$, $BAD$, $CMD$ sono simili.\\
3) Chiamata $Q$ l'intersezioni delle tangenti in $B$ e $D$, allora $BQDMO$ è ciclico

\item \textbf{Teorema di Pascal}
Siano $A,B,C,D,E,F$ su un cerchio $\Gamma$. Sia $Z=AE\cap BD$, $Y=AC\cap DF$, $X=BC\cap EF$. Allora i punti $X,Y,Z$ sono allineati.\\
Vale anche il teorema di Pappo, nel caso i punti stiano su due rette. \\
Una cosa a cui prestare attenzione è che non vale l'implicazione inversa: se i punti $X,Y,Z$ sono allineati, non è detto che $ABCDEF$ siano conciclici, si può dire al massimo che si trovano su una stessa conica.

\emph{Esercizio} Teorema di Newton, teorema di Brianchon

\item \textbf{Polo e Polare}
Sia $\Gamma$ una circonferenza e $P$ un punto. La polare di $P$ è la retta passante per $P'$ (l'inverso di $P$) e perpendicolare ad $OP$. \\
Se $P$ è esterno a $\Gamma$, la $\text{pol}(P)$ è la retta che passa per l'intersezioni delle tangenti da $P$ a $\Gamma$. \\
La polare di $O$ è la retta all'infinito.

\item\textbf{Teorema di La Hire} $A \in \text{pol}(B) \iff B \in \text{pol}(A)$\\
Corollari: $A,B,C$ sono allineati se e solo se $\text{pol}(A), \text{pol}(B), \text{pol}(C)$ concorrono.\\
Il polo di $PQ$ è l'intersezione della polare di $P$ e della polar e di $Q$: $\text{pol}(P)\cap \text{pol}(Q)= \text{pol}(PQ)$\\
La polare di $r\cap s$ è la retta per il polo di $r$ e il polo di $s$:  $\text{pol}(r\cap s)= \overline{\text{pol}(r)\text{pol}(s)}$\\

\item \textbf{Lemma della polare}
$A,B$ su $\Gamma$ cerchio, $C,D$ sulla retta $A,B$. Allora $(A,B;C,D)=-1$ se e solo se $C \in \text{pol}(D)$.

\item \textbf{Teorema di Brokard}
$A,B,C,D$ quattro punti su un cerchio $\Gamma$, considero il quadrilatero completo con $P=AB\cap CD, Q=AD\cap BC, R=AC\cap BD$. Allora $PQ$ è la polare di $R$ e cicliche.

\item $\bigstar$ \textbf{Polare per coniche} 
Si può definire la polare per una conica $\gamma$ e punto $P$: al variare di una retta $r$ passante per $P$ che interseca $\gamma$ in $A,B$, il luogo dei punti $Q$ tali che $(A,B;P,Q)=-1$ è una retta ed è la polare di $P$

\item $\bigstar$ \textbf{Proiettività} Una proiettività è una trasformazione del piano che conserva il birapporto di qualsiasi quaterna di punti. Può essere vista come proiezione di un piano su un altro [più in generale è un'applicazione lineare nel piano proiettivo]. Manda rette in rette, coniche in coniche, conserva intersezioni e tangenze.\\
Può essere usata per mandare una retta all'infinito, che può rendere la configurazione più semplice e simmetrica. Visto che un cerchio andranno in un ellisse, può essere utile applicare successivamente un'affinità per rimandarlo nel cerchio.

\end{itemize}



\subsection{GM - 2, Esercizi}
\begin{enumerate}

	\item \textbf{[Unicità del quarto armonico]}
	Assumiamo che $A$, $B$, $C$, $D_1$ e $D_2$ siano conciclici o allineati.
	
	Mostrare che se $(A,B;C,D_1)=(A,B;C,D_2)$ allora $D_1 \equiv D_2$.
	
	\item \textbf{Permutazioni in un birapporto} Siano $A,B,C,D$ quattro punti tali che $(A,B;C,D)=k$. Dimostrare che:
	\begin{itemize}
	 \item $(A,B;C,D)=(B,A;D,C)=(C,D;A,B)=(D,C;B,A)=k$
	 \item $(A,B;D,C)=(B,A;C,D)=(D,C;A,B)=(C,D;B,A)\frac{1}{k}$
	 \item $(A,C;B,D)=(C,A;D,B)=(B,D;A,C)=(D,B;C,A)=1-k$
	 \item $(A,C;D,B)=(C,A;B,D)=(D,B;A,C)=(B,D;A,C)\frac{1}{1-k}$
	 \item $(A,D;C,B)=(D,A;B,C)=(C,B;A,D)=(B,C;D,A)=\frac{k}{k-1}$
	 \item $(A,D;B,C)=(D,A;C,D)=(C,B;D,A)=(B,C;A,D)=\frac{k-1}{k}$
	\end{itemize}

	
	\item Siano $A$, $C$, $B$ e $D$ allineati in quest'ordine su una retta. Siano $M$ e $N$ i punti medi dei segmenti
	$CD$ e $AB$ rispettivamente. 
	
	Mostrare che sono equivalenti le seguenti proprietà:
	\begin{itemize}
		\item $(A,B;C,D)=-1$;
		\item $MA\cdot MB=MC^2$;
		\item $CA\cdot CB=CD\cdot CN$;
		\item $\displaystyle\frac{2}{AB}=\displaystyle\frac{1}{AC}+
		\displaystyle\frac{1}{AD}$;
		\item $AB^2+CD^2=4MN^2$.
		\item $\frac{NC}{ND}=\left(\frac{AC}{AD}\right)^2=\left(\frac{AC}{AD}\right)^2$
	\end{itemize}
	\item Siano $\gamma_1$ e $\gamma_2$ due circonferenze di centri $O_1$ e $O_2$ rispettivamente. Siano $S_1$ e $S_2$ rispettivamente il centro di similitudine interno ed esterno di $\gamma_1$ e $\gamma_2$.
	
	Mostrare che $(O_1,O_2;S_1,S_2)=-1$.	
	\item Siano $\gamma_1$ e $\gamma_2$ due circonferenze \textit{ortogonali} di centri $O_1$ e $O_2$ rispettivamente. Una generica retta passante per $O_1$ interseca $\gamma_1$ in $A$ e $B$ e interseca $\gamma_2$ in $C$ e $D$.
	
	Mostrare che $(A,B;C,D)=-1$.
	\item \textbf{[Conservazione del birapporto per inversione]} Assumiamo che $A$, $B$, $C$ e $D$ siano allineati o conciclici. Siano $A'$, $B'$, $C'$ e $D'$ (allineati o conciclici) le immagini dei precedenti punti tramite un'inversione circolare di centro $O\notin\left\{A,B,C,D\right\}$  qualsiasi. Allora
	\begin{equation}
	(A,B;C,D)=(A',B';C',D').
	\end{equation}
	
	Cosa succede se $O\in \{A,B,C,D\}$?
	
    \item Sia $ABC$ un triangolo e $M$ un punto sul segmento $BC$. Sia $N$ preso sulla retta di $BC$ dimodoché $\angle MAN=90$.
    
    Mostrare che $(B,C;M,N)=-1$ se e solo se $AM$ è bisettrice dell'angolo $\angle{BAC}$.
    \item Sia $ABC$ un triangolo scaleno e sia $D \in AC$ tale che $BD$ è la bisettrice di $\angle ABC$.
    Siano $E$ ed $F$ i piedi delle perpendicolari tracciate rispettivamente da $A$ e da $C$ sulla retta $BD$ e
    sia $M \in BC$ tale che $DM \perp BC$.
    
    Mostrare che $\angle EMD=\angle DMF$.
    \item \textbf{[Teorema della farfalla]} Sia $MN$ una corda di una circonferenza $\gamma$ e sia $P$ il suo punto medio. Siano $AB$ e $CD$ due corde qualsiasi di $\gamma$ che si intersecano in $P$ dimodoché $A$ e $C$ siano nello stesso semipiano generato dalla retta su cui giace $MN$. 
    
    Mostrare che $AD$ e $BC$ intersecano la corda $MN$ in due punti equidistanti da $P$. 
    \item Sia $ABCD$ un quadrilatero circoscritto a una circonferenza e siano $M$, $N$, $P$ e $Q$ i punti di tangenza di $AB$, $BC$, $CD$ e $DA$ con la circonferenza rispettivamente. 
    
    Mostrare che $AC$, $BD$, $MP$ e $NQ$ sono concorrenti.
    \item \emph{[Copiato in GB]} \textbf{[Lemma della \textit{simmediana}]} Sia $ABC$ un triangolo inscritto in una circonferenza $\gamma$. Le tangenti a $\gamma$ in $B$ e $C$ si intersecano in $P$.
    
    Mostrare che $AP$ è \textit{simmediana} relativa a $BC$, \textit{i.e.} simmetrica della mediana relativa a $BC$ rispetto alla bisettrice dell'angolo $\angle BAC$.
    
    \item Sia $ABCD$ un quadrilatero ciclico. Le rette $AB$ e $CD$ si intersecano in un punto $E$ e le diagonali $AC$ e $BD$ si intersecano in un punto $F$. Sia $H$ l'intersezione delle circonferenze circoscritte ai triangoli $AFD$ e $BFC$. 
    
    Mostrare che $\angle EHF=90^{\circ}$.
    
    \item Sia $ABCD$ un quadrilatero armonico inscritto in una circonferenza $\gamma$ di centro $O$ con diagonali $AB$ e $CD$. Sia $M$ il punto medio di $AB$.
    
    Mostrare $MA$ è la bisettrice dell'angolo $\angle CMD$.
    \item Usando gli argomenti della lezione \textbf{G2 - Medium} mostrare il \textbf{Teorema di Brocard} contenuto nella raccolta degli esercizi relativi alla lezione \textbf{G1 - Medium}. 
    
    \item Sia $\omega$ la circonferenza inscritta in un triangolo $ABC$ e sia $I$ il suo centro. $\omega$ interseca $BC$, $CA$ e $AB$ rispettivamente in $D$, $E$ e $F$. $BI$ interseca $EF$ in $K$.
    
    Mostrare che $BK\perp CK$. 
    \item Sia $ABC$ un triangolo la cui circonferenza inscritta, di centro $I$, tange $BC$,$CA$ e $AB$ in $D$,$E$ e $F$  rispettivamente. Siano $N$ l'intersezione di $ID$ con $EF$ e $M$ il punto medio di $BC$.
    
    Mostrare che $A$, $N$ e $M$ sono allineati.
\end{enumerate}



\subsection{GM - 2, Problemi}
\begin{enumerate}
	\item \textbf{[China NMO 2017 - 2]} Siano $\omega$ e $\Omega$ di centro $I$ e $O$ rispettivamente la circonferenza inscritta e circoscritta a un triangolo acutangolo
	$ABC$. La circonferenza $\omega$ interseca $BC$ in $D$ e le tangenti a $\Omega$ passanti per $B$ e $C$ si intersecano in $L$.
	Siano $AH$ l'altezza condotta da $A$ a $BC$ e $X$ l'intersezione di $AO$ con $BC$. Siano $P$ e $Q$ le 
	intersezioni di $OI$ con $\Omega$.
	
	Mostrare che $PQXH$ è ciclico se e solo se $A,D$ e $L$ sono allineati.
	\item \textbf{[IMO 2014 - 4]} Siano $P$ e $Q$ punti su un segmento $BC$ di un triangolo acutangolo $ABC$ tali che $\angle PAB = \angle BCA$ e $\angle CAQ=\angle ABC$. Siano $M$ e $N$ punti su $AP$ e $AQ$ rispettivamente tali che $P$ è punto medio di $AM$ e $Q$ è punto medio di $AN$.
	
	Mostrare che l'intersezione di $BM$ e $CN$ giace sulla circonferenza circoscritta di $ABC$.
	
	\item \textbf{[Iran TST 2007 - Day 2 - 3]}
	Sia $\omega$ la circonferenza inscritta ad un triangolo $ABC$ che tange $AB$ e $AC$ rispettivamente in $F$ e $E$. Siano $P$ e $Q$ su $AB$ e $AC$ rispettivamente in modo che $PQ$ sia parallelo a $BC$ e tangente ad $\omega$. Siano $T$ l'intersezione di $EF$ con $BC$ e $M$ il punto medio di $PQ$. 
	
	Mostrare che $TM$ tange $\omega$.
	
	\begin{sol}Se $X=AD\cap \omega$, $TX$ tange $\omega$ per quadrilateri armonici. Poi (XDAY)=-1 e proiettando da $T$ su $PQ$ ottengo che l'intersezione di $TX$
	 con $PQ$ è il suo punto medio
	\end{sol}
	
	\item \textbf{[Iran TST 2009 - Day 2 - 3]}
	In un triangolo $ABC$ è inscritta una circonferenza $\omega$ di centro $I$ che interseca i lati $BC$, $CA$ e $AB$ rispettivamente in $D$, $E$ e $F$. Sia $M$ il piede della perpendicolare da $D$ a $EF$. Sia $P$ il punto medio di $DM$ e $H$ l'ortocentro del triangolo $BIC$.
	
	Mostrare che $PH$ biseca $EF$. 
	\item \textbf{[Romania TST 2007 - Day 7 - 2]}	La circonferenza inscritta al triangolo $ABC$ è tangente 
	ad $AB$ e $AC$ in $F$ ed $E$ rispettivamente. Sia $M$ il punto di $BC$ e $N$ l'intersezione di $AM$ con $EF$. La circonferenza di diametro $BC$ interseca $BI$ e $CI$ in $X$ e $Y$ rispettivamente.
	
	Mostrare che $\displaystyle\frac{NX}{NY}=\displaystyle\frac{AC}{AB}$.
	
	\begin{sol}Usa l'esercizio 13 e nota che DXY è simile ad ABC e ID è bisettrice di YDX. Oppure semplicemente formula seni-lati su IXY e un po' di trigonometria
	\end{sol}
	
	\item \textbf{[IMO SL 2007 - G8]}
	Sul lato $AB$ di un quadrilatero convesso $ABCD$ è preso un punto $P$. Sia $\omega$ la circonferenza inscritta al triangolo $CPD$ e sia $I$ il suo centro. Supponiamo che $\omega$ sia tangente alle circonferenze inscritte ai triangoli $APD$ e $BPC$ in $K$ e $L$ rispettivamente. Siano $E$ l'intersezione delle rette $AC$ e $BD$ e $F$ l'intersezione delle rette $AK$ e $BL$.
	
	Mostrare che $E$, $I$ e $F$ sono allineati.
	
	\item \textbf{APMO 2012 - 4} Sia $ ABC $ un triangolo acutangolo e sia $ D $ su $BC$ il piede dell'altezza da $ A $. Indichiamo poi con $M$ il punto medio di $BC$ e con $H$ l'ortocentro. Sia $E$ il punto di intersezione della circonferenza circoscritta $\Gamma$ con la semiretta per $H$ uscente da $M$ e sia $F$ il punto di intersezione (diverso da $E$) di $ED$ con $\Gamma$.\\
	Dimostrare che $BF/CF=AB/AC$.
	
	\item \textbf{Romania TST 2010, Round 3 - 2} $ABC$ è un triangolo, $\Gamma$ è la sua circonferenza circoscritta e $I$ l'incentro. La bisettrice di $\widehat{ABC}$ interseca $AC$ in $B_0$ e $\Gamma$ in $B_1$, la bisettrice di $\widehat{ACB}$ interseca $AB$ in $C_0$ e $\Gamma$ in $C_1$. \\
	Dimostrare che $B_0C_0,B_1C_1$ e la retta parallela a $BC$ passante per $I$ concorrono.
	
	\item \textbf{IMO 2019 - 2}
	\item \textbf{Romania TST 2018} \emph{Non c'è su AoPS, viene da Senior 2016 GM2, Sam}
	\item \textbf{WC 2010 - 6} 

\end{enumerate}

 \clearpage
 
\section{GM - 3 [Configurazione di Miquel, rotomotetia, circonferenze mistilinee e inversioni sintetiche]}

\subsection{Programmi}

\begin{short}
 Angoli orientati, Miquel su triangolo e su quadrilatero. Lemma della rotomotetia. Quadrilatero completo e rotomotetie presenti nella configurazione. Altre applicazioni di inversione. mistilinei,
\end{short}

Angoli orientati ed esercizi/complementi sui quadrilateri ciclici. 

Punto di Miquel riferito a una terna di punti presi sui lati di un triangolo. Punto di Miquel riferito a un quadrilatero. Facendo opportuno riferimento all'esercizio 2 della sezione \textbf{GM-1}, osservare che il punto di Miquel di un quadrilatero $ABCD$ è il centro della \emph{spilar similarity} che manda $AB$ in $DC$ o $AD$ in $BC$. Il quadrilatero $ABCD$ è ciclico se e solo se il punto di Miquel $M$ sta su $QR$, dove $Q=AB\cap CD$ e $R=AD\cap BC$.

Nel caso di ciclicità:
\begin{itemize}
	\item $OM$ è perpendicolare a $QR$, essendo $O$ il circocentro di $ABCD$;
	\item $A,C,M,O$ e $B,D,M,O$ sono conciclici;
	\item $AC$, $BD$ e $OM$ sono concorrenti in $P$;
	\item $MO$ biseca $\angle CMA$ e $\angle BMD$;
	\item $P$ e $M$ sono inversi rispetto alla circonferenza circoscritta al quadrilatero $ABCD$.
\end{itemize} 

Un'avventura mistilinea: considerati quattro punti in senso antiorario su una circonferenza $\Gamma$ ($A,B,C,D$) ed essendo $P=AC\cap BD$, considero $\omega$ tangente ai segmenti $AP$ e $BP$ e a $\Omega$ rispettivamente in $E$, $F$ e $T$. Provare le seguenti:
\begin{itemize}
	\item $TE$ biseca l'arco $AC$ che contiene $D$;
	\item Detto $I$ l'incentro di $ABC$, $IFTB$ è ciclico e $I\in EF$
	\item Detto $J$ l'incentro di $APB$ allora $TJFB$ è ciclico e $TJ$ biseca $\angle ATB$.
\end{itemize}

Ripasso delle proprietà base riguardanti l'inversione. $\sqrt{bc}$-inversione più simmetria: risoluzione di alcuni problemi.

\vspace{0.3cm}
\large{\textbf{Versione estesa - Senior 2019}}\normalsize

\vspace{0.3cm}

\begin{itemize}
	\item \textbf{Angoli Orientati}:
	\emph{Definizione}. L'angolo orientato $\angle(l,r)$ è l'angolo di cui si deve ruotare $l$ in senso antiorario perché coincida con $r$. L'angolo orientato $\angle ABC$ è, per definizione, l'angolo orientato $\angle(AB,BC)$. Può variare in $[0,\pi]$, e l'addizione viene intesa $\mod \pi$
	
	\emph{Proprietà}. 
	\begin{enumerate}
		\item $\angle(l,m)+\angle(m,l)=\pi$,
		\item $\angle ABC+\angle BCA+\angle CAB=\pi$,
		\item $\angle AOP + \angle POB=\angle AOB$,
		\item $A,B,C$ allineati se e solo se per un punto (o per tutti) $\angle XBC=\angle XBA$,
		\item $A,B,X,Y$ ciclico se esolo se $\angle AXB=\angle AYB$. 
	\end{enumerate}
	
	\textbf{ATTENZIONE:} una cosa da spiegare ai ragazzi è che in una gara internazionale, la cosa fondamentale da scrivere è ``Angolo $\mod \pi$'' piuttosto che ``Angolo orientato'' (che presuppone solo il segno e non la periodicità).

	\item \textbf{Teorema di Miquel}:
	\emph{Versione triangolare}. Dato $ABC$ un triangolo e $D,E,F$ punti rispettivamente sulle rette dei lati $BC$, $CA$ e $AB$, le circonferenze $AEF$, $BDF$ e $CDE$ concorrono. 
	
	\emph{Versione quadrangolare}. Date $r_1,r_2,r_3,r_4$ rette che si intersecano in 6 punti (\emph{Cosa succede nei casi degeneri?}) siano $A_{ij}\doteq r_i\cap r_j$ i punti di intersezione. Le circonferenze circoscritte ai triangoli $A_{12}A_{23}A_{31}$, $A_{12}A_{24}A_{14}$, $A_{13}A_{34}A_{14}$ e $A_{23}A_{34}A_{24}$ concorrono.

\item \textbf{Rotomotetie}: Dati due punti $A,B,C,D$ distinti sul piano, ricordare che esiste una e una sola rotomotetia che porta $A$ in $B$ e $C$ in $D$ se e solo se $AC$ non è parallelo a $BD$. In tal caso, detto $X\doteq AC\cap BD$, il centro di tale rotomotetia è $W$ l'interesezione delle circonferenze circoscritte a $AXB$ e $CXD$. Discutere cosa succede se $X$ coincide con uno dei 4 punti $A,B,C,D$ (una delle due circonferenze da tracciare diviene tangente ad uno dei segmenti).

\emph{Dagli Esercizi:}
Discutere un caso degenere \item Sia $ABC$ un triangolo. Mostrare che il centro della (unica) rotomotetia che manda $B$ in $A$ e $A$ in $C$ è sulla simmediana uscente da $A$. 

\item \textbf{Inversione $\sqrt{bc}$ e simmetria}. Usare come pretesto la parte finale dell'esercizio precedente per introdurre questa tecnica. 

%\textbf{Soluzione:} Viste le considerazioni fatte nella parte sulla rotomotetia, tale centro è l'intersezione $X$ fra la circonferenza che passa per $A$ e $B$ e tange $AC$ in $A$ e la circonferenza che passa per $A$ e $C$ e tange $AB$ in $A$.
%Ora (anticipazione) faccio una inversione di centro in $A$ e raggio $\sqrt{AB\cdot AC}$ più una simmetria rispetto alla bisettrice. Si ha che $B\to C$ e $C\to B$. La circonferenza $ABX$ va in una retta passante per $C$ e parallela ad $AB$ e la circonferenza $ACX$ va in una retta passante per $B$ e parallela ad $C$. Dunque $X$ va in un punto sulla mediana e dunque prima era sulla simmediana. 
\emph{Dai problemi:}
\item \textbf{[USAMO 2006]} 
Sia $ABCD$ un quadrilatero e siano $E$ e $F$ punti su $AD$ e $BC$ rispettivamente tali che $AE/ED=BF/FC$.
La retta $FE$ incontra $BA$ e $CD$ in $S$ e $T$ rispettivamente. 

Mostrare che le circonferenze circoscritte ai triangoli $SAE$, $SBF$, $TCF$ e $TDE$ passano per uno stesso punto.

%\textbf{Soluzione:} Innanzitutto Per Miquel sui quadrangoli sappiamo che le circonferenze circoscritte ai triangoli $SAE$, $SBF$ e $ABX$ concorrono; così come le circonferenze circoscritte ai triangoli $TCF$, $TDE$ e $XCD$. Quindi, essendo la tesi vera, l'intersezione delle quattro circonferenze deve essere l'altra intersezione fra le circonferenze $XAB$ e $XCD$. Sia $Y$ questa intersezione. Per quanto visto sulle rotomotetie, questo punto è il centro della rotomotetia che manda $BC$ in $AD$ e dunque, visti i rapporti fra i segmenti, deve mandare $F$ in $E$. Allora 
%$$
%\angle YFB=\angle YEX
%$$
%e dunque $Y$ è sulla circonferenza circoscritta a $XEF$ e pertanto, per Miquel, anche su quella circoscritta a $SAE$ e $SBF$.

\textbf{Approfondimento sulla configurazione di Miquel:} Sia $ABCD$ un quadrilatero con $Q\doteq AB\cap CD$ e $R\doteq AD\cap BC$. Mostrare che 
\begin{enumerate}
	\item $M\in QR$ se e solo se $ABCD$ ciclico, 
	\item $ABCD$ ciclico implica $OM\perp QR$,
	\item $ABCD$ ciclico implica $MAOC$ e $BODM$ ciclici, 
	\item $ABCD$ ciclico implica $MO$ biseca $AMC$ e $BMD$,
	\item $ABCD$ ciclico implica $O$, $M$ e $AC\cap BD$ allineati. Da ciò, invertendo nella circonferenza circoscritta ad $ABCD$, si ottiene che $P$ e $M$ sono uno l'inverso dell'altro.
\end{enumerate}

\textbf{Fatti sulle circonferenze mistilinee:} \emph{Le ceviane delle mistilinee concorrono}.
Sia $ABC$ un triangolo inscritto in $\Omega$ e $\omega_A$ una circonferenza tangente internamente a $\Omega$ in $T_A$ e tangente anche ad $AB$ e $AC$ in $B_1$ e $C_1$ rispettivamente. Da un'inversione $\sqrt{bc}$ più simmetria rispetto alla bisettrice mostrare che $AT_A$ e cicliche concorrono - nel coniugato isogonale del punto di Nagel. Osservare che $AT_A$ contiene il centro di similitudine esterno fra $\omega$, la circonferenza inscritta ad $ABC$, e $\Omega$ e dunque il coniugato isogonale del punto di Nagel è il centro di similitudine esterno fra $\omega$ e $\Omega$. 

\emph{Altri fatti}. Con riferimento alla precedente,  mostrare che
\begin{enumerate}
	\item L'incentro di $ABC$, $I$, è su $B_1C_1$. Ciò segue da Pascal su $M_{AB}T_AM_{AC}BAC$, dove $M_{AB}$ e $M_{AC}$ sono i punti medi degli archi $AB$ e $AC$ che non contengono $C$ e $B$; più il lemma che $T_A,B_1,M_{AB}$ sono allineati,  
	\item L'incentro $I$, dopo l'inversione, va nell'excentro $I_A$. Ciò segue dal fatto che $AI\cdot AI_A=AB\cdot AC$ il quale a sua volta segue dal fatto che $BCII_A$ è ciclico più un semplice conto di angoli. Da ciò segue che $BT_AB_1I$ e $CT_AC_1I$ sono ciclici,
	\item Dal punto precedente segue $T_AI$ biseca $\angle BT_AC$, 
	\item $T_AM_A$, $BC$ e $B_1C_1$ concorrono applicando Pascal su $BCM_{AB}T_AM_AA$, 
	\item Da due punti sopra viene $M_AT_A$ perpendicolare a $T_AI$ e dunque $T_AI$ interseca $\Omega$ nel diametralmente opposto di $M_A$.
\end{enumerate}
	\emph{Dai problemi:}
	\item \textbf{[EGMO 2013 - 5]}
	Sia $\Omega$ la circonferenza circoscritta ad un triangolo $ABC$. La circonferenza $\omega$ è tangente ai lati $AC$ e $BC$ e internamente alla circonferenza $\Omega$ in un punto $P$. Una retta parallela ad $AB$ che interseca l'interno del triangolo $ABC$ è tangente a $\omega$ in $Q$.
	
	Mostrare che $\angle ACP = \angle QCB$.
	
	%\textbf{Soluzione:} Per inversione più simmetria $AP$ è la simmetrica della ceviana di Nagel rispetto alla bisettrice, che , per omotetia, coincide con $AQ$.
\item \textbf{Qualche problema sull'inversione:} 
\emph{Dagli esercizi:}

\textbf{[Teorema di Feuerbach]} Mostrare che la circonferenza di Feuerbach è tangente alla circonferenza inscritta e alle circonferenze exinscritte.

%\emph{Suggerimento:} Sia $M$ il punto medio di $BC$ e $D$ e $G$ rispettivamente i punti in cui la circonferenza inscritta e quella ex-inscritta opposta ad $A$ incontrano $BC$. Invertire in $M$ con raggio $MD$.

\emph{Dai problemi:}

\textbf{[IMO 2015 - 3]} Sia $ABC$ un triangolo acutangolo con $AB > AC$. Sia $\Gamma$ la sua circonferenza circoscritta, $H$ il suo ortocentro, e $F$ il piede dell'altezza condotta da $A$. Sia $M$ il punto medo di $BC$. Sia $Q$ il punto su $\Gamma$ tale che $\angle HQA = 90^{\circ}$ e sia $K$ il punto su $\Gamma$ tale che $\angle HKQ = 90^{\circ}$. Assumiamo che $A$, $B$, $C$, $K$ e $Q$ sono tutti distinti e giacciono su $\Gamma$ in quest'ordine. 

Mostrare che le circonferenze circoscritte ai triangoli $KQH$ e $FKM$ sono fra loro tangenti.
%\begin{sol}Inversione di centro H che fissa la circonferenza circoscritta ad ABC. K'Q' diviene perpendicolare ad AK' che è l'asse di F'M' e dunque K'Q' è la tangente a K' nella circonferenza circoscritta a F'M'K'.
%\end{sol}

\end{itemize}

\subsection{Esercizi}
\begin{enumerate}
	\item Sia $ABCD$ un quadrilatero ciclico di circocentro $O$. Le rette $AB$ e $CD$ si intersecano in $E$, le rette $AD$ e $BC$ si intersecano in $F$ e le rette $AC$ e $BD$ si intersecano in $P$. Sia $K$ l'intersezione di $EP$ e $AD$ e $M$ la proiezione di $O$ su $AD$.
	
	Mostrare che $BCMK$ è ciclico. 
	\item Sia $ABC$ un triangolo. Mostrare che il centro della (unica) rotomotetia che manda $B$ in $A$ e $A$ in $C$ è sulla simmediana uscente da $A$. 
	
	\textbf{Soluzione:} Viste le considerazioni fatte nella parte sulla rotomotetia, tale centro è l'intersezione $X$ fra la circonferenza che passa per $A$ e $B$ e tange $AC$ in $A$ e la circonferenza che passa per $A$ e $C$ e tange $AB$ in $A$.
	Ora (anticipazione) faccio una inversione di centro in $A$ e raggio $\sqrt{AB\cdot AC}$ più una simmetria rispetto alla bisettrice. Si ha che $B\to C$ e $C\to B$. La circonferenza $ABX$ va in una retta passante per $C$ e parallela ad $AB$ e la circonferenza $ACX$ va in una retta passante per $B$ e parallela ad $C$. Dunque $X$ va in un punto sulla mediana e dunque prima era sulla simmediana. 
	\item \textbf{[Fatti su triangolo con mistilinea]} Sia $ABC$ un triangolo iscritto in una circonferenza $\Gamma$ e sia $\gamma$ la circonferenza tangente ai segmenti $AB$, $AC$ e a $\Gamma$ rispettivamente in $E$, $F$ e $T$. Sia $I$ l'incentro di $ABC$. Sia $M$ il punto medio dell'arco $BC$ che non contiene $A$. Sia $V$ l'intersezione di $AT$ con $EF$. 
	
	Mostrare che:
	\begin{itemize}
		\item $I\in EF$ e $IE=IF$;
		\item $MT$, $EF$ e $BC$ sono concorrenti;
		\item $\angle BVE=\angle CVF$.
	\end{itemize}

	\item \textbf{[Teorema di Sawyama-Thébault]} 
	Sia $ABC$ un triangolo di incentro $I$ e sia $D$ un punto sul lato $BC$. Sia $P$ (rispettivamente $Q$) il centro della circonferenza che tange i segmenti $AD$ e $DC$ (rispettivamente $DB$) e la circonferenza circoscritta ad $ABC$. 
	
	Mostrare che $P$, $Q$ e $I$ sono allineati.
	\item \textbf{[NUSAMO 2015/2016 - 5]}
	Sia $ABC$ un triangolo, $I_A$ l'excentro opposto ad $A$
	e $I$ il suo incentro. Sia $M$ il circocentro del triangolo $BIC$ e sia $G$ la proiezione di $I_A$ su $BC$.
	Sia, infine, $P$ l'intersezione fra la circonferenza circoscritta di $ABC$ e la circonferenza di diametro $AI_A$. 
	
	Mostrare che $M$, $G$ e $P$ sono allineati.
	\begin{sol}
	Inversione nella circonferenza circoscritta a BIC che ha centro in M
    \end{sol}

	\item \emph{[Copiato in GB]} Siano $A$, $B$ e $C$ tre punti allineati e supponiamo che $P$ sia un punto qualsiasi del piano distinto dai precedenti 3. 
	
	Mostrare che i circocentri dei triangoli $PAB$, $PAC$, $PBC$ e $P$ sono conciclici.
	\item \emph{[Copiato in GB]} Sia $ABC$ un triangolo con ortocentro $H$ e siano $D$, $E$ e $F$ i piedi delle altezze che cadono sui lati $BC$, $CA$ e $AB$ rispettivamente. Sia $T=EF\cap BC$.
	
	Mostrare che $TH$ è perpendicolare alla mediana condotta da $A$.
    %inversione di centro A e raggio AH\cdot HD. La tesi diventa equivalente a mostrare che la circonferenza per D (immagine di H), per l'intersezione della circoscritta con AEF (immagine di T) e A ha la retta AM come diametro. Questo segue perché in effetti M,T',A,D sono ciclici
    \item \textbf{[Teorema di Feuerbach]}\emph{[Copiato in GB]} Mostrare che la circonferenza di Feuerbach è tangente alla circonferenza inscritta e alle circonferenze exinscritte.
    
    \emph{Suggerimento:} Sia $M$ il punto medio di $BC$ e $D$ e $G$ rispettivamente i punti in cui la circonferenza inscritta e quella ex-inscritta opposta ad $A$ incontrano $BC$. Invertire in $M$ con raggio $MD$.
    %Per prima cosa si nota che il piede della perpendicolare e il piede della bisettrice su BC si scambiano perché MH\cdot MI=MD^2. Inoltre si mostra passando per la circoscritta che la retta immagine della circonferenza dei nove punti fa un angolo di beta - gamma con BC. Dunque la cfr dei nove punti va nella simmetrica della retta BC rispetto alla bisettrice che tange entrambe le circonferenze inscritta ed exinscritta. Inoltre queste due si scambiano
    
    \item La circonferenza inscritta nel triangolo $ABC$ è tangente a $BC$, $CA$ e $AB$ in $M$, $N$ e $P$ rispettivamente. 
    
    Mostrare che il circocentro e l'incentro di $ABC$ sono allineati con l'ortocentro di $MNP$.
    %Inversione nella circonferenza inscritta 
\end{enumerate}

\subsection{Problemi}
\begin{enumerate}
	\item \textbf{[USA TST 2007 - 5]} Il triangolo $ABC$ è inscritto in una circonferenza $\Gamma$. Le tangenti a $\Gamma$ condotte da $B$ e $C$ si intersecano in $T$. Il punto $S$ è sulla retta $BC$ dimodoché $AS\perp AT$. Siano $B_1$ e $C_1$ sulla retta $ST$ dimodoché $B_1T=BT=C_1T$.
	
	Mostrare che $ABC$ e $AB_1C_1$ sono simili.
	\item \textbf{[IMO 2005 - 5]}
	Sia $ABCD$ un quadrialtero convesso con $BC=DA$ e $BC$ non parallelo a $DA$. Siano $E$ e $F$ su $BC$ e $DA$ rispettivamente tali che $BE=DF$. Siano $P$ l'intersezione di $AC$ e $BD$, $Q$ l'intersezione di $BD$ e $EF$ e $R$ l'intersezione di $EF$ e $AC$.

	Mostra che, al variare di $E$ e $F$, la circonferenza circoscritta al triangolo $PQR$ passa per un punto fisso (oltre $P$). 
	
	\item \textbf{[?]}  Sia $ABC$ un triangolo e siano $D$ e $E$ i piedi delle altezze relative ad $A$ e $B$, rispettivamente,  le quali siintersecano  in $H$.   Sia $M$ il  punto  medio  di $AB$ e  supponiamo  che  le  circonferenze circoscritte a $ABH$ e $DEM$ si intersechino nei punti $P$ e $Q$ (con $P$ e $A$ sullo stesso lato di $CH$).
	%PreIMO Mattino 4 2016
	
	Mostrare che le rette $PH$ e $MQ$ si incontrano sulla circonferenza circoscritta ad $ABC$. 
	\item \textbf{[IMO SL 2006 - 9]}
	Sui lati $BC$, $CA$ e $AB$ di un triangolo $ABC$ si scelgano tre punti $A_1$, $B_1$ e $C_1$ rispettivamente. Le circonferenze circoscritte a $AB_1C_1$, $BC_1A_1$ e $CA_1B_1$ intersecano la circonferenza circoscritta ad $ABC$ in $A_2$, $B_2$ e $C_3$ rispettivamente. Siano, inoltre, $A_3$, $B_3$ e $C_3$ rispettivamente i simmetrici di $A_1$, $B_1$ e $C_1$ rispetto ai punti medi dei lati del triangolo su cui giacciono. 
	
	Mostrare che i triangoli $A_2B_2C_2$ e $A_3B_3C_3$ sono simili.
	
	\begin{sol}$A_2$ è il centro della spilar similiarity che porta $BC_1$ in $CB_1$ quindi $A_2C/A_2B=B_1C/C_1B=AB_3/AC_3$ da cui $A_2BC$ è
	simile ad $AC_3B_3$ e da qui sono angoli 
	\end{sol}
	\item \textbf{[EGMO 2013 - 5]}
	Sia $\Omega$ la circonferenza circoscritta ad un triangolo $ABC$. La circonferenza $\omega$ è tangente ai lati $AC$ e $BC$ e internamente alla circonferenza $\Omega$ in un punto $P$. Una retta parallela ad $AB$ che interseca l'interno del triangolo $ABC$ è tangente a $\omega$ in $Q$.
	
	Mostrare che $\angle ACP = \angle QCB$.
	
	\textbf{Soluzione:} Per inversione più simmetria $AP$ è la simmetrica della ceviana di Nagel rispetto alla bisettrice, che , per omotetia, coincide con $AQ$.
	\item \textbf{[IMO SL 2003 - 4]}
	 Siano  $\Gamma_1$, $\Gamma_2$, $\Gamma_3$, $\Gamma_4$ 
	 circonferenze distinte tali che
	 $\Gamma_1$ e $\Gamma_3$ (così come $\Gamma_2$ e $\Gamma_4$) siano tangenti esternamente in $P$. Supponiamo che $\Gamma_1$ e $\Gamma_2$, $\Gamma_2$ e $\Gamma_3$, $\Gamma_3$ e $\Gamma_4$, $\Gamma_4$ e $\Gamma_1$ si intersechino in $A$, $B$, $C$ e $D$ rispettivamente e che nessuno di questi punti sia $P$.
	 
	 Mostrare che 
	 $$
	 \frac{AB\cdot BC}{AD\cdot DC}=\frac{PB^2}{PD^2} .
	 $$
	 
	 \item \textbf{[IMO 2015 - 3]} Sia $ABC$ un triangolo acutangolo con $AB > AC$. Sia $\Gamma$ la sua circonferenza circoscritta, $H$ il suo ortocentro, e $F$ il piede dell'altezza condotta da $A$. Sia $M$ il punto medo di $BC$. Sia $Q$ il punto su $\Gamma$ tale che $\angle HQA = 90^{\circ}$ e sia $K$ il punto su $\Gamma$ tale che $\angle HKQ = 90^{\circ}$. Assumiamo che $A$, $B$, $C$, $K$ e $Q$ sono tutti distinti e giacciono su $\Gamma$ in quest'ordine. 
	 
	 Mostrare che le circonferenze circoscritte ai triangoli $KQH$ e $FKM$ sono fra loro tangenti.
	 \begin{sol}Inversione di centro H che fissa la circonferenza circoscritta ad ABC. K'Q' diviene perpendicolare ad AK' che è l'asse di F'M' e dunque K'Q' è la tangente a K' nella circonferenza circoscritta a F'M'K'.
	 \end{sol}
 
 \item \textbf{[USAMO 2006]} 
 Sia $ABCD$ un quadrilatero e siano $E$ e $F$ punti su $AD$ e $BC$ rispettivamente tali che $AE/ED=BF/FC$.
 La retta $FE$ incontra $BA$ e $CD$ in $S$ e $T$ rispettivamente. 
 
 Mostrare che le circonferenze circoscritte ai triangoli $SAE$, $SBF$, $TCF$ e $TDE$ passano per uno stesso punto.

 \textbf{Soluzione:} Innanzitutto Per Miquel sui quadrangoli sappiamo che le circonferenze circoscritte ai triangoli $SAE$, $SBF$ e $ABX$ concorrono; così come le circonferenze circoscritte ai triangoli $TCF$, $TDE$ e $XCD$. Quindi, essendo la tesi vera, l'intersezione delle quattro circonferenze deve essere l'altra intersezione fra le circonferenze $XAB$ e $XCD$. Sia $Y$ questa intersezione. Per quanto visto sulle rotomotetie, questo punto è il centro della rotomotetia che manda $BC$ in $AD$ e dunque, visti i rapporti fra i segmenti, deve mandare $F$ in $E$. Allora 
 $$
 \angle YFB=\angle YEX
 $$
 e dunque $Y$ è sulla circonferenza circoscritta a $XEF$ e pertanto, per Miquel, anche su quella circoscritta a $SAE$ e $SBF$.
 

		
\end{enumerate}

\clearpage


 \begin{thebibliography}{9}
	\bibitem[1]{} Gunmay Anda, \emph{Inversion on the fly}, \url{http://services.artofproblemsolving.com/download.php?id=YXR0YWNobWVudHMvNy85LzRiMmFiYzk1NTgxNjQyZGNhNjEzZDkxOGQ0OTFmN2UyYWFlMDc3LnBkZg==&rn=SW52ZXJzaW9uLnBkZg==}
	\bibitem[2]{} Du\v{s}an Djuki\' c, \emph{Inversion}, \url{http://memo.szolda.hu/feladatok/inversion_ddj.pdf}
	\bibitem[3]{} Paul Yiu, \emph{Introduction to the geometry of the triangle}, \url{http://math.fau.edu/yiu/YIUIntroductionToTriangleGeometry121226.pdf}
	\bibitem[4]{} Marko Radovanovi\'c, \emph{Complex numbers in geometry}, 
	\url{https://www.google.com/url?sa=t&rct=j&q=&esrc=s&source=web&cd=1&ved=2ahUKEwjZiqjuhLDkAhUN16QKHZTNBnAQFjAAegQIAhAC&url=http\%3A\%2F\%2Fservices.artofproblemsolving.com\%2Fdownload.php\%3Fid\%3DYXR0YWNobWVudHMvOS9iLzZhNGM2M2Y0NzZiNGY3MWE3ZTI0ZTRiY2Y4OGIwMzhiN2IyNzFhLnBkZg\%3D\%3D\%26rn\%3DbWFya28tcmFkb3Zhbm92aWMtY29tcGxleC1udW1iZXJzLWluLWdlb21ldHJ5LnBkZg\%3D\%3D&usg=AFQjCNFBeoyb2eMJWQnC3Q7qMMS3okG1Kw}
	
	%\url{http\%3A\%2F\%2Fservices.artofproblemsolving.com\%2Fdownload.php\%3Fid\%3DYXR0YWNobWVudHMvOS9iLzZhNGM2M2Y0NzZiNGY3MWE3ZTI0ZTRiY2Y4OGIwMzhiN2IyNzFhLnBkZg\%3D\%3D\%26rn\%3DbWFya28tcmFkb3Zhbm92aWMtY29tcGxleC1udW1iZXJzLWluLWdlb21ldHJ5LnBkZg\%3D\%3D&usg=AFQjCNFBeoyb2eMJWQnC3Q7qMMS3okG1Kw}
	\bibitem[5]{} Milivoje Luki\'c, \emph{Projective geometry}, \url{http://memo.szolda.hu/feladatok/projg_ml.pdf}
	\bibitem[6]{} Ercole Suppa, \emph{Divisione armonica}, \url{http://www.dma.unifi.it/~mugelli/Incontri_Olimpici_2010/19-Geometria-Testi/02-Ercole_Suppa-Divisione_Armonica.pdf}
	\bibitem[7]{} Yufei Zaho, \emph{Cyclic quadrilaterals -- the big picture}, \url{http://yufeizhao.com/olympiad/cyclic_quad.pdf}
	\bibitem[8]{} Yufei Zaho, \emph{Circles}, \url{http://yufeizhao.com/olympiad/imo2008/zhao-circles.pdf}
	\bibitem[9]{} \emph{Mathlinks}, \url{https://artofproblemsolving.com/wiki/index.php?title=MathLinks}
	\bibitem[10]{napoleoncomplex}Cut the Knot \emph{Napoleon Theorem - Proof with complex numbers} \url{ https://www.cut-the-knot.org/proofs/napoleon_complex2.shtml}
	\bibitem[11]{echencomplex} Evan Chen - Complessi \url{http://web.evanchen.cc/handouts/cmplx/en-cmplx.pdf}
	\bibitem[12]{echengeometry} Evan Chen - Dispensa di Geometria (basic+medium) \url{http://web.evanchen.cc/textbooks/tr011ey.pdf}
	\bibitem[13]{engeofigures} Arseniy Akopyan \emph{Geometry in Figures}
	\bibitem[14]{} Raccolta di problemi di geometria da gare internazionali \url{https://imogeometry.blogspot.com/p/blog-page_2.html}
\end{thebibliography}
 
 
\end{document}
