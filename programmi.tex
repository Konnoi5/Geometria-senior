\section{Programmi Basic}

\subsection{GB - 1 [Metodi Algebrici]}
\begin{short}
 Luoghi di punti con la geometria analitica (es: apollonio, luogo degli ortocentri) e scelte opportune di coordinate; distanze con i prodotti scalari e scrittura di vari punti con i vettori; rette e circonferenze con i complessi (e corde e tangenti); applicazioni della trigonometria.
\end{short}

Ricapitolazione veloce della geometria analitica: piano cartesiano, equazione di una retta ($y=mx+k$ o $Ax+By+C=0$), coefficiente angolare, rette perpendicolari.\\
Formula della distanza, Equazione di una circonferenza come luogo di punti a distanza fissata.\\
Applicazioni: luogo degli ortocentri [e problema tipo da vettori]

Complessi: Sono simili alle coordinate cartesiane, sono formati da parte reale e immaginaria. Addizione e moltiplicazione. Forma polare di un numero complesso. Coniugato e sue proprietà. \\
Formule per allineamento, perpendicolarità, ciclicità.



Trigonometria: Recap delle formule in un triangolo. 

IMOSL 2015 G1, Teorema di Stewart, esercizi con tanti teoremi del seno in un triangolo. 
Problemi che utilizzano $R=abc/4S$ e cose simili, formula di erone(?), esercizio G1 - 12

Analitica/Complessi: servono problemi con riferimento figo da usare.



\vspace{0.3cm}
\large{\textbf{Versione estesa - Senior 2019}}\normalsize
Ricapitolazione di analitica. Il piano cartesiano è formato da coppie di punti $(x,y)$. L'equazione di una retta è $y=mx+k$ o $Ax+By+C=0$. Il coefficiente angolare è $m$ e indica la pendenza di una retta; se passa per $(x_1,y_1),(x_2,y_2)$ il coefficente angolare è $m=\frac{y_2-y_1}{x_2-x_1}$. Due rette sono perpendicolari se il prodotto dei coefficienti angolari fa $-1$.\\

\vspace{0.3cm}

Esercizio sul luogo degli ortocentri.


\textbf{Trigonometria}\\
Recap di teorema dei seni, teorema del coseno. Calcolare la lunghezza della mediana. Teorema di Stewart. 
IMOSL 2015 G1

\vspace{0.4cm}
\textbf{Vettori}\\


\vspace{0.4cm}
\textbf{Complessi}\\
Sono numeri della forma $a+bi$, con $a,b$ reali e $i$ tale che $i^2=-1$. Si rappresentano nel piano complessi in maniera simile al piano cartesiano. Parte reale e immaginaria. Somma e moltiplicazione di numeri complessi, (divisione).\\
Scrittura in forma polare, passaggio da forma polare a cartesiana. Coniugato. Moltiplicare=ruotare. 


\vspace{0.4cm}
\Large \textbf{Problemi autonomi}\normalsize
\begin{itemize}
 \item EGMO 2013 - 1: Analitica, vettori, complessi
 \item ITA TST 2016 - 1
\end{itemize}





\subsection{GB - 2 [Trasformazioni]}


\begin{short}
Isometrie: Traslazione, Simmetria, Rotazione. Similitudine. Scrittura di queste trasformazioni in complessi. [$\bigstar$ Affinità] 
Inversione Circolare. Inversione + Simmetria in un triangolo.
\end{short}



\vspace{0.3cm}
\textbf{Isometrie. }Le isometrie sono trasformazioni che conservano la distanza. Le figure mantengono la stessa forma: le rette vanno in rette, circonferenze in circonferenze, poligoni in poligoni, gli angoli mantengono la misura. 

Le isometrie più importanti sono traslazione, riflessione e rotazione. 

La traslazione si definisce con un vettore $\vec{v}$, che manda ogni punto $P$ in $P+\vec{v}$ (in cartesiane e in complessi).

La rotazione si definisce tramite un centro $C$ e un angolo $\alpha$ tra 0 e 360.  In complessi, se il centro è l'origine, il punto z viene mandato in $z\cdot e^{i\alpha}$; se il centro è un altro punto, allora bisogna fare una traslazione, rotazione e traslare indietro: $z\rightarrow (z-c)\cdot e^{i\alpha}+c$.

La riflessione si definisce tramite una retta $r$, ogni punto viene mandato nel suo simmetrico rispetto a questa retta. La riflessione inverte l'orientazione a differenza della traslazione e della rotazione.

Come per la rotazione, per scrivere in complessi la riflessione si compongono tre trasformazioni: si sceglie un punto $c$ sulla retta e sia $\alpha$ l'angolo che forma con l'asse reale, allora $z$ va in $\overline{(z-c)e^{-i\alpha}}\cdot e^{i\alpha}+c=\overline{(z-c)}\cdot e^{2i\alpha}+c$.

\textit{Esempio easy:}  $ABC$ triangolo, $H$ ortocentro, $AH$ interseca $BC$ in $D$ e la circonferenza circoscritta in $N$. Dimostrare $DH=HN$.

\vspace{0.3cm}
[$\bigstar$ Fatti sparsi: 

1) ogni isometria è composizione di al massimo tre riflessioni.

2)Si possono dividere in due gruppi, a seconda se mantengono l'orientamento oppure no. Quelle che mantengono l'orientamento sono traslazione e rotazione, quelle che lo invertono sono riflessione e glissoriflessione(=traslazione lungo una retta e riflessione lungo quella retta), questa sono tutte le isometrie possibili

3) rotazione di $\alpha$+rotazione di $\beta$ = rotazione di $\alpha+\beta$ se $\alpha+\beta \neq 0$, altrimenti è traslazione. Traslazione+rotazione di $\alpha$=rotazione di $\alpha$ con un altro centro. analogamente per rotazione+traslazione]

\vspace{0.4cm}

$\bigstar$ \textbf{Affinità}

\vspace{0.4cm}

\textbf{Inversione.} A ogni punto $P$ associa $P'$ tale che $OP\cdot OP'=R^2$. Costruzione con le tangenti (per punto esterno) e al contrario per punto interno. È involutiva, scambia interno ed esterno, i punti sulla circonferenza di inversione rimangono gli stessi. 

Le rette per l'origine rimangono rette per l'origine, circonferenze per l'origine diventano rette non per l'origine [questo si può dimostrare], circonferenze non per l'origine diventano circonferenze non per l'origine. Calcolo di $A'B'=\frac{AB\cdot R^2}{OA \cdot OB}$, dire che $OAB$ e $OB'A'$ sono simili. L'inversione conserva gli angoli tra le curve, ma non gli angoli tra punti.

\textit{Esempio} Teorema di Tolomeo.

In complessi, l'inversione nell'origine di raggio $R$ manda $z$ in $R^2\cdot \overline{z}^{-1}$. 


[$\bigstar$ Si può fare un ponte tra potenze e inversione: una circonferenza $\gamma$ è invariata per inversione in $O$ di raggio $R$ se $pow_{\gamma}(O)=R^2$. er esempio $\gamma$ circoscritta ad $ABC$, $P$ è l'intersezione della tangente in $A$ con $BC$, allora l'inversione in $P$ di raggio $PA$ scambia $B$ e $C$ e di conseguenza lascia invariata $\gamma$ ]

\vspace{0.4cm}
Inversione + Simmetria


\subsection{GB - 3 [Sintetica]}
\begin{short}
 Circonferenza di Apollonio. Circonferenza di Feuerbach. Simmediana. Segmenti di tangenza di Incerchi/Excerchi, punti di Gergonne e Nagel. Retta di simson. Applicazioni di potenze e assi radicali. 
\end{short}



\clearpage
\section{Programmi Medium}

\subsection{GM - 1 [Numeri complessi e coordinate baricentriche]}
Numeri complessi nella geometria euclidea. Si assume che si possegga una discreta maneggevolezza con il piano complesso.
Rapido ripasso sulla forma polare dei numeri complessi e significato geometrico delle operazioni.

Condizione di allineamento e scrittura dell'equazione di una retta per due punti. Condizione di parallelismo e scrittura della parallela ad una retta passante per un punto ad essa esterno. Condizione di perpendicolarità e scrittura della perpendicolare ad una retta passante per un punto ad essa esterno. Birapporto fra 4 numeri complessi e condizione di ciclicità.

Equazione di una generica circonferenza. Scelta classica delle coordinate: circonferenza circoscrita $\equiv$ circonferenza unitaria. Punti notevoli nella scelta classica delle coordinate. Esempio di quanto si semplificano i conti: intersezione di due corde generiche. Coordinate $u,v,w$ per l'incentro. 

\vspace{0.5cm}
Definizione di coordinate baricentriche. 

Come verificare l'allineamento di tre punti ed equazione di una retta generica. Intersezione di due rette. Area di un triangolo di cui si conoscono le coordinate dei vertici. Punto all'infinito di una retta. Quando due rette sono parallele?

Punti notevoli e notazione di Conway: baricentro, incentro, ortocentro, circocentro, excentri, nagel, gergonne, lemoine... Coniugati isogonali e coniugati isotomici.

Equazione della circonferenza circoscritta (come coniugato isogonale della retta all'infinito).
Equazione di una circonferenza in posizione generale. Equazione dell'asse radicale fra una circonferenza in posizione generale e la circonferenza circoscritta al triangolo referenziale: relazione di tale equazione con le potenze dei vertici del triangolo referenziale rispetto alla circonferenza in posizione generale. Formula di sdoppiamento per la tangente e la polare.

\vspace{0.3cm}
\large{\textbf{Versione estesa - Senior 2019}}\normalsize

\vspace{0.3cm}
\textbf{Numeri Complessi}:
\begin{itemize}
\item \textbf{Introduzione:} Un numero complesso si scrive come $z=a+bi$, dove $i^2=-1$ e $a,b$ sono numeri reali. Il numero $a$ si dice parte reale e il numero $b$ si dice parte immaginaria. Si può anche scrivere come $z=\rho e^{i\theta}$, dove $\rho>0$ è detto modulo e $0\leq \theta\leq 2\pi$ è detto argomento. Sul piano di Gauss, $\rho$ è la distanza del punto $(a,b)$ dall'origine e $\theta$ è l'angolo formato, in senso antiorario, col semiasse positivo delle $x$.

Identifichiamo un numero complesso con il punto $(a,b)$ del piano di Gauss e alle volte con il vettore che parte dall'origine e arriva ad $(a,b)$.

Per passare dalle coordinate polari a quelle cartesiane: $a=\rho \cos\theta$ e $b=\rho\sin\theta$. Viceversa $\rho=\sqrt{a^2+b^2}$ e $\theta=\arctan{\frac{b}{a}}$. 

\item \textbf{Operazioni:} Per quanto riguarda le operazioni, $z\to z+w$ corrisponde ad una traslazione del vettore $w$; $z\to zw$ - con $w=\rho e^{i\theta}$ corrisponde ad una rotazione in senso antiorario di $\theta$ più una omotetia di centro l'origine e fattore $\rho$; $z\to \bar{z}$ corrisponde ad una simmetria rispetto all'asse reale. 

\item \textbf{Angoli e similitudini:}

\emph{Osservazione}. Sia $g(z)\doteq \frac{z}{\bar z}$. Se $z=\rho e^{i\theta}$, allora $g(z)=e^{2i\theta}$. 

Dati tre numeri complessi (punti) nel piano di Gauss $a$, $b$ e $c$, detto $\theta$ l'angolo $\angle abc$ (ovvero l'angolo di cui ruotare $ab$ in senso antiorario attorno a $b$ perché la retta $ab$ coincida con $bc$ e in più $a$ e $c$ siano dalla stessa parte rispetto a $b$), si ha che esiste un numero reale $\rho>0$ tale che 
$$
c-b=(a-b)\rho e^{i\theta},
$$
dove $\rho$ non è altro che il rapporto fra le lunghezze dei segmenti $\overline{cb}$ e $\overline{ab}$. 

\textbf{Controlla} \emph{Conseguenza 1: Triangoli simili}. Se i triangoli $abc$ e $def$ sono ordinatamente simili, allora 
$$
\frac{c-b}{a-b}=\frac{\overline{cb}}{\overline{ab}}e^{i\angle abc}=\frac{\overline{fe}}{\overline{de}}e^{i\angle def}=\frac{f-e}{d-e},
$$
e vale anche il viceversa.

\emph{Conseguenza 2: Equazione dell'angolo}. Dall'osservazione, se $\theta$ è l'angolo $\angle abc$ si ha 
$$
e^{2i\theta}=g\left(\frac{c-b}{a-b}\right)=\frac{c-b}{a-b}\cdot \frac{\bar a-\bar b}{\bar c-\bar b}.
$$

Dunque per mostrare che $\angle abc=\angle def$ basta mostrare 
$$
\frac{c-b}{a-b}\cdot \frac{\bar a-\bar b}{\bar c-\bar b}=\frac{f-e}{d-e}\cdot \frac{\bar d-\bar e}{\bar f-\bar e}
$$
che è come dire che
$$
\frac{c-b}{a-b}\frac{d-e}{f-e} \qquad \mbox{è reale}.
$$
\item \textbf{Allineamenti, parallelismi e perpendicolarità:}

\emph{Allineamento.} Se $a$, $b$ e $c$ sono allineati, allora $\angle abc=\pi$ e dunque dall'equazione dell'angolo 
$$
\frac{c-b}{a-b}=\frac{\bar c-\bar b}{\bar a-\bar b},
$$
e vale anche il viceversa. 
\emph{Perpendicolarità 1.} Se $ab\perp bc$ allora dall'equazione dell'angolo 
$$
\frac{c-b}{a-b}=-\frac{\bar c-\bar b}{\bar a-\bar b}.
$$
\emph{Parallelismo.} Come esercizio, mostrare che $ab\parallel cd$ se e solo se 
$$
\frac{d-c}{b-a}=\frac{\bar d-\bar c}{\bar b-\bar a}.
$$
\emph{Perpendicoarità 2.} Come esercizio mostrare che $ab\perp cd$ se e solo se 
$$
\frac{d-c}{b-a}=-\frac{\bar d-\bar c}{\bar b-\bar a}.
$$
\item \textbf{Birapporti e ciclicità:}

\emph{Birapporto.} Dati quattro numeri complessi $z_1,z_2,z_3,z_4$ si dice birapporto $[z_1,z_2,z_3,z_4]$ la quantità
$$
\frac{z_1-z_2}{z_3-z_2}\cdot \frac{z_3-z_4}{z_1-z_4}.
$$
Mediante l'equazione dell'angolo è immediato notare che $[z_1,z_2,z_3,z_4] \in \mathbb R$ se e solo se $z_1z_2z_3z_4$ è ciclico.

\item \textbf{Circonferenza unitaria e scelta delle coordinate:}
 
\emph{Circonferenza unitaria e coordinate classiche.} Nel piano cartesiano la circonferenza unitaria ha equazione $z\bar z=1$. Molto spesso nella risoluzione dei problemi è utile settare la circonferenza circoscritta come circonferenza unitaria, dunque tutti i punti su essa soddisfano $z\bar z=1$ e $o$, il circocentro, diviene l'origine degli assi. Siccome vale $h+2o=3g$ in generale, visti i rapporti sulla retta di Eulero, e, sempre in generale, $g=\frac{a+b+c}{3}$, si ottiene che in questa scelta di coordinate $h=a+b+c$.

\emph{Coordinate dell'incentro} L'incentro è più difficile da gestire. In un problema con l'incentro conviene usare la notazione $u,v,w$. Infatti (dare come esercizio), dato un triangolo $abc$, esistono sempre tre numeri complessi $u,v,w$ tali che $a=u^2$, $b=v^2$, $c=w^2$ e l'incentro $i=-(uv+vw+uw)$. \emph{Hint:} Mostrare che esistono $u$, $v$ e $w$ tali che $a=u^2$, $b=v^2$, $c=w^2$ e i punti $-uv$, $-vw$, $-uw$ sono i punti medi degli archi $ab$, $bc$ e $ca$ che non contengono i terzi punti. 

\emph{Dagli esercizi:}

\item \textbf{[Seconda intersezione di due circonferenze in complessi]} Siano dati 4 punti $a, b, c, d$ nel piano complesso che non formano un parallelogrammo.

Mostrare che esiste una e una sola rotomotetia che
manda $a$ in $b$ e $c$ in $d$. Detto $x$ il centro di tale rotomotetia e $\alpha$ il numero complesso che rappresenta la rotomotetia, si ha
$$
x=\frac{ad-bc}{a-b-c+d}
$$
$$
\alpha=\frac{b-d}{a-c}.
$$

Mostrare che l'intersezione delle circonferenze circoscritte a $ABX$ e $CDX$ dove $AC$ e $BD$ sono segmenti non paralleli le cui rette si intersecano in $X$, è il centro della rotomotetia che manda $A$ in $B$ e $C$ in $D$. 

%\textbf{Soluzione:} Sia $x$ il centro della rotomotetia, $\alpha$ il numero complesso che rappresenta la rotomotetia - ovvero l'argomento di $\alpha$ è l'angolo di rotazione e il modulo di $\alpha$ è la ragione della rotomotetia. Se manda $a$ in $b$, allora 
%$$
%b-x = (a-x)\alpha ,
%$$
%e poiché manda $c$ in $d$ si ha anche 
%$$
%d-x = (c-x)\alpha.
%$$
%Dunque per confronto 
%$$
%\frac{b-x}{a-x}=\frac{d-x}{c-x}
%$$
%da cui 
%$$
%(b-x)(c-x)=(d-x)(a-x)\Rightarrow x=\frac{ad-bc}{a-b-c+d}.
%$$

%Svolgendo i calcoli si ha infine

%$$
%\alpha=\frac{b-d}{a-c}.
%$$
 
\emph{Dai problemi:}

\item \textbf{[BMO 2009 - 2]} Sia $MN$ una segmento parallelo al lato $BC$ del triangolo $ABC$, con $M$ sul lato $AB$ e $N$ sul lato $AC$. Le rette $BN$ e $CM$ si incontrano in $P$. Le circonferenze circoscritte a $BMP$ e $CNP$ si incontrano in due punti distinti $P$ e $Q$. 
 
 Mostrare che $\angle BAQ = \angle PAC$.
 
 %\textbf{Soluzione:} Diciamo che $a$ è l'origine del nostro piano di Gauss, mentre $b$ e $c$ sono due generici punti. Visto che $mn\parallel bc$ e $m\in ab$, $m\in ac$ si ha che esiste $\lambda \in \mathbb R$ tale che $m=\lambda b$ e $n=\lambda c$. Essendo $q$ il centro della rotomotetia che manda $m$ in $b$ e $c$ in $n$, allora 
 %$$
 %q=\frac{mn-bc}{m+n-b-c}=\frac{\lambda^2bc-bc}{\lambda b+\lambda c-b-c}=\frac{(\lambda+1)bc}{b+c}.
 %$$
 
 %Per trovare trovare $p$ basterebbe imporre $p\in mc$ e $p\in bn$. \emph{Proporlo come esercizio}. D'altra parte non ce n'è bisogno: infatti noi siamo interessati poi solo all'angolo $\angle CAP$ e dunque non tanto ci servono le coordinate di $P$ quanto capire chi è la retta $AP$, che è la mediana di $ABC$. Dunque possiamo dire che esiste un certo $\eta$ reale tale che 
% $$
 %p=\eta(b+c).
 %$$ 
 
 %Per l'equazione dell'angolo, se $\theta=\angle BAQ$ si ha 
 %$$
 %e^{2i\theta}=\frac{q-a}{b-a}\frac{\bar b-\bar a}{\bar q-\bar a}=\frac{c(\bar b+\bar c)}{\bar c(b+c)},
 %$$
 %mentre se $\theta'=\angle PAC$ si ha 
 %$$
 %e^{2i\theta'}=\frac{c-a}{p-a}\frac{\bar p-\bar a}{\bar c-\bar a}=\frac{c(\bar b+\bar c)}{\bar c(b+c)}.
 %$$
 
 %Da ciò, con un attimo di discussione, si ottiene che $\theta=\theta'$ che implica la tesi. 
 
 \item \textbf{[RMM 2012 - 2]} Sia $ABC$ un triangolo non isoscele e siano $D$, $E$ e $F$ rispettivamente i punti medi dei lati $BC$, $CA$ e $AB$. La circonferenza $BCF$ e la retta $BE$ si intersecano nuovamente in $P$ e la circonferenza $ABE$ e la retta $AD$ in $Q$. Le rette $DP$ e $FQ$ si incontrano in $R$. 
 
 Mostrare che il baricentro $G$ del triangolo $ABC$ giace sulla circonferenza circoscritta al triangolo $PQR$.
 
 %\textbf{Soluzione:} Per mostrare la ciclità è sufficiente mostrare che, detto $\theta=\angle GPD$ e $\theta'=\angle GQF$, si ha 
 %$$
 %\theta=\theta'.
 %$$  
 %Dall'equazione dell'angolo risulta che per fare ciò è sufficiente mostrare
 %$$
 %\frac{d-p}{g-p}\frac{g-q}{f-q}\in\mathbb R.
 %$$
 %Il problema è dunque spostato a trovare i punti $p$ e $q$. Qui usiamo un'osservazione sintetica. Si ha che 
 %$$
 %\angle GDE=\angle GAB = \angle QEG,
 %$$
 %dove la prima è vera per il parallelismo $AB\parallel ED$ e la seconda è vera poiché $ABEQ$ è ciclico. 
 %Analogamente si ha $\angle EQD=\angle GED$ e dunque i triangoli $GDE$ e $GEQ$ sono ordinatamente simili.
 %Dunque, scegliendo $g=0$, risulta, visto che $GD\cdot GQ=GE^2$,
 %$$
 %q=d\frac{|e|^2}{|d|^2}=\frac{e\bar e}{\bar d}
 %$$
 %e analogamente 
 %$$
 %p=\frac{f\bar f}{\bar e}.
 %$$
 %A questo punto 
 %$$
 %\frac{d-p}{g-p}\frac{g-q}{f-q}=\frac{(d\bar e-f\bar f)e\bar e}{(f\bar d-e\bar e)f\bar f}
 %$$
 %e poiché, essendo $g=0$, si ha $d+e+f=0$, la precedente espressione è uguale a 
 %$$
 %\frac{|e|^2}{|f|^2}
 %$$ 
 %che è un numero reale, come si voleva.
\end{itemize}
\textbf{Coordinate baricentriche}:
\begin{itemize}
	\item \textbf{Definizioni:}
	\emph{Terna omogenea}. Con $[x:y:z]$ indico una terna omogenea di numeri non tutti nulli, ovvero $[x:y:z]=[u:v:w]$ se e solo se esiste $k\in\mathbb R\setminus\{0\}$ tale che $u=kx$, $v=ky$ e $w=kz$. 
	
	\emph{Coordinate baricentriche}. Dato $ABC$ un triangolo e $P$ un punto sullo stesso piano di $ABC$, allora le coordinate baricentriche di $P$ sono 
	$$
	[ |BCP|: |CAP|: |ABP| ],
	$$
	dove $|\cdot|$ indica l'area con segno, ovvero è un numero che ha come modulo l'area di $\cdot$ e come segno $+$ o $-$ a seconda che il verso in cui sono scritti i vertici sia lo stesso o l'opposto rispetto a quello in cui sono assegnati $ABC$. 
	\item \textbf{Alcuni punti:} 
	\emph{Notazione di Conway}. Da ora in poi scriveremo
	$$
	S_A\doteq\frac{b^2+c^2-a^2}{2}, \quad S_B\doteq\frac{a^2+c^2-b^2}{2}, \quad 
	S_C\doteq\frac{a^2+b^2-c^2}{2}.
	$$ 
	
	Inoltre indichiamo con $a$ la lunghezza di $BC$, con $b$ la lunghezza di $AC$ e con $c$ la lunghezza di $AB$. Siano $\alpha,\beta,\gamma$ rispettivamente gli angoli in $A$, $B$ e $C$.
	
	I vertici hanno coordinate 
	$$
	A=[1:0:0], \quad B=[0:1:0], \quad C=[0:0:1],
	$$
	i punti medi hanno coordinate 
	$$
	M_{BC}=[0:1:1], \quad \mbox{e cicliche},
	$$
	i piedi delle bisettrici hanno coordinate 
	$$
	D_{BC}=[0:b:c], \quad \mbox{e cicliche},
	$$
	il baricentro ha coordinate 
	$$
	G=[1:1:1],
	$$
	e l'incentro ha coordinate 
	$$
	I=[a:b:c].
	$$
	Per l'ortocentro calcoliamo le aree
	$$
	H=\left[\frac{a}{\cos\alpha}:\frac{b}{\cos b}:\frac{c}{\cos\gamma}\right]=[\tan\alpha:\tan\beta:\tan\gamma]=[S_BS_C:S_CS_A:S_AS_B],
	$$
	dove nell'ultimo passaggio usiamo la notazione di Conway. Per i piedi delle altezze notiamo che in generale le tracce di un punto sono semplici da trovare e dunque
	$$
	H_{BC}=[0:S_C:S_B].
	$$
	Troviamo il circocentro 
	$$
	O=[\sin2\alpha:\sin2\beta:\sin2\gamma]=[a^2S_A:b^2S_B:c^2S_C].
	$$
	dove nell'ultimo passaggio usiamo la notazione di Conway.
	
	\emph{Esercizio:} Il coniugato isogonale di $[x:y:z]$ ha coordinate $\left[\frac{a^2}{x}:\frac{b^2}{y}:\frac{c^2}{z}\right]$ e quello isotomico ha coordinate $\left[\frac{1}{x}:\frac{1}{y}:\frac{1}{z}\right]$. 
	
	\emph{Esercizio:} Mostrare che le coordinate dell'excentro relativo ad $A$ sono
	$$
	I_A=[-a:b:c],
	$$
	le coordinate del punto di Lemoine sono 
	$$
	L=[a^2:b^2:c^2],
	$$
	quelle di Gergonne sono 
	$$
	Ge=\left[\frac{1}{p-a}:\frac{1}{p-b}:\frac{1}{p-c}\right],
	$$
	e di Nagel 
	$$
	Na=\left[p-a:p-b:p-c\right].
	$$
	
	\item \emph{Osservazioni.} Ci sono dei punti che \emph{non stanno sul nostro piano}. Infatti si mostra che $S$, l'area di $ABC$, è uguale a $|BCP|+|CAP|+|ABP|$. Dunque i punti $[x:y:z]$ tali che $x+y+z=0$ sono dei punti \emph{fantasma} sul nostro piano. Diciamo che sono \emph{sulla retta all'infinito}.
	
	Notare che per qualsiasi altra scelta di $[x:y:z]$ con $x+y+z\neq 0$ esiste uno e un solo punto sul piano che ha quelle come coordinate.
	
	Le coordinate $P=[\alpha:\beta:\gamma]$ tali che $\alpha+\beta+\gamma=1$ sono dette coordinate baricentriche esatte di $P$ e sono tali che
	$$
	\vec{P}=\alpha\vec{A}+\beta\vec{B}+\gamma\vec{C}.
	$$
	
	Questo segue molto velocemente dalla definizione delle coordinate che tira in ballo l'area. Dunque per trovare il punto medio fra due punti bisogna usare le coordinate baricentriche esatte - o perlomeno con la stessa somma delle coordinate!
	
	\emph{Esercizio:} trova le coordinate del punto di Feuerbach, ovvero il centro della circonferenza di Feuerbach, che è il punto medio fra $O$ e $H$.
	
	\item \textbf{Rette}: 
	\emph{Equazione.} Una generica retta ha equazione $lx+my+nz=0$ per qualche $l,m,n$ reali. \emph{Pensarci per esercizio}.
	
	\emph{Punto all'infinito.} Il punto all'infinito di questa retta è $[m-n:n-l:l-m]$. 
	
	\emph{Intersezione di due rette.} Due rette $lx+my+nz=0$ e $l'x+m'y+n'z=0$, una non multipla dell'altra, si intersecano sempre nell'unico punto di coordinate omogenee
	$$
	\left[n'm-nm':nl'-n'l:lm'-l'm\right].
	$$
	
	\emph{Rette parallele}. Dunque anche due rette parallele si intersecano sempre, visto il punto precedente. In effetti si intersecano nel punto all'infinito di entrambe.
	
	\emph{Retta per due punti}. Dati due punti $[a:b:c]$ e $[a':b':c']$, la retta passante per questi due punti ha equazione 
	$$
	\begin{bmatrix}
		a & b & c \\
		a' & b' & c' \\
		x & y & z
	\end{bmatrix}
	=0
	$$
	
	Da questa segue anche la condizione di allineamento di tre punti, e la scrittura di una retta parallela ad una data, passante per un punto.
	
	\emph{Rette perpendicolari}. \emph{Da fare come esercizio}
	Il punto all'infinito di una retta perpendicolare a $px+qy+rz=0$ è
	\begin{equation}
	[S_Bg-S_Ch:S_Ch-S_Af:S_Af-S_Bg]
	\end{equation}
	dove $[f:g:h]=[q-r:r-p:p-q]$ è il punto all'infinito della retta.
	
	\emph{Dagli esercizi:}
	\item \textbf{[Coordinate dei vertici del \textit{triangolo tangenziale} in baricentriche]}  Dato un triangolo $ABC$ referenziale in un sistema di coordinate baricentriche, mostrare che la tangente condotta da $A$ alla circonferenza circoscritta ad $ABC$ ha equazione
	\begin{equation}
	c^2y+b^2z=0.
	\end{equation}
	
	Ciclando opportunamente, calcolare le coordinate dei vertici del triangolo tangenziale (\textit{i.e.} il triangolo formato dalle intersezione delle tangenti condotte da $A$, $B$ e $C$ alla circonferenza circoscritta ad $ABC$).
	%\textbf{Soluzione:}
	%Calcoliamo le coordinate di $P$, intersezione della tangente condotta da $A$ alla circoscritta e $BC$. Risulta che 
	%$$
	%P=[0:-b^2:c^2]
	%$$
	%da cui si ottiene subito che la tangente, dovendo passare per $A=[1:0:0]$ è
	%$$
	%c^2y+b^2z=0.
	%$$
	
	%Ciclando si ottiene che i vertici del triangolo tangenziale sono $[a^2:b^2:-c^2]$ e ciclici.
	\emph{Dai problemi:}
	\item \textbf{[MOP 2006]} Sia $ABC$ un triangolo inscritto in una circonferenza $\omega$. $P$ giace su $BC$ in modo tale che $PA$ è tangente a $\omega$. La bisettrice di $\angle APB$ interseca i segmenti $AB$ e $AC$ rispettivamente in $D$ ed $E$ e i segmenti $BE$ e $CD$ si intersecano in $Q$. Supponiamo che la retta $PQ$ passi per il centro di $\omega$. 
	
	Calcolare $\angle BAC$.
	
	%\textbf{Soluzione:}
	%Dagli esercizi sappiamo che $P=[0:b^2:-c^2]$. Dal teorema della bisettrice si deduce che $D=[c:b:0]$ e $E=[b:0:c]$, da cui $Q=[bc:b^2:c^2]$. A questo punto usando $O=[a^2S_A:b^2S_B:c^2S_C]$ e imponendo il determinante uguale a 0 si deduce $\angle BAC=60$.
	
	 
	
\end{itemize}
\clearpage
\subsection{GM - 2, [Geometria proiettiva, poli e polari, quadrilateri armonici]}

\begin{short}
 Lunghezze con segno (velocemente). Birapporto tra 4 punti su una retta. Proiezione del birapporto, quindi birapporto tra 4 rette o 4 punti su circonferenza. Quaterna Armonica. Teorema di Desargues.\\
 Polo e Polare. Teorema di La Hire. Lemma della polare. Teorema di Pappo. Teorema di Pascal. Dualità polo-polare. 
\end{short}

Lunghezze con segno. Definizione di birapporto fra 4 punti $(A,B;C,D)$ su una retta e quando si dicono coniugati armonici. Conservazione del birapporto fra punti per proiezione da un punto esterno. Discussione del caso in cui, proiettando da un punto esterno su una retta, un punto va nel punto all'infinito: cosa diventa il birapporto nel caso in cui un punto sia all'infinito? Unicità del quarto armonico. 
Definizione di birapporto fra 4 rette e relazione con gli angoli che queste formano. Definizione di birapporto fra 4 punti su una circonferenza.

Teorema di Desargues, Teorema di Pappo e Teorema di Pascal.


\vspace{0.5cm}

Definizione proiettiva di polare: data una circonferenza (o due rette) e punto $P$, si traccino le rette che passano per P e secano lo circonferenza (o incontrano le rette) in due punti $A$ e $B$. Il luogo dei punti $X$ tali che $(A,B;P,X)=-1$ è una retta detta polare di $P$ rispetto alla circonferenza (o alle due rette).  Proprietà geometriche nel caso della circonferenza.

Dualità poli-polari. Lemma della polare: Dato un punto $P$ e una circonferenza (o due rette), se traccio due rette secanti che tagliano la circonferenza (o le rette) in due coppie di punti $A,B$ e $C,D$ dimodoché i punti siano nell'ordine $P,A,B$ e $P,C,D$, allora $AD\cap BC$ e $AC\cap BD$ sono sulla polare di $P$ rispetto alla circonferenza (o alle due rette).  Teorema di Brianchion.

Definizione di quadrilatero armonico con i birapporti. In un quadrilatero armonico una diagonale e la tangenti alla circonferenza in cui è inscritto condotte per gli altri due punti concorrono.


\vspace{0.3cm}
\large{\textbf{Versione estesa}}\normalsize

\vspace{0.3cm}
Il setting della geometria proiettiva è quello della retta euclidea a cui viene aggiunto un punto all'infinito, o del piano euclideo in cui viene aggiunta una retta all'infinito.

\vspace{0.3cm}
\textbf{Lunghezze con segno} Su una retta $r$ sono presenti alcuni punti $A,B,C\ldots$. Si scelga un verso sulla retta e si considerino i segmenti su di essa come vettori, con segno positivo se orientati nel verso scelto e negativo altrimenti. Il vantaggio di questo è che vale $\overline{AC}=\overline{AB}+\overline{BC}$ per qualsiasi posizione reciproca di $A,B,C$.

\vspace{0.3cm}
\textbf{Birapporto} Dati 4 punti $A,B,C,D$ su una retta, si definisce il birapporto è la seguente quantità:
$$(A,B;C,D)=\frac{\frac{AC}{AD}}{\frac{BC}{BD}}=\frac{AC\cdot BD}{BC\cdot AD}$$
dove le lunghezze sono prese con segno.

\vspace{0.3cm}
Se $(A,B;C,D)=k$, qual è il valore del birapporto se si permuta l'ordine in cui si prendono i punti? Le $4!=24$ possibilità si dividono in $6$ gruppi in ciascuno dei quali il birapporto è lo stesso. Se si scambiano le due coppie oppure si inverte l'ordine in entrambe il birapporto non cambia: $(A,B;C,D)=(C,D;A,B)=(B,A;D,C)$.

Se si scambiano i primi due o gli ultimi due, il birapporto diventa reciproco: $(A,B;D,C)=(B,A;C,D)=1/k$.

Se si scambia il secondo e il terzo $B \leftrightarrow C$, si ottiene $(A,C;B,D)=1-k$.

Se si scambia il primo e il terzo $A \leftrightarrow C$, si ottiene $(C,B;A,D)=\frac{k}{k-1}$.


Combinando queste trasformazioni, si possono ottenere i valori di $(A,C;D,B)=\frac{1}{1-k}$ e $(A,D;B,C)=\frac{k-1}{k}$.

Un'altra cosa interessante è fissare i punti $A,B,C$ e vedere come varia il birapporto $(A,B;C,D)$ al variare di $D$ sulla retta. Questa è una funzione biettiva dalla retta proeittiva in $\mathbb{R}\cup\infty$, nei casi degeneri in cui $D$ coincide con uno dei punti assume i valori degeneri di $0,1,\infty$; se $D=\infty$, il birapporto vale $AC/BC$.

\vspace{0.3cm}
\textbf{Quaterna Armonica} Quattro punti su una retta si dicono una quaterna armonica se $(A,B;C,D)=-1$. Per quanto detto sulle permutazioni, una quaterna è armonica se e solo se non è degenere e $(A,B;C,D)=(B,A;C,D)$.

Una quaterna armonica dev'essere "incatenata": fissati $A,B$, uno tra $C$ e $D$ deve stare all'interno del segmento $AB$ e uno all'esterno. Analogamente si avrà che uno tra $A$ e $B$ sta all'interno del segmento $CD$ e uno all'esterno.

\vspace{0.3cm}



\clearpage
\subsection{GM - 3 [Configurazione di Miquel, rotomotetia, qualcosa sulle mistilinee e inversioni sintetiche]}

\begin{short}
 Angoli orientati, Miquel su triangolo e su quadrilatero. Lemma della rotomotetia. Quadrilatero completo e rotomotetie presenti nella configurazione. Altre applicazioni di inversione. mistilinei,
\end{short}

Angoli orientati ed esercizi/complementi sui quadrilateri ciclici. 

Punto di Miquel riferito a una terna di punti presi sui lati di un triangolo. Punto di Miquel riferito a un quadrilatero. Facendo opportuno riferimento all'esercizio 2 della sezione \textbf{GM-1}, osservare che il punto di Miquel di un quadrilatero $ABCD$ è il centro della \emph{spilar similarity} che manda $AB$ in $DC$ o $AD$ in $BC$. Il quadrilatero $ABCD$ è ciclico se e solo se il punto di Miquel $M$ sta su $QR$, dove $Q=AB\cap CD$ e $R=AD\cap BC$.

Nel caso di ciclicità:
\begin{itemize}
	\item $OM$ è perpendicolare a $QR$, essendo $O$ il circocentro di $ABCD$;
	\item $A,C,M,O$ e $B,D,M,O$ sono conciclici;
	\item $AC$, $BD$ e $OM$ sono concorrenti in $P$;
	\item $MO$ biseca $\angle CMA$ e $\angle BMD$;
	\item $P$ e $M$ sono inversi rispetto alla circonferenza circoscritta al quadrilatero $ABCD$.
\end{itemize} 

Un'avventura mistilinea: considerati quattro punti in senso antiorario su una circonferenza $\Gamma$ ($A,B,C,D$) ed essendo $P=AC\cap BD$, considero $\omega$ tangente ai segmenti $AP$ e $BP$ e a $\Omega$ rispettivamente in $E$, $F$ e $T$. Provare le seguenti:
\begin{itemize}
	\item $TE$ biseca l'arco $AC$ che contiene $D$;
	\item Detto $I$ l'incentro di $ABC$, $IFTB$ è ciclico e $I\in EF$
	\item Detto $J$ l'incentro di $APB$ allora $TJFB$ è ciclico e $TJ$ biseca $\angle ATB$.
\end{itemize}

Ripasso delle proprietà base riguardanti l'inversione. $\sqrt{bc}$-inversione più simmetria: risoluzione di alcuni problemi.

\vspace{0.3cm}
\large{\textbf{Versione estesa - Senior 2019}}\normalsize

\vspace{0.3cm}

\begin{itemize}
	\item \textbf{Angoli Orientati}:
	\emph{Definizione}. L'angolo orientato $\angle(l,r)$ è l'angolo di cui si deve ruotare $l$ in senso antiorario perché coincida con $r$. L'angolo orientato $\angle ABC$ è, per definizione, l'angolo orientato $\angle(AB,BC)$.
	
	\emph{Proprietà}. 
	\begin{enumerate}
		\item $\angle(l,m)+\angle(m,l)=\pi$,
		\item $\angle ABC+\angle BCA+\angle CAB=\pi$,
		\item $\angle AOP + \angle POB=\angle AOB$,
		\item $A,B,C$ allineati se e solo se per un punto (o per tutti) $\angle XBC=\angle XBA$,
		\item $A,B,X,Y$ ciclico se esolo se $\angle AXB=\angle AYB$. 
	\end{enumerate}

	\item \textbf{Teorema di Miquel}:
	\emph{Versione triangolare}. Dato $ABC$ un triangolo e $D,E,F$ punti rispettivamente sulle rette dei lati $BC$, $CA$ e $AB$, le circonferenze $AEF$, $BDF$ e $CDE$ concorrono. 
	
	\emph{Versione quadrangolare}. Date $r_1,r_2,r_3,r_4$ rette che si intersecano in 6 punti (\emph{Cosa succede nei casi degeneri?}) siano $A_{ij}\doteq r_i\cap r_j$ i punti di intersezione. Le circonferenze circoscritte ai triangoli $A_{12}A_{23}A_{31}$, $A_{12}A_{24}A_{14}$, $A_{13}A_{34}A_{14}$ e $A_{23}A_{34}A_{24}$ concorrono.

\item \textbf{Rotomotetie}: Dati due punti $A,B,C,D$ distinti sul piano, ricordare che esiste una e una sola rotomotetia che porta $A$ in $B$ e $C$ in $D$ se e solo se $AC$ non è parallelo a $BD$. In tal caso, detto $X\doteq AC\cap BD$, il centro di tale rotomotetia è $W$ l'interesezione delle circonferenze circoscritte a $AXB$ e $CXD$. Discutere cosa succede se $X$ coincide con uno dei 4 punti $A,B,C,D$ (una delle due circonferenze da tracciare diviene tangente ad uno dei segmenti).

\emph{Dagli Esercizi:}
Discutere un caso degenere \item Sia $ABC$ un triangolo. Mostrare che il centro della (unica) rotomotetia che manda $B$ in $A$ e $A$ in $C$ è sulla simmediana uscente da $A$. 

\item \textbf{Inversione $\sqrt{bc}$ e simmetria}. Usare come pretesto la parte finale dell'esercizio precedente per introdurre questa tecnica. 

%\textbf{Soluzione:} Viste le considerazioni fatte nella parte sulla rotomotetia, tale centro è l'intersezione $X$ fra la circonferenza che passa per $A$ e $B$ e tange $AC$ in $A$ e la circonferenza che passa per $A$ e $C$ e tange $AB$ in $A$.
%Ora (anticipazione) faccio una inversione di centro in $A$ e raggio $\sqrt{AB\cdot AC}$ più una simmetria rispetto alla bisettrice. Si ha che $B\to C$ e $C\to B$. La circonferenza $ABX$ va in una retta passante per $C$ e parallela ad $AB$ e la circonferenza $ACX$ va in una retta passante per $B$ e parallela ad $C$. Dunque $X$ va in un punto sulla mediana e dunque prima era sulla simmediana. 
\emph{Dai problemi:}
\item \textbf{[USAMO 2006]} 
Sia $ABCD$ un quadrilatero e siano $E$ e $F$ punti su $AD$ e $BC$ rispettivamente tali che $AE/ED=BF/FC$.
La retta $FE$ incontra $BA$ e $CD$ in $S$ e $T$ rispettivamente. 

Mostrare che le circonferenze circoscritte ai triangoli $SAE$, $SBF$, $TCF$ e $TDE$ passano per uno stesso punto.

%\textbf{Soluzione:} Innanzitutto Per Miquel sui quadrangoli sappiamo che le circonferenze circoscritte ai triangoli $SAE$, $SBF$ e $ABX$ concorrono; così come le circonferenze circoscritte ai triangoli $TCF$, $TDE$ e $XCD$. Quindi, essendo la tesi vera, l'intersezione delle quattro circonferenze deve essere l'altra intersezione fra le circonferenze $XAB$ e $XCD$. Sia $Y$ questa intersezione. Per quanto visto sulle rotomotetie, questo punto è il centro della rotomotetia che manda $BC$ in $AD$ e dunque, visti i rapporti fra i segmenti, deve mandare $F$ in $E$. Allora 
%$$
%\angle YFB=\angle YEX
%$$
%e dunque $Y$ è sulla circonferenza circoscritta a $XEF$ e pertanto, per Miquel, anche su quella circoscritta a $SAE$ e $SBF$.

\textbf{Approfondimento sulla configurazione di Miquel:} Sia $ABCD$ un quadrilatero con $Q\doteq AB\cap CD$ e $R\doteq AD\cap BC$. Mostrare che 
\begin{enumerate}
	\item $M\in QR$ se e solo se $ABCD$ ciclico, 
	\item $ABCD$ ciclico implica $OM\perp QR$,
	\item $ABCD$ ciclico implica $MAOC$ e $BODM$ ciclici, 
	\item $ABCD$ ciclico implica $MO$ biseca $AMC$ e $BMD$,
	\item $ABCD$ ciclico implica $O$, $M$ e $AC\cap BD$ allineati. Da ciò, invertendo nella circonferenza circoscritta ad $ABCD$, si ottiene che $P$ e $M$ sono uno l'inverso dell'altro.
\end{enumerate}

\textbf{Fatti sulle circonferenze mistilinee:} \emph{Le ceviane delle mistilinee concorrono}.
Sia $ABC$ un triangolo inscritto in $\Omega$ e $\omega_A$ una circonferenza tangente internamente a $\Omega$ in $T_A$ e tangente anche ad $AB$ e $AC$ in $B_1$ e $C_1$ rispettivamente. Da un'inversione $\sqrt{bc}$ più simmetria rispetto alla bisettrice mostrare che $AT_A$ e cicliche concorrono - nel coniugato isogonale del punto di Nagel. Osservare che $AT_A$ contiene il centro di similitudine esterno fra $\omega$, la circonferenza inscritta ad $ABC$, e $\Omega$ e dunque il coniugato isogonale del punto di Nagel è il centro di similitudine esterno fra $\omega$ e $\Omega$. 

\emph{Altri fatti}. Con riferimento alla precedente,  mostrare che
\begin{enumerate}
	\item L'incentro di $ABC$, $I$, è su $B_1C_1$. Ciò segue da Pascal su $M_{AB}T_AM_{AC}BAC$, dove $M_{AB}$ e $M_{AC}$ sono i punti medi degli archi $AB$ e $AC$ che non contengono $C$ e $B$; più il lemma che $T_A,B_1,M_{AB}$ sono allineati,  
	\item L'incentro $I$, dopo l'inversione, va nell'excentro $I_A$. Ciò segue dal fatto che $AI\cdot AI_A=AB\cdot AC$ il quale a sua volta segue dal fatto che $BCII_A$ è ciclico più un semplice conto di angoli. Da ciò segue che $BT_AB_1I$ e $CT_AC_1I$ sono ciclici,
	\item Dal punto precedente segue $T_AI$ biseca $\angle BT_AC$, 
	\item $T_AM_A$, $BC$ e $B_1C_1$ concorrono applicando Pascal su $BCM_{AB}T_AM_AA$, 
	\item Da due punti sopra viene $M_AT_A$ perpendicolare a $T_AI$ e dunque $T_AI$ interseca $\Omega$ nel diametralmente opposto di $M_A$.
\end{enumerate}
	\emph{Dai problemi:}
	\item \textbf{[EGMO 2013 - 5]}
	Sia $\Omega$ la circonferenza circoscritta ad un triangolo $ABC$. La circonferenza $\omega$ è tangente ai lati $AC$ e $BC$ e internamente alla circonferenza $\Omega$ in un punto $P$. Una retta parallela ad $AB$ che interseca l'interno del triangolo $ABC$ è tangente a $\omega$ in $Q$.
	
	Mostrare che $\angle ACP = \angle QCB$.
	
	%\textbf{Soluzione:} Per inversione più simmetria $AP$ è la simmetrica della ceviana di Nagel rispetto alla bisettrice, che , per omotetia, coincide con $AQ$.
\item \textbf{Qualche problema sull'inversione:} 
\emph{Dagli esercizi:}

\textbf{[Teorema di Feuerbach]} Mostrare che la circonferenza di Feuerbach è tangente alla circonferenza inscritta e alle circonferenze exinscritte.

%\emph{Suggerimento:} Sia $M$ il punto medio di $BC$ e $D$ e $G$ rispettivamente i punti in cui la circonferenza inscritta e quella ex-inscritta opposta ad $A$ incontrano $BC$. Invertire in $M$ con raggio $MD$.

\emph{Dai problemi:}

\textbf{[IMO 2015 - 3]} Sia $ABC$ un triangolo acutangolo con $AB > AC$. Sia $\Gamma$ la sua circonferenza circoscritta, $H$ il suo ortocentro, e $F$ il piede dell'altezza condotta da $A$. Sia $M$ il punto medo di $BC$. Sia $Q$ il punto su $\Gamma$ tale che $\angle HQA = 90^{\circ}$ e sia $K$ il punto su $\Gamma$ tale che $\angle HKQ = 90^{\circ}$. Assumiamo che $A$, $B$, $C$, $K$ e $Q$ sono tutti distinti e giacciono su $\Gamma$ in quest'ordine. 

Mostrare che le circonferenze circoscritte ai triangoli $KQH$ e $FKM$ sono fra loro tangenti.
%\begin{sol}Inversione di centro H che fissa la circonferenza circoscritta ad ABC. K'Q' diviene perpendicolare ad AK' che è l'asse di F'M' e dunque K'Q' è la tangente a K' nella circonferenza circoscritta a F'M'K'.
%\end{sol}






	
\end{itemize}
\clearpage

 








