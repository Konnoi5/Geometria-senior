\section{Problemi Basic}
\subsection{GB - 1, Problemi}
\begin{itemize}
 \item \textbf{EGMO 2013 - 1} Nel triangolo $ABC$, si prolunghi il lato $BC$ dalla parte di $C$ di un segmento $CD$ tale che $CD=BC$. Si prolunghi poi il lato $CA$ dalla parte di $A$ di un segmento $AE$ tale che $AE= 2CA$.Dimostrare che, se $AD=BE$, allora il triangolo $ABC$ è rettangolo
 
 \item \textbf{IMOSL 1998 - 5} Sia $ABC$ un triangolo, $H$ l'ortocentro, $O$ il circocentro e $R$ il raggio della circonferenza circoscritta. Sia $D$ il simmetrico di $A$ rispetto a $BC$, $E$ il simmetrico di $B$ rispetto $AC$ e $F$ il simmetrico di $C$ rispetto $AB$.\\
 Dimostrare che $D,E,F$ sono allineati se e solo se $OH=2R$.

 \begin{sol}
 Complessi con circoscritta = circonferenza unitaria
\end{sol}

 
 
 \item \textbf{IMOSL 2015 - G1} Sia $ABC$ un triangolo acutangolo con ortocentro $H$. Sia $G$ il punto per cui il quadrilatero $ABGH$ risulta un parallelogrammo. Sia $I$ il punto della retta $GH$ per cui
la retta $AC$ biseca il segmento $HI$. Sia $J$ l’ulteriore intersezione tra la retta $AC$ e la
circonferenza circoscritta al triangolo $GCI$.
Dimostrare che $IJ = AH$.

\begin{sol}
 Sia $M=GH\cap AC$, Teorema dei seni su $\triangle IJM$ da $\frac{\sin\alpha}{IJ}=\frac{\sin IMJ}{IJ}=\frac{\sin IJM}{MH}=\frac{\sin IGC}{MH}$ per la ciclicità di $GCIJ$. Teorema dei seni su $\triangle MAH$ da $\frac{\sin\alpha}{AH}=\frac{\sin CJH}{MH}$. Per la tesi basta dimostrare che $\widehat{CGH}=\widehat{CAH}=90-\gamma$, ma $CHG$ è rettangolo e $CH=c\cdot cotg(\gamma)=HG\cdot cotg(\gamma)$.
\end{sol}

\item \textbf{ITA TST 2016 - 1} Sia $ABCD$ un quadrilatero. Supponiamo che esista un punto $P$ interno al quadrilatero tale che $\angle APD = \angle BPC = 90^{\circ}$ e $PA \cdot PD = PB \cdot PC$. Sia $O$ il circocentro di $\triangle CPD$.\\
Dimostrare che la retta $OP$ passa per il punto medio di $AB$.
 
 \end{itemize}

\subsection{GB - 2, Problemi}
 \begin{itemize}
    \item \textbf{IMOSL2013 - G2}  Sia $ABC$ un triangolo, e sia $\omega$ la sua circonferenza circoscritta.  Siano $M$ il punto medio di $AB$, $N$ il punto medio di $AC$, $T$ il punto medio dell’arco $BC$ di $\omega$ che noncontiene $A$. La circonferenza circoscritta al triangolo $AMT$ interseca l’asse di $AC$ in un punto $X$ interno al triangolo $ABC$. La circonferenza circoscritta al triangolo $ANT$ interseca l’asse di $AB$ in un punto $Y$ interno al triangolo $ABC$. Le rette $MN$ e $XY$ si intersecano in $K$.\\
    Dimostrare che $KA=KT$.
    
    \begin{sol}La simmetria rispetto all'asse di $AT$ manda $M$ in $X$ e $N$ in $Y$, quindi $K$ rimane fisso e sta sull'asse.
    \end{sol}
    
    \item \textbf{EGMO 2016 - 4} Due circonferenze aventi lo stesso raggio, $\omega_1$ e $\omega_2$ , si intersecano in due punti distinti $X_1$ and $X_2$ . Si consideri una circonferenza $\omega$ tangente esternamente a $\omega_1$ nel punto $T_1$ e internamente a $\omega_2$ nel punto $T_2$.\\ Si dimostri che il punto d’intersezione fra le rette $X_1T_1$ e $X_2T_2$ giace su $\omega$.
    \begin{sol}
    Inversione in $X_1$
    \end{sol}
    
    \item \textbf{Allenamenti EGMO 2019 - G6}
    Dato il triangolo $\Delta ABC$ consideriamo $\omega_B$ la circonferenza passante per A, B e tangente in A al lato AC e, simmetricamente, $\omega_C$ la circonferenza passante per A, C e tangente in A al lato AB. Sia D il punto di intersezione di $\omega_B$ e $\omega_C$ , e sia E il punto sulla retta AD tale che $AD = DE$.\\
    Dimostrare che E sta sulla circonferenza circoscritta al triangolo $\triangle ABC$.

    \begin{sol}
    invertire in A.
    \end{sol}

    \item \textbf{Senior 2013 TF}
    Sia $ABC$ un triangolo. Sia D l’ulteriore intersezione tra la circonferenza passante per C e tangente alla retta AB in A e la circonferenza passante per B e tangente alla retta AC in A.
    Sia E il punto sulla retta AB (diverso da A) tale che $BA = BE$. Sia F l’ulteriore intersezione tra la retta AC e la circonferenza circoscritta al triangolo $ADE$.\\
    Dimostrare che $AC = AF$.

    \begin{sol}
    Invertire in A.
    \end{sol}

    
    
	\item \textbf{IMOSL2011 - G4} Sia $ABC$ un triangolo acutangolo e $\Gamma$ la sua circonferenza circoscritta. Sia $B_0$ il punto medio di $AC$ e $C_0$ il punto medio di $AB$. Sia $D$ il piede dell'altezza da $A$ su $BC$ e sia $G$ il baricentro di $ABC$. Sia $\omega$ la circonferenza passante per $B_0,C_0$ e tangente a $\Gamma$ in un punto $X\neq A$. 
	
	Dimostrare che $D,X,G$ sono allineati.
	
	\begin{sol}
	\emph{nota: la soluzione proposta è basic difficile/medium facile}
	
	Inversione + simmetria di centro A e raggio $\sqrt(AB*AB_0)$, scambia B e $B_0$, C e $C_0$, manda $D$ nel centro di $\Gamma$ O e $\omega$ in una circonferenza per $B$ e $C$ tangente all'immagine di $\Gamma$, $B_0C_0$, in un punto $Y$. Poiché $BC$ e $B_0C_0$ sono paralleli, $Y$ sta sull'asse di $BC$, quindi $OY$ è perpendicolare a $B_0C_0$. 
	
	Sia $T$ l'intersezione delle tangenti a $\Gamma$ per A,X e di $B_0C_0$, è centro radicale di $\Gamma, \omega$ e la circoscritta a $AB_0C_0$. ATXYO è ciclico, l'immagine sotto inversione è la retta XDY. Ora basta mostrare DY intersecato la mediana $AA_0$ è G, ma $AD$ è il doppio di $XA_0$ e sono paralleli, quindi l'intersezione è proprio G.
	\end{sol}
	


	\end{itemize}
	
	
\subsection{GB - 3, Problemi}
\begin{enumerate}
    \item \textbf{Polish MO 2018 - 5} Sia $ABC$ un triangolo acutangolo con $AB\neq AC$
    e siano $E,F$ i piedi delle altezze su $AC$ e $AB$. La tangente in $A$ alla circoscritta interseca $BC$ in $P$. La retta parallela a $BC$ passante per $A$ interseca $EF$ in $Q$. 
    
    Dimostrare che $PQ$ è perpendicolare alla mediana passante per $A$ del triangolo $ABC$
    
    \begin{sol}Assi radicali swag: 1) La circonferenza degenere di centro $A$, la circoscritta a $AEF$ e a $BCEF$ hanno $Q$ come centro radicale (in quanto sta su $EF$ per le ultime due e $AQ$ tange la circoscritta $AEF$ per le prime due). 2) $PA^2=PB\cdot PC$, quindi P sta sull'asse radicale tra $A$ e la circoscritta a $BCEF$. Dunque $PQ$ è asse radicale delle due circonferenze ed è perpendicolare alla congiungente dei centri, che è $AM$
    \end{sol}

\end{enumerate}


\clearpage

\section{Problemi Medium}
\subsection{GM - 1, Problemi}
\begin{enumerate}
	\item \textbf{[BMO 2009 - 2]} Sia $MN$ una segmento parallelo al lato $BC$ del triangolo $ABC$, con $M$ sul lato $AB$ e $N$ sul lato $AC$. Le rette $BN$ e $CM$ si incontrano in $P$. Le circonferenze circoscritte a $BMP$ e $CNP$ si incontrano in due punti distinti $P$ e $Q$. 
	
	Mostrare che $\angle BAQ = \angle PAC$.
	
	\textbf{Soluzione:} Diciamo che $a$ è l'origine del nostro piano di Gauss, mentre $b$ e $c$ sono due generici punti. Visto che $mn\parallel bc$ e $m\in ab$, $m\in ac$ si ha che esiste $\lambda \in \mathbb R$ tale che $m=\lambda b$ e $n=\lambda c$. Essendo $q$ il centro della rotomotetia che manda $m$ in $b$ e $c$ in $n$, allora 
	$$
	q=\frac{mn-bc}{m+n-b-c}=\frac{\lambda^2bc-bc}{\lambda b+\lambda c-b-c}=\frac{(\lambda+1)bc}{b+c}.
	$$
	
	Per trovare trovare $p$ basterebbe imporre $p\in mc$ e $p\in bn$. \emph{Proporlo come esercizio}. D'altra parte non ce n'è bisogno: infatti noi siamo interessati poi solo all'angolo $\angle CAP$ e dunque non tanto ci servono le coordinate di $P$ quanto capire chi è la retta $AP$, che è la mediana di $ABC$. Dunque possiamo dire che esiste un certo $\eta$ reale tale che 
	$$
	p=\eta(b+c).
	$$ 
	
	Per l'equazione dell'angolo, se $\theta=\angle BAQ$ si ha 
	$$
	e^{2i\theta}=\frac{q-a}{b-a}\frac{\bar b-\bar a}{\bar q-\bar a}=\frac{c(\bar b+\bar c)}{\bar c(b+c)},
	$$
	mentre se $\theta'=\angle PAC$ si ha 
	$$
	e^{2i\theta'}=\frac{c-a}{p-a}\frac{\bar p-\bar a}{\bar c-\bar a}=\frac{c(\bar b+\bar c)}{\bar c(b+c)}.
	$$
	
	Da ciò, con un attimo di discussione, si ottiene che $\theta=\theta'$ che implica la tesi. 
	
	\item \textbf{[RMM 2012 - 2]} Sia $ABC$ un triangolo non isoscele e siano $D$, $E$ e $F$ rispettivamente i punti medi dei lati $BC$, $CA$ e $AB$. La circonferenza $BCF$ e la retta $BE$ si intersecano nuovamente in $P$ e la circonferenza $ABE$ e la retta $AD$ in $Q$. Le rette $DP$ e $FQ$ si incontrano in $R$. 
	
	Mostrare che il baricentro $G$ del triangolo $ABC$ giace sulla circonferenza circoscritta al triangolo $PQR$.
	
	\textbf{Soluzione:} Per mostrare la ciclità è sufficiente mostrare che, detto $\theta=\angle GPD$ e $\theta'=\angle GQF$, si ha 
	$$
	\theta=\theta'.
	$$  
	Dall'equazione dell'angolo risulta che per fare ciò è sufficiente mostrare
	$$
	\frac{d-p}{g-p}\frac{g-q}{f-q}\in\mathbb R.
	$$
	Il problema è dunque spostato a trovare i punti $p$ e $q$. Qui usiamo un'osservazione sintetica. Si ha che 
	$$
	\angle GDE=\angle GAB = \angle QEG,
	$$
	dove la prima è vera per il parallelismo $AB\parallel ED$ e la seconda è vera poiché $ABEQ$ è ciclico. 
	Analogamente si ha $\angle EQD=\angle GED$ e dunque i triangoli $GDE$ e $GEQ$ sono ordinatamente simili.
	Dunque, scegliendo $g=0$, risulta, visto che $GD\cdot GQ=GE^2$,
	$$
	q=d\frac{|e|^2}{|d|^2}=\frac{e\bar e}{\bar d}
	$$
	e analogamente 
	$$
	p=\frac{f\bar f}{\bar e}.
	$$
	A questo punto 
	$$
	\frac{d-p}{g-p}\frac{g-q}{f-q}=\frac{(d\bar e-f\bar f)e\bar e}{(f\bar d-e\bar e)f\bar f}
	$$
	e poiché, essendo $g=0$, si ha $d+e+f=0$, la precedente espressione è uguale a 
	$$
	\frac{|e|^2}{|f|^2}
	$$ 
	che è un numero reale, come si voleva.
%	Dall'equazione dell'angolo si ha 
%	$$
%	e^{2i\theta}=\frac{d-p}{g-p}\frac{\bar g-\bar p}{\bar d-\bar p}%=\frac{f\bar f-d\bar e}{f\bar f-\bar d e}
%	$$
%	e 
%	$$
%	e^{2i\theta'}=\frac{f-q}{g-q}\frac{\bar g-\bar q}{\bar f-\bar q}%=\frac{e\bar e-f\bar d}{e\bar e-\bar f d}.
%	$$
%	Per mostrare che $e^{2i\theta}=e^{2i\theta'}$, e dunque $\theta=\theta'$ e dunque la tesi, è sufficiente mostrare che 
	
	che segue poiché $d+e+f=0$, essendo $g=0$, e sostituendo. 
	
	
	\item \textbf{[USAMO 2016 - Day 2 - 2]} Un pentagono equilatero $AMNPQ$ è inscritto in un triangolo $ABC$ in modo che $M\in AB$, $Q\in AC$ e $N,p \in BC$. Sia $S$ l'intersezione di $MN$ e $PQ$ e denotiamo con $l$ la bisettrice di $\angle MSQ$. 
	
	Mostrare che, detto $I$ l'incentro di $ABC$, $OI$ è parallelo a $l$.
	\item \textbf{[IMO 2008 - 6]} Sia $ABCD$ un quadrilatero convesso con $BA \neq BC$. Siano $\omega_1$ e $\omega_2$ le circonferenze inscritte ai triangoli $ABC$ e $ADC$ rispettivamente. Supponiamo che esista una circonferenza $\omega$ tangente alla retta $BA$ oltre A, alla retta $BC$ oltre $C$, alla retta $AD$ e alla retta $CD$.
	
	Mostrare che le tangenti esterne comuni a $\omega_1$ e $\omega_2$ si intersecano su $\omega$.
	
	\item \textbf{[BMO 2015 - 2]} Sia $ABC$ un triangolo scaleno con incentro $I$ e circonferenza circoscritta $\omega$. $AI$, $BI$ e $CI$ intersecano $\omega$ di nuovo nei punti $D$, $E$ e $F$ rispettivamente. Le rette parallele a $BC$, $CA$ e $AB$ condotte da $I$ intersecano $EF$, $DF$ e $DE$ rispettivamente nei punti $K$, $L$ e $M$.
	
	Mostrare che $K$, $L$ e $M$ sono allineati.
	
	\item \textbf{[IMO 2012 - 1]} Dato un triangolo $ABC$, sia $J$ il centro della circonferenza ex-inscritta opposta al vertice $A$, la quale tange $BC$ in $M$ e le rette $AB$ e $AC$ in $K$ e $L$ rispettivamente. Le rette $LM$ e $BJ$ si intersecano in $F$ e le rette $KM$ e $CJ$ si intersecano in $G$. Sia $S$ il punto d'intersezione fra $AF$ e $BC$ e sia $T$ il punto d'intersezione fra $AG$ e $BC$. 
	
	Mostrare che $M$ è il punto medio di $ST$.
	
	\textbf{Soluzione:}
	
	\item \textbf{[IMO SL 2011 - 4]} Sia $ABC$ un triangolo acutangolo scaleno, e sia $\gamma$ la sua circonferenza circoscritta.
	Siano $A_0$ il punto medio di BC, $B_0$ il punto medio di $AC$ e $C_0$ il punto medio di $AB$. Sia
	$D$ il piede dell’altezza uscente da $A$, $D_0$ la proiezione di $A_0$ sulla retta $B_0C_0$ e $G$ il
	baricentro di $ABC$. Sia $\gamma_1$ la circonferenza passante per $B_0$ e $C_0$, e tangente a $\gamma$ in un
	punto $P$ diverso da $A$.
	\begin{itemize}
	\item Dimostrare che la retta $B_0C_0$ e le tangenti a $\gamma$ nei punti $A$ e $P$ sono concorrenti.
	\item Dimostrare che i punti $D_0$, $G$, $D$, e $P$ sono allineati.
	\end{itemize}
	\item \textbf{[USA TST 2012 - December Test - 1]} In un triangolo acutangolo $ABC$ si ha $\angle A<\angle B$ e $\angle A<\angle C$. Sia $P$ un punto variabile su $BC$. I punti $D$ e $E$ giacciono su $AB$ e $AC$ rispettivamente in modo che $BP=PD$ e $CP=PE$.
	
	Mostrare che al variare di $P$ sul segmento $BC$, la circonferenza circoscritta al triangolo $ADE$ passa per un punto fisso oltre $A$.
	\item  \textbf{[IMO 2019 - 6]} 
	Sia $I$ l'incentro di un triangolo acutangolo $ABC$ con $AB\neq AC$. La circonferenza inscritta $\omega$ di $ABC$ è tangente a $BC$, $CA$ e $AB$ in $D$, $E$ e $F$ rispettivamente. La retta per $D$ e perpendicolare ad $EF$ interseca $\omega$ di nuovo in $R$ e la retta $AR$ interseca $\omega$ di nuovo in $P$. Sia $Q$ la seconda intersezione, diversa da $P$, delle circonferenze circoscritte ai triangoli $PBF$ e $PCE$.
	
	Mostrare che le rette $DI$ e $PQ$ si incontrano su una retta per $A$ perpendicolare ad $AI$.
	
	\textbf{Soluzione:} Usare come circonferenza unitaria la circonferenza inscritta. Consulta \url{https://artofproblemsolving.com/community/c6h1876745p12752769}.
\end{enumerate}
\clearpage
\subsection{GM - 2, Problemi}
\begin{enumerate}
	\item \textbf{[China NMO 2017 - 2]} Siano $\omega$ e $\Omega$ di centro $I$ e $O$ rispettivamente la circonferenza inscritta e circoscritta a un triangolo acutangolo
	$ABC$. La circonferenza $\omega$ interseca $BC$ in $D$ e le tangenti a $\Omega$ passanti per $B$ e $C$ si intersecano in $L$.
	Siano $AH$ l'altezza condotta da $A$ a $BC$ e $X$ l'intersezione di $AO$ con $BC$. Siano $P$ e $Q$ le 
	intersezioni di $OI$ con $\Omega$.
	
	Mostrare che $PQXH$ è ciclico se e solo se $A,D$ e $L$ sono allineati.
	\item \textbf{[IMO 2014 - 4]} Siano $P$ e $Q$ punti su un segmento $BC$ di un triangolo acutangolo $ABC$ tali che $\angle PAB = \angle BCA$ e $\angle CAQ=\angle ABC$. Siano $M$ e $N$ punti su $AP$ e $AQ$ rispettivamente tali che $P$ è punto medio di $AM$ e $Q$ è punto medio di $AN$.
	
	Mostrare che l'intersezione di $BM$ e $CN$ giace sulla circonferenza circoscritta di $ABC$.
	
	\item \textbf{[Iran TST 2007 - Day 2 - 3]}
	Sia $\omega$ la circonferenza inscritta ad un triangolo $ABC$ che tange $AB$ e $AC$ rispettivamente in $F$ e $E$. Siano $P$ e $Q$ su $AB$ e $AC$ rispettivamente in modo che $PQ$ sia parallelo a $BC$ e tangente ad $\omega$. Siano $T$ l'intersezione di $EF$ con $BC$ e $M$ il punto medio di $PQ$. 
	
	Mostrare che $TM$ tange $\omega$.
	
	\begin{sol}Se $X=AD\cap \omega$, $TX$ tange $\omega$ per quadrilateri armonici. Poi (XDAY)=-1 e proiettando da $T$ su $PQ$ ottengo che l'intersezione di $TX$
	 con $PQ$ è il suo punto medio
	\end{sol}
	
	\item \textbf{[Iran TST 2009 - Day 2 - 3]}
	In un triangolo $ABC$ è inscritta una circonferenza $\omega$ di centro $I$ che interseca i lati $BC$, $CA$ e $AB$ rispettivamente in $D$, $E$ e $F$. Sia $M$ il piede della perpendicolare da $D$ a $EF$. Sia $P$ il punto medio di $DM$ e $H$ l'ortocentro del triangolo $BIC$.
	
	Mostrare che $PH$ biseca $EF$. 
	\item \textbf{[Romania TST 2007 - Day 7 - 2]}	La circonferenza inscritta al triangolo $ABC$ è tangente 
	ad $AB$ e $AC$ in $F$ ed $E$ rispettivamente. Sia $M$ il punto di $BC$ e $N$ l'intersezione di $AM$ con $EF$. La circonferenza di diametro $BC$ interseca $BI$ e $CI$ in $X$ e $Y$ rispettivamente.
	
	Mostrare che $\displaystyle\frac{NX}{NY}=\displaystyle\frac{AC}{AB}$.
	
	\begin{sol}Usa l'esercizio 13 e nota che DXY è simile ad ABC e ID è bisettrice di YDX. Oppure semplicemente formula seni-lati su IXY e un po' di trigonometria
	\end{sol}
	
	\item \textbf{[IMO SL 2007 - G8]}
	Sul lato $AB$ di un quadrilatero convesso $ABCD$ è preso un punto $P$. Sia $\omega$ la circonferenza inscritta al triangolo $CPD$ e sia $I$ il suo centro. Supponiamo che $\omega$ sia tangente alle circonferenze inscritte ai triangoli $APD$ e $BPC$ in $K$ e $L$ rispettivamente. Siano $E$ l'intersezione delle rette $AC$ e $BD$ e $F$ l'intersezione delle rette $AK$ e $BL$.
	
	Mostrare che $E$, $I$ e $F$ sono allineati.
	

\end{enumerate}
\clearpage
\subsection{GM - 3, Problemi}
\begin{enumerate}
	\item \textbf{[USA TST 2007 - 5]} Il triangolo $ABC$ è inscritto in una circonferenza $\Gamma$. Le tangenti a $\Gamma$ condotte da $B$ e $C$ si intersecano in $T$. Il punto $S$ è sulla retta $BC$ dimodoché $AS\perp AT$. Siano $B_1$ e $C_1$ sulla retta $ST$ dimodoché $B_1T=BT=C_1T$.
	
	Mostrare che $ABC$ e $AB_1C_1$ sono simili.
	\item \textbf{[IMO 2005 - 5]}
	Sia $ABCD$ un quadrialtero convesso con $BC=DA$ e $BC$ non parallelo a $DA$. Siano $E$ e $F$ su $BC$ e $DA$ rispettivamente tali che $BE=DF$. Siano $P$ l'intersezione di $AC$ e $BD$, $Q$ l'intersezione di $BD$ e $EF$ e $R$ l'intersezione di $EF$ e $AC$.

	Mostra che, al variare di $E$ e $F$, la circonferenza circoscritta al triangolo $PQR$ passa per un punto fisso (oltre $P$). 
	
	\item \textbf{[?]}  Sia $ABC$ un triangolo e siano $D$ e $E$ i piedi delle altezze relative ad $A$ e $B$, rispettivamente,  le quali siintersecano  in $H$.   Sia $M$ il  punto  medio  di $AB$ e  supponiamo  che  le  circonferenze circoscritte a $ABH$ e $DEM$ si intersechino nei punti $P$ e $Q$ (con $P$ e $A$ sullo stesso lato di $CH$).
	%PreIMO Mattino 4 2016
	
	Mostrare che le rette $PH$ e $MQ$ si incontrano sulla circonferenza circoscritta ad $ABC$. 
	\item \textbf{[IMO SL 2006 - 9]}
	Sui lati $BC$, $CA$ e $AB$ di un triangolo $ABC$ si scelgano tre punti $A_1$, $B_1$ e $C_1$ rispettivamente. Le circonferenze circoscritte a $AB_1C_1$, $BC_1A_1$ e $CA_1B_1$ intersecano la circonferenza circoscritta ad $ABC$ in $A_2$, $B_2$ e $C_3$ rispettivamente. Siano, inoltre, $A_3$, $B_3$ e $C_3$ rispettivamente i simmetrici di $A_1$, $B_1$ e $C_1$ rispetto ai punti medi dei lati del triangolo su cui giacciono. 
	
	Mostrare che i triangoli $A_2B_2C_2$ e $A_3B_3C_3$ sono simili.
	
	\begin{sol}$A_2$ è il centro della spilar similiarity che porta $BC_1$ in $CB_1$ quindi $A_2C/A_2B=B_1C/C_1B=AB_3/AC_3$ da cui $A_2BC$ è
	simile ad $AC_3B_3$ e da qui sono angoli 
	\end{sol}
	\item \textbf{[EGMO 2013 - 5]}
	Sia $\Omega$ la circonferenza circoscritta ad un triangolo $ABC$. La circonferenza $\omega$ è tangente ai lati $AC$ e $BC$ e internamente alla circonferenza $\Omega$ in un punto $P$. Una retta parallela ad $AB$ che interseca l'interno del triangolo $ABC$ è tangente a $\omega$ in $Q$.
	
	Mostrare che $\angle ACP = \angle QCB$.
	\item \textbf{[IMO SL 2003 - 4]}
	 Siano  $\Gamma_1$, $\Gamma_2$, $\Gamma_3$, $\Gamma_4$ 
	 circonferenze distinte tali che
	 $\Gamma_1$ e $\Gamma_3$ (così come $\Gamma_2$ e $\Gamma_4$) siano tangenti esternamente in $P$. Supponiamo che $\Gamma_1$ e $\Gamma_2$, $\Gamma_2$ e $\Gamma_3$, $\Gamma_3$ e $\Gamma_4$, $\Gamma_4$ e $\Gamma_1$ si intersechino in $A$, $B$, $C$ e $D$ rispettivamente e che nessuno di questi punti sia $P$.
	 
	 Mostrare che 
	 $$
	 \frac{AB\cdot BC}{AD\cdot DC}=\frac{PB^2}{PD^2} .
	 $$
	 
	 \item \textbf{[IMO 2015 - 3]} Sia $ABC$ un triangolo acutangolo con $AB > AC$. Sia $\Gamma$ la sua circonferenza circoscritta, $H$ il suo ortocentro, e $F$ il piede dell'altezza condotta da $A$. Sia $M$ il punto medo di $BC$. Sia $Q$ il punto su $\Gamma$ tale che $\angle HQA = 90^{\circ}$ e sia $K$ il punto su $\Gamma$ tale che $\angle HKQ = 90^{\circ}$. Assumiamo che $A$, $B$, $C$, $K$ e $Q$ sono tutti distinti e giacciono su $\Gamma$ in quest'ordine. 
	 
	 Mostrare che le circonferenze circoscritte ai triangoli $KQH$ e $FKM$ sono fra loro tangenti.
	 \begin{sol}Inversione di centro H che fissa la circonferenza circoscritta ad ABC. K'Q' diviene perpendicolare ad AK' che è l'asse di F'M' e dunque K'Q' è la tangente a K' nella circonferenza circoscritta a F'M'K'.
	 \end{sol}

		
\end{enumerate}
