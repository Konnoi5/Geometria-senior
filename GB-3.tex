\section{GB - 3 [Sintetica]}
\subsection{Programmi}
\begin{short}
 Circonferenza di Apollonio. Circonferenza di Feuerbach. Simmediana. Segmenti di tangenza di Incerchi/Excerchi, punti di Gergonne e Nagel. Retta di simson. Applicazioni di potenze e assi radicali. 
\end{short}

\begin{itemize}

\item \textbf{Coniugato isogonale.} Prendiamo un punto $P$ in $\triangle ABC$. Sia $A’$ su $BC$ tale che $AA’$ sia la simmetrica di $AP$ wrt bisettrice di $\angle BAC$. Definiamo $B’,C’$ analogamente. Allora $AA’,BB’,CC’$ concorrono in $P’$, il coniugato isogonale di $P$.\\
\textit{Fatto:} $H$ e $O$ sono coniugati isogonali.\\
%Esercizio (per casa?): Si prendano $P,Q$ in $\triangle ABC$ tali che $P,Q$ sono coniugati isogonali. Allora il triangolo pedale di $P$ e il triangolo pedale di $Q$ giacciono su una circonferenza che ha come centro il punto medio di $PQ$.

\item \textbf{Coniugato isotomico.} Prendiamo un punto $P$ in $\triangle ABC$ e tracciamo le ceviane che concorrono in $P$. Siano $A_1,B_1,C_1$ le intersezioni di queste ceviane con i lati $BC,AC,AB$ rispettivamente. Si prenda $A_2$ il simmetrico di $A_1$ rispetto al punto medio di $BC$. $B_2,C_2$ sono definiti analogamente. Allora $AA_2,BB_2,CC_2$ concorrono in $P'$, il coniugato isotomico di $P$.\\

\item \textbf{Punto di Gergonne.} Si prendano $D,E,F$ i punti di tangenza dell'inscritta ad $\triangle ABC$ con i lati $AB,BC,CA$. Le rette $AD,BE,CF$ concorrono nel punto di Gergonne. 

\item \textbf{Punto di Nagel.} Considieriamo un triangolo $\triangle ABC$, con $D$ punto di tangenza dell'$A-$excerchio con $BC$, $E$ punto di tengenza del $B-$excerchio con $AC$ e $F$ punto di tangenza del $C-$excerchio con $AB$. Le rette $AD,BE,CF$ concorrono nel punto di Nagel. In un triangolo, il punto di Nagel è il coniugato isotomico del punto di Gergonne. \\
\textit{Fatto:} In un triangolo $\triangle ABC$ l'incentro è il punto di Nagel del triangolo mediale di $\triangle ABC$.\\

Segmenti di tangenza dell'incerhio e dell'excerchio.

\item \textbf{Retta di Simson.} Prendiamo un punto $P$ sulla circoscritta al triangolo $\triangle ABC$. Le proiezioni di $P$ sui lati di $\triangle ABC$ stanno su una stessa retta. \\
\textit{Fatto:} Se $P,Q$ giacciono sulla circoscritta ad $\triangle ABC$, l'angolo tra le le rette di Simson di $P$ e $Q$ è $\frac 12\angle POQ$. Se $P,Q$ sono diametralmente opposti, l'intersezione delle rette di Simson sta sulla circonferenza di Feuerbach di $\triangle ABC$.


\item \textbf{Potenze, assi radicali e centri radicali.} Il luogo dei punti che hanno la stessa potenza rispetto a due circonferenze è una retta, l'asse radicale delle due circonferenze. Prese $3$ circonferenze $\Gamma_1,\Gamma_2,\Gamma_3$, gli assi radicali $r_{12},r_{23},r_{31}$ concorrono nel centro radicale delle circonferenze.

\end{itemize}


\subsection{Esercizi}
\begin{enumerate}

	\item \emph{[Copiato da GM]} Sia $ABC$ un triangolo con ortocentro $H$ e siano $D$, $E$ e $F$ i piedi delle altezze che cadono sui lati $BC$, $CA$ e $AB$ rispettivamente. Sia $T=EF\cap BC$.
	
	Mostrare che $TH$ è perpendicolare alla mediana condotta da $A$.
	
	\begin{sol}Oltre alla soluzione per inversione, pensavo anche qualcosa con gli assi radicali: il centro radicale delle circonferenze per $AEFH, BCH, ABEF$ è T, quindi TH passa per l'intersezione di AEFH e BCH che chiamo P. Allora APH è retto in quanto diametro.\\
	Poi sia $A'$ il simmetrico di $A$ rispetto $M$ punto medio di $BC$. Allora $BHCA'$ è ciclico e per angoli $\widehat{HPA'}=\widehat{HBA'}=90$.

    inversione di centro A e raggio $AH\cdot HD$. La tesi diventa equivalente a mostrare che la circonferenza per D (immagine di H), per l'intersezione della circoscritta con AEF (immagine di T) e A ha la retta AM come diametro. Questo segue perché in effetti $M,T',A,D$ sono ciclici
	\end{sol}
	
	
	\item Sia $ABC$ un triangolo, $E,F$ i piedi delle altezze su $AC,AB$. Sia $H$ l'ortocentro, $M$ il punto medio di $BC$ e $Q$ l'intersezione più vicina ad $A$ di $HM$ con la circoscritta $\Gamma$. Sia $T=EF\cap BC$. Dimostrare che $T,Q,A$ sono allineati.
	
	\begin{sol} Usare due fatti 1) Il punto Q è l'intersezione di $\Gamma$ con la circonferenza di diametro $AH$ ed è allineato con H,M e il simmetrico di A rispetto O 2) Assi radicali di AQH, ABC, BCEF.
	\end{sol}	
	
	\item Sia $ABC$ un triangolo con $I$ incentro e $I_A$ centro della circonferenza ex-inscritta relativa ad $A$. Sia $\Gamma$ la circonferenza circoscritta ad $ABC$ e sia $M$ il punto medio dell'arco $BC$ non contenente $A$.
	
	Dimostrare che $B,I,C,I_A$ si trovano su una stessa circonferenza di centro $M$
	

	
\begin{sol}Calcolare i segmenti di tangenza di inscritta ed ex-inscritta, poi omotetia in $A$
 
\end{sol}

	
	\item Proprietà varie della circonferenza di Feuerbach. 
\end{enumerate}

\subsection{Problemi}
\begin{enumerate}
    \item \textbf{Polish MO 2018 - 5} Sia $ABC$ un triangolo acutangolo con $AB\neq AC$
    e siano $E,F$ i piedi delle altezze su $AC$ e $AB$. La tangente in $A$ alla circoscritta interseca $BC$ in $P$. La retta parallela a $BC$ passante per $A$ interseca $EF$ in $Q$. 
    
    Dimostrare che $PQ$ è perpendicolare alla mediana passante per $A$ del triangolo $ABC$
    
    \begin{sol}Assi radicali swag: 1) La circonferenza degenere di centro $A$, la circoscritta a $AEF$ e a $BCEF$ hanno $Q$ come centro radicale (in quanto sta su $EF$ per le ultime due e $AQ$ tange la circoscritta $AEF$ per le prime due). 2) $PA^2=PB\cdot PC$, quindi P sta sull'asse radicale tra $A$ e la circoscritta a $BCEF$. Dunque $PQ$ è asse radicale delle due circonferenze ed è perpendicolare alla congiungente dei centri, che è $AM$
    \end{sol}
    
    \item \textbf{RUSSIAN OLYMPIAD 2010} Il triangolo $\triangle ABC$ ha perimetro $4$. I punti $X,Y$ sono sulle semirette $AB,AC$ e sono tali che $AX=AY=1$. I segmenti $BC$ e $XY$ si intersecano in $M$. Dimostrare che o il perimetro di $\triangle ABM$ o il perimetro di $\triangle ACM$ è $2$.
    
    \begin{sol}
    Prendo $U,V$ i simmetrici di $A$ rispetto a $B,C$: sono i punti di tangenza dell'$A-$excerchio. Diciamo che l'excerchio tange $BC$ in $T$. Prendendo circonferenza degenere di centro $A$, la retta $XY$ è asse radicale di quella circonferenza e l'$A-$excerchio. (Quale triangolo ha perimetro $2$ è in base a $T\in MC$ o $T\in BM$)
    \end{sol}
    
    \item \textbf{IMO 2006 - 1} Sia $I$ l'incentro di $\triangle ABC$. Si prenda un punto $P$ interno ad $\triangle ABC$ che soddisfa 
    $$\angle PBA + \angle PCA = \angle PBC + \angle PCB.$$
    Dimostrare che $AP\geq AI$, dove l'uguaglianza vale se e solo se $P$ coincide con $I$.
    
    \begin{sol} 
    Sia $P'$ la seconda intersezione tra $CP$ e $\odot ABC$. $PP'=P'B$ e $\angle BPC = \pi/2 + \alpha/2$, quindi $BPIC$ è ciclico. Il centro di $\odot BPIC$ sta su $AI$, quindi $\angle DIP \leq 90$, da cui $\angle AIP\geq 90$, da cui la tesi.
    \end{sol}
    
    \item \textbf{IMO 2008 - 1} Sia $\triangle ABC$ un triangolo acutangolo con ortocentro $H$. La circonferenza $\Gamma_A$ con centro il punto medio di $BC$ passante per $H$ interseca $BC$ in $A_1,A_2$. Definiamo analogamente $B_1,B_2,C_1,C_2$. Dimostrare che $A_1A_2B_1B_2C_1C_2$ è ciclico.  
    
    \begin{sol}
    Sia $X$ la seconda intersezione di $\Gamma_A$ e $\Gamma_B$. Dimostriamo che $A,X,H$ sono allineati, il che implica $B_1B_2C_1C_2$ ciclico. Il centro della circonferenza è $O$, quindi passa anche per $A_1A_2$.
    \end{sol}
    
\end{enumerate}
