\section{GB - 1 [Metodi Algebrici]}
\subsection{Programmi}
\begin{short}
 Luoghi di punti con la geometria analitica (es: apollonio, luogo degli ortocentri) e scelte opportune di coordinate; distanze con i prodotti scalari e scrittura di vari punti con i vettori; rette e circonferenze con i complessi (e corde e tangenti); applicazioni della trigonometria.
\end{short}



\vspace{0.3cm}
\large{\textbf{Versione estesa - Senior 2019}}\normalsize
\begin{itemize}
 \item Ricapitolazione veloce di analitica. Il piano cartesiano è formato da coppie di punti $(x,y)$. L'equazione di una retta è $y=mx+k$ o $Ax+By+C=0$. Il coefficiente angolare è $m$ e indica la pendenza di una retta; se passa per $(x_1,y_1),(x_2,y_2)$ il coefficente angolare è $m=\frac{y_2-y_1}{x_2-x_1}$. Due rette sono perpendicolari se il prodotto dei coefficienti angolari fa $-1$. Equazione di un cerchio\\
 Più che mostrare/dimostrare le formule, è meglio svolgere qualche esercizio spiegandole sul momento e applicandole direttamente.

\item \emph{Esercizio} Fissati due vertici di un triangolo e facendo variare il terzo sulla circonferenza circoscritta (fissata), qual è il luogo dei punti percorso dall'ortocentro?

%\begin{sol}$B=(0,0),C=(2,0),O=(1,l),A=(a,b)$. 
%$m_{AC}=\frac{b}{a-2}$, $m_{BH}=\frac{2-a}{b}$. $H=(a,\frac{2-a}{b}%a)$.
%Circocerchio: $x^2-2x+y^2-2yl$, sostituisco e ho $H=(a,b-2l)$. Quindi è lo stesso cerchio spostato in basso di $2l$.
%\end{sol}


\item \textbf{Trigonometria} Ricapitolazione del teorema del seno e del coseno. Tanti esempi e applicazioni: Teorema di Stewart, calcolo di segmenti notevoli in un triangolo, ceviane nestate. 

\emph{Esercizio} IMO SL 2015 G1


\item \textbf{Vettori} Prodotto scalare e uso per calcolare distanze, per esempio lunghezze notevoli in un triangolo. Esercizi:

Engel 12 - E6,E7. Teorema di Varignon.


\end{itemize}
\textbf{Complessi}\\
\begin{itemize}
\item Sono numeri della forma $a+bi$, con $a,b$ reali e $i$ tale che $i^2=-1$. Si rappresentano nel piano complessi in maniera simile al piano cartesiano. Parte reale e immaginaria. Somma e moltiplicazione di numeri complessi, (divisione).\\
Scrittura in forma polare, passaggio da forma polare a cartesiana. Coniugato. Moltiplicare=ruotare. 

\textbf{NOTA} Si guardi il programma di G2-Medium per una traccia più approfondita

\item \textbf{Teorema di Napoleone} Costruiti tre triangoli equilateri sui lati di un triangolo $ABC$, dimostrare che i tre centri formano un triangolo equilatero

\item \textbf{Teorema di Vecten} Costruiti sui lati di $ABC$ tre quadrati con centro $O_A,O_B,O_C$, dimostrare che $O_BO_C\perp AO_A$

\item \textbf{IMO 1998-5}


\item Problemi di gare, in cui una furba scelta delle coordinate porta ad una soluzione abbordabile.
\end{itemize}

\subsection{Esercizi}

\textbf{Analitica}
\begin{enumerate}
\item  Dimostrare la formula distanza di un punto di coordinate $(p,q)$ dalla retta di equazione $Ax+By+C=0$:

$$\text{distanza}=\frac{Ap+Bq+C}{\sqrt{A^2+B^2}}$$

\item  Se $\Gamma$ è un cerchio di equazione $f(x,y)=x^2+y^2+ax+by+c=0$ e $P=(s,t)$ un punto del piano, la potenza di $P$ rispetto a $\Gamma$ si ottiene sostituendo $s,t$ nell'equazione della circonferenza:\\
$$\text{Pow}_P(\Gamma)=f(s,t)=s^2+t^2+as+bt+c$$

\item Formule di sdoppiamento

\item Dati due punti $A$ e $B$, trovare il luogo di punti $C$ tali che $AC/BC=\lambda$ costante fissata.\\
\begin{sol}
 Pongo $A=(0,0),B=(1,0)$ e wlog $\lambda < 1$. $\frac{\sqrt{x^2+y^2}}{\sqrt{(x-1)^2+y^2}}=\lambda$.\\
 $\lambda^2 x^2 + \lambda^2 y^2 =x^2-2x+1+y^2$\\
 $ x^2+y^2+\frac{2}{\lambda^2-1}x - \frac{1}{\lambda^2-1}=0$
 è un cerchio centrato in $(-\frac{1}{\lambda^2-1},0)$.
\end{sol}

\item \textbf{Luogo degli ortocentri} Sia $\Gamma$ una circonferenza fissa e sia $BC$ una corda. Sia $A$ un punto su $\Gamma$ e sia $H$ l'ortocentro di $ABC$.\\
Determinare al variare di $A$ su $\Gamma$ il luogo geometrico descritto da $H$.


\suspend{enumerate}
\textbf{Vettori}
\resume{enumerate}
\item \textbf{Engel 12 - E6} $ABCD$ quadrilatero, $AC\perp BD$ se e solo se $AD^2+BC^2=AB^2+CD^2$.\\
\begin{sol}
 In vettori, si vuole $(A-C)\cdot (B-D) = 0$ sse $(A-B)^2+(C-D)^2=(A-D)^2+(B-C)^2$.
\end{sol}

\item \textbf{Engel 12 - E7} $ABCD$ quadrilatero, $M,N,P,Q$ punti medi di $AB,BC,CD,DA$. $AC\perp BD$ se e solo se $MN=PQ$.\\
\begin{sol}
 $M=(A+B)/2$, $MN=PQ$ si traduce come $(\frac{A+B}{2}-\frac{C+D}{2})^2=(\frac{B+C}{2}-\frac{A+D}{2})^2$, facendo i conti esce la stessa cosa.
\end{sol}

\item \textbf{Teorema di Varignon} $ABCD$ quadrilatero, $M,N,P,Q$ punti medi di $AB,BC,CD,DA$. Allora $MNPQ$ è un parallelogramma.



\suspend{enumerate}
\textbf{Trigonometria}
\resume{enumerate}

\item  Calcolare, in termini dei lati e degli angoli del triangolo $ABC$, le seguenti lunghezze: 

$$AH, HH_a , BH_a , H_bH_c , OM_a , OH, AI, IA' , IO$$
dove $H$ è l’ortocentro, $O$ è il circocentro, $I$ l’incentro, $H_a$ la proiezione di $H$ su $BC$ (e similmente
sono definiti $H_b$ e $H_c$ ), $M_a$ il punto medio di $BC$, $A'$ il punto medio dell’arco $BC$ che non contiene
$A$ nella circonferenza circoscritta ad $ABC$.

\item \textbf{Teorema di Stewart} Sia $ABC$ un triangolo e $P$ un punto sul lato $BC$. Dimostrare la seguente formula: 
$$AB^2\cdot PC + AC^2 \cdot BP = BC \cdot BP \cdot PC + AP^2 \cdot BC$$\\
Caso particolare: lunghezza della mediana è $AM=\frac{1}{2}\sqrt{2b^2+2c^2-a^2}$.

\item \textbf{Ceviane nestate} Sia $ABC$ un triangolo, $AD,BE,CF$ ceviane concorrenti e $P,Q,R$ sui lati di $DEF$ in modo che $DP,EQ,FR$ concorrano anch'esse. Allora anche $AP,BQ,CR$ concorrono.

\item  Triangolo $ABC$, sia $Y$ punto su $BC$ tale che $AY=CY$, sia $Z$ sul segmento $AY$ in modo che $AB=CZ$ e infine sia $X=CZ\cap AB$.\\
Dimostrare che $BXYZ$ è ciclico.\\
\begin{sol}
 Let $\angle CAY = \angle ACY = \alpha$, $\angle BAY = \beta$.
By law of sines in $\triangle ACZ$ and $\triangle ABY$:
$\frac{CZ}{\sin \alpha} = \frac{AC}{\sin \angle AZC}$ and $\frac{AY}{\sin(2\alpha+\beta)} = \frac{AB}{\sin 2\alpha} \Rightarrow \frac{CZ \cdot AY}{\sin \alpha \sin (2\alpha + \beta)} = \frac{AC \cdot AB}{\sin \angle AZC \sin 2\alpha} \Rightarrow \frac{AY}{\sin (2\alpha + \beta)}=\frac{AC \sin \alpha}{\sin \angle AZC \sin 2\alpha}$.
But in $\triangle ACY$ we have $AC = \frac{AY \cdot \sin 2\alpha}{\sin \alpha}$. So previous equation implies that $\frac{AY}{\sin(2\alpha + \beta)} = \frac{AY}{\sin \angle AZC}$.
Therefore $\sin \angle AZC = \sin(2\alpha + \beta)$. If $\angle AZC = 180^{\circ}-2\alpha-\beta$, then $\angle ACZ = \alpha + \beta$, which is impossible. So $\angle XZY = \angle AZC = 2\alpha + \beta = 180^{\circ}-\angle XBY$, which completes the proof.
\end{sol}

\item \textbf{Germania BWM 2003, Round 1 - 3} Sia $ABCD$ un parallelogramma, si prendano $X$ sul lato $AB$ e $Y$ su $BC$ in modo che $AX=CY$. Sia $T=AY\cap CX$. Dimostrare che $DT$ biseca l'angolo $\widehat{ADC}$.\\
\begin{sol}
 Teorema dei seni su $AXT$ e $BYT$ per ottenere $\frac{\sin\widehat AXT}{AT}=\frac{\sin\widehat ATX}{AX}=\frac{\sin\widehat BTY}{BY}=\frac{\sin\widehat BYT}{BT}$.
 Poi $\widehat AXT+\widehat TCD=180$, quindi i seni uguali. Si conclude facendo teorema dei seni su $\triangle ATD$ e $\triangle CTD$.\\
 Sintetica: $AY\cap CD=Z$, similitudine e teorema della bisettrice.
\end{sol}





\suspend{enumerate}
\textbf{Complessi}
\resume{enumerate}

\item (\emph{Proposto come fatto in GM1 senza soluzione}) \textbf{Allineamento} $A,B,C$ sono allineati se e solo se 
$$\frac{a-c}{b-c}=\frac{\overline{a}-\overline{c}}{\overline{b}-\overline{c}}$$

\item (\emph{Proposto come fatto in GM1 senza soluzione}) \textbf{Perpendicolarità} $AC\perp BC$ se e solo se 
$$\frac{a-c}{b-c}= - \frac{\overline{a}-\overline{c}}{\overline{b}-\overline{c}}$$


\item \textbf{Eserciziario Senior 17, G2 - 10} (\emph{Proposto in GB2})

\item \textbf{Teorema di Napoleone} Sia $ABC$ un triangolo e si costruisca un triangolo equilatero su ciascuno dei lati di $ABC$, esterno ad esso. Siano $O_A,O_B,O_C$ i centri dei tre triangoli. Dimostrare che:
\begin{itemize}
\item $O_AO_BO_C$ è un triangolo equilatero.
\item le rette $AO_A,BO_B,CO_C$ concorrono.
\end{itemize}


\begin{sol} Siano $A_1,B_1,C_1$ i vertici dei triangoli equilateri. $BA=\sqrt{3}BO_C$ e analogamente per gli altri lati. Una rotazione di 30 centrata in $B$ manda $BO_C$ in $BA$ e $BO_A$ in $BA_1$. Il triangolo $BO'_CO_A'$ è simile a $BAA_1$, quindi per Talete $O_CO_A=O'_CO'_A=\frac{1}{\sqrt{3}}AA_1=O_BO_C$. Quindi i tre lati sono uguali.

Si fa benissimo in complessi (per G1). Su cut-the-knot ci sono tanti approcci.\cite{napoleoncomplex}
\end{sol}
            
\item \textbf{Teorema di Vecten } Sia $ABC$ un triangolo e si costruisca un quadrato su ciascuno dei lati di $ABC$, esterno ad esso. Chiamati $O_A,O_B,O_C$ i centri dei tre quadrati. Dimostrare che:
\begin{itemize}
 \item le rette $AO_A,BO_B,CO_C$ concorrono.
 \item I segmenti $AO_A$ e $O_BO_C$ sono uguali e perpendicolari tra loro
\end{itemize}

\item su Geometry in Figures \cite{engeofigures}, capitolo 9 ci sono fatti sparsi con quadrati/triangoli costruiti sui lati.


\suspend{enumerate}
\textbf{Vario}
\resume{enumerate}




\item \textbf{Eserciziario Senior 2017, G1 - 12} Sia dato un triangolo $ABC$ e si fissino i punti $A'$,$B'$,$C'$ sui lati opposti ai vertici $A$, $B$,
$C$, rispettivamente, in modo che le rette $AA'$ , $BB'$ , $CC'$ siano concorrenti in un punto $P$ interno al triangolo. Sia $d$ il diametro del cerchio circoscritto al triangolo $ABC$, e sia $S'$ l’area del triangolo $A'B'C'$.\\
Dimostrare che $d \cdot S' = AB'\cdot BC'\cdot CA'$.

\item In un triangolo $ABC$, trovare il punto $P$ che minimizza la quantità $AP^2+BP^2+CP^2$.

\begin{sol}
 $AP^2=(x_A-x_P)^2+(y_A-y_P)^2$, quindi posso risolvere separatamente il trovare la coordinata $x_P$ e $y_P$. 
 Per $x_P$ bisogna minimizzare $3x_P^2-2x_P(x_A+x_B+x_C)$, che è una parabola con zeri $x=0, \frac{2}{3}(x_A+x_B+x_c)$, quindi $x_P=\frac{x_A+x_B+x_c}{3}$ e $P$ è il baricentro
\end{sol}

\end{enumerate}


\subsection{GB - 1, Problemi}
\begin{enumerate}
 \item \textbf{EGMO 2013 - 1} Nel triangolo $ABC$, si prolunghi il lato $BC$ dalla parte di $C$ di un segmento $CD$ tale che $CD=BC$. Si prolunghi poi il lato $CA$ dalla parte di $A$ di un segmento $AE$ tale che $AE= 2CA$.Dimostrare che, se $AD=BE$, allora il triangolo $ABC$ è rettangolo
 
 \item \textbf{IMOSL 1998 - 5} Sia $ABC$ un triangolo, $H$ l'ortocentro, $O$ il circocentro e $R$ il raggio della circonferenza circoscritta. Sia $D$ il simmetrico di $A$ rispetto a $BC$, $E$ il simmetrico di $B$ rispetto $AC$ e $F$ il simmetrico di $C$ rispetto $AB$.\\
 Dimostrare che $D,E,F$ sono allineati se e solo se $OH=2R$.

 \begin{sol}
 Complessi con circoscritta = circonferenza unitaria
\end{sol}

\item \textbf{Yugoslavia TST 1992} All'esterno del triangolo $ABC$ sono costruiti i quadrati $BCDE,CAFG,ABHI$. Siano $P,Q$ punti tali che $GCDQ$ e $EBHP$ siano parallelogrammi.\\
Dimostrare che il triangolo $APQ$ è isoscele e rettangolo.

 
 
 \item \textbf{IMOSL 2015 - G1} Sia $ABC$ un triangolo acutangolo con ortocentro $H$. Sia $G$ il punto per cui il quadrilatero $ABGH$ risulta un parallelogrammo. Sia $I$ il punto della retta $GH$ per cui
la retta $AC$ biseca il segmento $HI$. Sia $J$ l’ulteriore intersezione tra la retta $AC$ e la
circonferenza circoscritta al triangolo $GCI$.
Dimostrare che $IJ = AH$.

\begin{sol}
 Sia $M=GH\cap AC$, Teorema dei seni su $\triangle IJM$ da $\frac{\sin\alpha}{IJ}=\frac{\sin IMJ}{IJ}=\frac{\sin IJM}{MH}=\frac{\sin IGC}{MH}$ per la ciclicità di $GCIJ$. Teorema dei seni su $\triangle MAH$ da $\frac{\sin\alpha}{AH}=\frac{\sin CJH}{MH}$. Per la tesi basta dimostrare che $\widehat{CGH}=\widehat{CAH}=90-\gamma$, ma $CHG$ è rettangolo e $CH=c\cdot cotg(\gamma)=HG\cdot cotg(\gamma)$.
\end{sol}

\item \textbf{ITA TST 2016 - 1} Sia $ABCD$ un quadrilatero. Supponiamo che esista un punto $P$ interno al quadrilatero tale che $\angle APD = \angle BPC = 90^{\circ}$ e $PA \cdot PD = PB \cdot PC$. Sia $O$ il circocentro di $\triangle CPD$.\\
Dimostrare che la retta $OP$ passa per il punto medio di $AB$.

\begin{sol}
Trigonometria: 
\end{sol}
 
 
 
 \end{enumerate}



